%%%%%%%%%%%%%%%%%%%%%%%%%%%%%%%%%%%%%%%%%%%%%%%%%%%%%%%%%%%%%%%%%%%%%%%%%%%%%%%%%%%%%%%%%%%%%%%%%%%%%%
% Plantilla básica de Latex en Español.
%
% Autor: Andrés Herrera Poyatos (https://github.com/andreshp)
%
% Es una plantilla básica para redactar documentos. Utiliza el paquete fancyhdr para darle un
% estilo moderno pero serio.
%
% La plantilla se encuentra adaptada al español.
%
%%%%%%%%%%%%%%%%%%%%%%%%%%%%%%%%%%%%%%%%%%%%%%%%%%%%%%%%%%%%%%%%%%%%%%%%%%%%%%%%%%%%%%%%%%%%%%%%%%%%%%

%-----------------------------------------------------------------------------------------------------
%	TEX OPTIONS
%-----------------------------------------------------------------------------------------------------

% !TEX program = pdflatex
% !TEX option = -shell-escape
% !TEX language = es

%-----------------------------------------------------------------------------------------------------
%	INCLUSIÓN DE PAQUETES BÁSICOS
%-----------------------------------------------------------------------------------------------------

\documentclass{article}

%-----------------------------------------------------------------------------------------------------
%	SELECCIÓN DEL LENGUAJE
%-----------------------------------------------------------------------------------------------------

% Paquetes para adaptar Látex al Español:
\usepackage[spanish,es-noquoting, es-tabla, es-lcroman]{babel} % Cambia
\usepackage[utf8]{inputenc}                                    % Permite los acentos.
\selectlanguage{spanish}                                       % Selecciono como lenguaje el Español.

%-----------------------------------------------------------------------------------------------------
%	SELECCIÓN DE LA FUENTE
%-----------------------------------------------------------------------------------------------------

% Fuente utilizada.
\usepackage{courier}                    % Fuente Courier.
\usepackage{microtype}                  % Mejora la letra final de cara al lector.

%-----------------------------------------------------------------------------------------------------
%	LICENCIA
%-----------------------------------------------------------------------------------------------------

\usepackage[
    type={CC},
    modifier={by},
    version={4.0},
]{doclicense}

%-----------------------------------------------------------------------------------------------------
%	ESTILO DE PÁGINA
%-----------------------------------------------------------------------------------------------------

% Estílo de capítulos (usar book en lugar de article).
%\usepackage[Lenny]{fncychap}

% Paquetes para el diseño de página:
\usepackage{fancyhdr}               % Utilizado para hacer títulos propios.
\usepackage{lastpage}               % Referencia a la última página. Utilizado para el pie de página.
\usepackage{extramarks}             % Marcas extras. Utilizado en pie de página y cabecera.
\usepackage[parfill]{parskip}       % Crea una nueva línea entre párrafos.
\usepackage{geometry}               % Asigna la "geometría" de las páginas.

% Se elige el estilo fancy y márgenes de 3 centímetros.
\pagestyle{fancy}
\geometry{left=3cm,right=3cm,top=3cm,bottom=3cm,headheight=1cm,headsep=0.5cm} % Márgenes y cabecera.
% Se limpia la cabecera y el pie de página para poder rehacerlos luego.
%\fancyhfb{}

% Espacios en el documento:
\linespread{1.1}                        % Espacio entre líneas.
\setlength\parindent{0pt}               % Selecciona la indentación para cada inicio de párrafo.

% Cabecera del documento. Se ajusta la línea de la cabecera.
\renewcommand\headrule{
	\begin{minipage}{1\textwidth}
	    \hrule width \hsize
	\end{minipage}
}

% Texto de la cabecera:
\lhead{A. Herrera, N. Rodríguez, J.L. Suárez}                          % Parte izquierda.
\chead{}                                    % Centro.
\rhead{\subject \ - \doctitle}              % Parte derecha.

% Pie de página del documento. Se ajusta la línea del pie de página.
\renewcommand\footrule{
\begin{minipage}{1\textwidth}
    \hrule width \hsize
\end{minipage}\par
}

\lfoot{}                                                 % Parte izquierda.
\cfoot{}                                                 % Centro.
\rfoot{Página\ \thepage\ de\ \protect\pageref{LastPage}} % Parte derecha.

%-----------------------------------------------------------------------------------------------------
%	PORTADA
%-----------------------------------------------------------------------------------------------------

% Elija uno de los siguientes formatos.
% No olvide incluir los archivos .sty asociados en el directorio del documento.
\usepackage{title1}
%\usepackage{title2}
%\usepackage{title3}

%----------------------------------------------------------------------------------------
%   MATEMÁTICAS
%----------------------------------------------------------------------------------------

\usepackage{mathematics}

%----------------------------------------------------------------------------------------
%   GRÁFICOS
%----------------------------------------------------------------------------------------

\usepackage{pgfplots}
\usepackage{tikz}
% Load the library (Descomentar para SOs distintos de Windows)
\usetikzlibrary{external}
% Enable the library !!!>>> MUST be in the preamble <<<!!!!
\tikzexternalize
\pgfplotsset{compat=newest}

\usepackage{float}

%----------------------------------------------------------------------------------------
%   ENLACES
%----------------------------------------------------------------------------------------
\usepackage{hyperref}
\hypersetup{
    colorlinks   = true,   % Quita las cajas y añade un color al texto.
    % Tipos de enlaces cuyo color se puede configurar:
    linkcolor    = [rgb]{0,0.2,0.5},        % Por defecto red
    anchorcolor  = gray,        % Por defecto black
    citecolor    = magenta,     % Por defecto green
    filecolor    = red,         % Por defecto cyan
    menucolor    = green,       % Por defecto red
    runcolor     = red,         % Por defecto cyan
    urlcolor     = cyan        % Por defecto magenta
}

%-----------------------------------------------------------------------------------------------------
%	TÍTULO, AUTOR Y OTROS DATOS DEL DOCUMENTO
%-----------------------------------------------------------------------------------------------------

% Título del documento.
\newcommand{\doctitle}{Apuntes}
% Subtítulo.
\newcommand{\docsubtitle}{}
% Fecha.
\newcommand{\docdate}{\date}
% Asignatura.
\newcommand{\subject}{Inferencia Estadística}
% Autor.
\newcommand{\docauthor}{Andrés Herrera Poyatos \\ Nuria Rodríguez Barroso \\ Juan Luis Suárez Díaz}
\newcommand{\docaddress}{Universidad de Granada}
\newcommand{\docemail}{}%andreshp9@gmail.com}


%-----------------------------------------------------------------------------------------------------
%	RESUMEN
%-----------------------------------------------------------------------------------------------------

% Resumen del documento. Va en la portada.
% Puedes también dejarlo vacío, en cuyo caso no aparece en la portada.
\newcommand{\docabstract}{}
%\newcommand{\docabstract}{En este texto puedes incluir un resumen del documento. Este informa al lector sobre el contenido del texto, indicando el objetivo del mismo y qué se puede aprender de él.}

\begin{document}

\hypersetup{pageanchor=false}
\maketitle
\hypersetup{pageanchor=true}

%-----------------------------------------------------------------------------------------------------
%	ÍNDICE
%-----------------------------------------------------------------------------------------------------

% Profundidad del Índice:
%\setcounter{tocdepth}{1}

\newpage
\tableofcontents
\vspace*{\fill}
\doclicenseThis
\newpage

%-----------------------------------------------------------------------------------------------------
%	SECCIÓN 1
%-----------------------------------------------------------------------------------------------------

\section{Familias de distribuciones}

\subsection{Distribuciones discretas}

En esta sección se desarrollan varias de las distribuciones discretas más importantes de la estadística.

    \subsubsection{Distribución uniforme}
La distribución uniforme es una distribución de probabilidad que asume un número finito de valores con la misma probabilidad. Es fácil comprobar que la función masa de probabilidad es $f(x|n) = \frac{1}{n}$. Claramente $\sum^n_{i=1} \frac{1}{n} = 1$.

La función generatriz de momentos es fácil calcularla y viene definida por $\varphi_X(t) = \frac{e^t (1 - e^tN)}{N(1-e^t)}$. De ella podemos obtener su media y varianza las cuales quedan de la siguiente forma:

\begin{center}
	$E[X] = \frac{N+1}{2}$
	\\$Var(X) =  E[X^2] - (E[X])^2 =  \frac{(N+1)(N-1)}{12}$
\end{center}

\subsubsection{Distribución de Poisson}
Esta distribución expresa, a partir de una frecuencia de ocurrencia media, la probabilidad de que ocurra un determinado número de eventos durante cierto período de tiempo. La función de masa o probabilidad de la distribución de Poisson es $f(x| \lambda) = \frac{e^{-\lambda}{\lambda}^x}{x!}$. \\Claramente $\sum^n_{i=1}  \frac{e^{-\lambda}{\lambda}^x}{x!} = e^{-\lambda} \sum^n_{i=1}  \frac{{\lambda}^x}{x!}  = 1$.


La función generatriz de momentos de dicha distribución se calcula de la siguiente manera $\varphi_X(t) = \sum^n_{i=0} \frac{e^{tx}e^{-\lambda}{\lambda}^x}{x!} = e^{-\lambda} \sum^n_{i=0}   \frac{(e^{t}  \lambda)^x}{x!} =  e^{-\lambda}  e ^{e^{t} \lambda} = e ^{\lambda (e^t -1 )}$.

A partir de la función generatriz de momentos podemos fácilmente deducir la media y la varianza:
\begin{center}
	$E[X] = \lambda $
	\\$Var(X) =  E[X^2] - (E[X])^2 =  \lambda$
\end{center}

\subsubsection{Distribución binomial}

Considérese un experimento de Bernoulli con probabilidad $\theta \in [0,1]$. Repetimos el experimento $n$ veces y nos preguntamos cuál es la probabilidad de que se hayan conseguido $x$ aciertos, donde $x = 0, 1, \ldots, n$. Es fácil ver que esta probabilidad viene dada por $\binom{n}{x} \theta^x (1-\theta)^{n-x}$. Esta cuestión, que es habitual en la estadística, origina la distribución binomial.

\begin{definition}
    Una variable aleatoria sigue una distribución binomial con parámetros $n \in \mathbb{N}$ y $\theta \in [0,1]$  si su función masa de probabilidad viene dada por $f(x|n,\theta) = \binom{n}{x} \theta^x (1-\theta)^{n-x}$. En tal caso se denota $X \sim B(x|n,\theta)$.
\end{definition}

\subsection{Distribuciones continuas}

En esta sección se desarrollan varias de las distribuciones continuas más importantes de la estadística.

\subsubsection{Distribución uniforme}

La distribución uniforme asigna una credibilidad uniforme a todos los puntos de un intervalo $[a,b]$. Esto es, su función de densidad viene dada por
\[f(x|a,b) = \begin{cases}\frac{1}{b-a} \text{ si } x \in [a,b], \\ 0 \text{ en otro caso.}\end{cases}\]
Claramente tenemos que $\int_{-\infty}^{\infty} f(x |a,b) dx = 1$. Además, podemos calcular fácilmente sus momentos como sigue (y, por tanto, también su varianza)
\[E[X^j] = \int_a^b \frac{x^j}{b-a} dx = \frac{b^{j+1} - a^{j+1}}{(b-a) (j+1)},\]
\[Var(X) = E[X^2] - E[X]^2 = \frac{a^2 + ab + b^2}{3} - \frac{(a+b^2)}{4} = \frac{(b-a)^2}{12}.\]

\subsubsection{Distribución normal}

La distribución normal, también llamada distribución gaussiana, es la distribución más importante de la estadística. Esto se debe a sus numerosas aplicaciones en análisis de poblaciones y al teorema central del límite.

\begin{definition}
    Sean $\mu \in \mathbb{R}$ y $\sigma^2 > 0$. Definimos la distribución $N(x | \mu, \sigma^2)$ como la distribución que tiene función de densidad
    \[f(x | \mu, \sigma^2) = \frac{1}{\sqrt{2\pi}\sigma}e^{-(x-\mu)^2 / (2\sigma^2)}, x \in \mathbb{R}.\]
\end{definition}

La distribución normal está bien definida como consecuencia del siguiente lema.
\begin{lem}
    Sean $\mu \in \mathbb{R}$ y $\sigma > 0$. Tenemos que $\int_{-\infty}^\infty e^{-(x-\mu)^2 / (2\sigma^2)}dx = \sqrt{2\pi}\sigma.$
\end{lem}
\begin{proof}
    En primer lugar, vamos a calcular la integral para $\mu = 0$ y $\sigma = 1$. La demostración consiste en reducir el problema en calcular una integral en dos variables. Para ello, elevamos al cuadrado y obtenemos
    \[\left(\int_{-\infty}^\infty e^{-x^2 / 2}dx\right)^2 = \left(\int_{-\infty}^\infty e^{-t^2 / 2}dt\right) \left(\int_{-\infty}^\infty e^{-s^2 / 2}ds \right) = \int_{-\infty}^\infty \int_{-\infty}^\infty e^{-(t^2+s^2) / 2} dt ds. \]
    Resolvemos esta última integral mediante un cambio a polares
    \[\int_{-\infty}^\infty \int_{-\infty}^\infty e^{-(t^2+s^2) / 2} dt ds = \int_{-\pi}^\pi \left(\int_{0}^\infty \rho e^{-\rho^2 / 2} d\rho \right) d\theta = 2\pi \int_{0}^\infty \rho e^{-\rho^2 / 2} d\rho = 2\pi.\]
    Por último, utilizamos el cambio de variable $y = (x - \mu) / \sigma$ para obtener
    \[\int_{-\infty}^\infty e^{-(x-\mu)^2 / (2\sigma^2)}dx = \int_{-\infty}^\infty \sigma e^{-y^2 / 2}dy = \sqrt{2\pi}\sigma. \qedhere\]
\end{proof}

Nótese que si $X \sim N(x | \mu, \sigma^2)$, entonces $Y = (X - \mu)/\sigma$ sigue una distribución $N(x|0,1)$.

\begin{prop} \label{prop:normal:cf}
    La función característica de la distribución $N(x|\mu, \sigma^2)$ viene dada por $\varphi_X(t) = e^{it\mu - t^2 \sigma^2 / 2}$.
\end{prop}
\begin{proof}
    En primer lugar, tenemos que
    \[\varphi_X(t) = E[e^{itX}] = \frac{1}{\sqrt{2\pi}\sigma} \int_{-\infty}^{\infty} e^{itx-(x-\mu)^2 / (2\sigma^2)} dx = \frac{1}{\sqrt{2\pi}\sigma} \int_{-\infty}^{\infty} e^{-((x-\mu)^2 - 2itx\sigma^2) / (2\sigma^2)} dx.\]
    Completamos cuadrados como sigue
    \[(x-\mu)^2 - 2 itx\sigma^2 = (x - (it \sigma^2 + \mu))^2 + t^2 \sigma^4 - 2it\sigma^2 \mu.\]
    Esto sugiere utilizar el cambio de variable $g(y) = y + it \sigma^2$. Obtenemos
    \begin{align*}
        \sqrt{2\pi}\sigma \varphi_X(t) = \int_{-\infty}^{\infty} e^{-((x-\mu)^2 - 2itx\sigma^2) / (2\sigma^2)} = e^{it\mu - t^2 \sigma^2 / 2} \int_{-\infty}^{\infty} e^{-((x-(it \sigma^2 + \mu))^2 ) / (2\sigma^2)} dx \\
        = e^{it\mu - t^2 \sigma^2 / 2} \int_{-\infty}^{\infty} e^{-(y - \mu)^2 ) / (2\sigma^2)} dy = \sqrt{2\pi}\sigma e^{it\mu - t^2 \sigma^2 / 2},
    \end{align*}
    como se quería.
    Nótese que a pesar de ser una integral de contorno compleja el cambio de variable es válido. En efecto, el cambio de variable es afín y la función a integrar es entera. Por tanto, utilizando el camino cerrado $g([0, \infty]) + [\infty, 0]$ se puede probar que el cambio es válido.
\end{proof}

Análogamente se puede probar el siguiente resultado.

\begin{prop} \label{prop:normal:gm}
    La función generatriz de momentos de la distribución $N(x|\mu, \sigma^2)$ viene dada por $\varphi_X(t) = e^{t\mu - t^2 \sigma^2 / 2}$.
\end{prop}

\begin{cor} \label{cor:normal:rec}
    Los momentos de la distribución $N(x|\mu,\sigma^2)$ verifican la ecuación recurrente
    \[E[X^k] = -(k-1)\sigma^2 E[X^{k-2}] + (\mu - t \sigma^2) E[X^{k-1}], \ \  k \ge 2.\]
\end{cor}
\begin{proof}
    Sabemos que $E[X^k] = \varphi_X^{(k)}(t)$. Tenemos $\varphi_X^{(1)}(t) = (\mu - t \sigma^2) \varphi_X(t)$. Consecuentemente,
    \[\varphi_X^{(2)}(t) = -\sigma^2 \varphi_X(t) + (\mu - t \sigma^2) \varphi_X^{(1)}(t).\]
    Por inducción se extiende el resultado fácilmente para $k \ge 2$.
\end{proof}

\begin{cor}
    Si $X \sim N(x|\mu,\sigma^2)$, entonces $E[X] = \mu$ y $E[X^2] = \sigma^2 + \mu^2$. Consecuentemente, $Var(X) = \sigma^2$. Como consecuencia de este resultado al parámetro $\mu$ se le llama media y al parámetro $\sigma^2$ varianza.
\end{cor}

Podemos utilizar los dos corolarios anteriores para calcular los momentos de la distribución normal resolviendo una ecuación recurrente de segundo orden. Evidentemente, la fórmula obtenida será bastante larga. Sin embargo, esta ecuación se simplifica en el caso de los momentos centrados, como pone de manifiesto el siguiente resultado, que se puede demostrar fácilmente por inducción a partir del Corolario \ref{cor:normal:rec}.

\begin{cor}
    Si $X \sim N(x|0,\sigma^2)$, entonces
    \[E[X^k] = \begin{cases} 0 & \text{ si } k \text{ es impar;} \\ (k-1)!! \sigma^{k} & \text{ si } k \text{ es par;} \end{cases}\]
    donde $n!!$ denota al doble factorial, definido como el producto de los números desde $1$ hasta $n$ con la misma paridad que $n$.
\end{cor}

La Figura \ref{fig:normal} muestra la función de densidad de una distribución normal. Podemos ver que la densidad se concentra en torno a la media. De hecho, $P(|X - \mu| \ge 2\sigma) \approx 0.046$. Es más, $P(|X - \mu| \ge 3\sigma) \approx 0.03$.

\begin{figure}[H]
\centering
\begin{tikzpicture}[
     declare function={gauss(\x,\mu,\sigma)=1/(\sigma*sqrt(2*pi))*exp(-((\x-\mu)^2)/(2*\sigma^2));}
    ]
\begin{axis}[
    no markers,
    domain=0:6,
    samples=100,
    ymin=0,
    axis lines*=left,
    xlabel=$x$,
    every axis y label/.style={at=(current axis.above origin),anchor=south},
    every axis x label/.style={at=(current axis.right of origin),anchor=west},
    height=5cm,
    width=15cm,
    xtick=\empty,
    ytick=\empty,
    enlargelimits=false,
    clip=false,
    axis on top,
    grid = major,
    hide y axis
]

\addplot[very thick,cyan!50!black] {gauss(x, 3, 1)};

        \pgfmathsetmacro\valueA{gauss(1,3,1)}
        \pgfmathsetmacro\valueB{gauss(2,3,1)}
        \draw [gray] (axis cs:1,0) -- (axis cs:1,\valueA)
            (axis cs:5,0) -- (axis cs:5,\valueA);
            \draw [gray] (axis cs:2,0) -- (axis cs:2,\valueB)
            (axis cs:4,0) -- (axis cs:4,\valueB);
            \draw [yshift=1.4cm, latex-latex](axis cs:2, 0) -- node [fill=white] {$0.683$} (axis  cs:4, 0);
            \draw [yshift=0.3cm, latex-latex](axis cs:1, 0) -- node [fill=white] {$0.954$} (axis cs:5, 0);

            \node[below] at (axis cs:1, 0)  {$\mu - 2\sigma$};
            \node[below] at (axis cs:2, 0)  {$\mu - \sigma$};
            \node[below] at (axis cs:3, 0)  {$\mu$};
            \node[below] at (axis cs:4, 0)  {$\mu + \sigma$};
            \node[below] at (axis cs:5, 0)  {$\mu + 2\sigma$};
\end{axis}

\end{tikzpicture}
\caption{Función de densidad de una distribución normal.}
\label{fig:normal}
\end{figure}

\begin{prop}
    Sean $X_1$ e $Y_2$ dos variables aleatorias independientes que siguen una distribución $N(x|\mu_1,\sigma_1^2)$ y $N(x|\mu_2,\sigma_2^2)$ respectivamente. Entonces $X+Y$ sigue una distribución $N(x|\mu_1+\mu_2,\sigma_1^2+\sigma_2^2)$.
\end{prop}
\begin{proof}
    Basta darse cuenta de que $\varphi_{X+Y}(t) = \varphi_{X}(t)\varphi_{Y}(t) = e^{t(\mu_1 + \mu_2) - t^2 (\sigma_1^2 + \sigma_2^2) / 2}$ es la función característica asociada a la distribución $N(x|\mu_1+\mu_2,\sigma_1^2+\sigma_2^2)$. Recordemos que la función característica determina de forma unívoca a la distribución.
\end{proof}

El recíproco del resultado anterior también es cierto.

\begin{thm}[Cramer]
    Sean $X$ e $Y$ dos variables aleatorias independientes. Si $X+Y$ es normal, entonces $X$ e $Y$ son normales.
\end{thm}

\subsubsection{Distribución gamma}

La famila de distribuciones gamma se encuentra definida sobre el intervalo $[0, \infty)$. En su definición entra en juego la famosa función gamma, de ahí su nombre.

\begin{definition}
    Se define la función gamma como la aplicación $\Gamma: (0, \infty) \to (0, \infty)$ dada por
    \[\Gamma(\alpha) = \int_0^\infty t^{\alpha-1}e^{-t}dt.\]
\end{definition}
\begin{prop}
    La función gamma está bien definida.
\end{prop}
\begin{proof}
    Sea $\alpha > 0$. Tenemos que probar que $\int_0^\infty t^{\alpha-1}e^{-t}dt < \infty$. Tomando $b > 0$, escribimos
    \[\int_0^\infty t^{\alpha-1}e^{-t}dt = \int_0^b t^{\alpha-1}e^{-t}dt + \int_b^\infty t^{\alpha-1}e^{-t}dt.\]
    Sabemos que la función $t^{\alpha - 1}$ tiene a $t^{\alpha} / \alpha$ como primitiva y, por tanto, es integrable en $[0,b]$. Puesto que $t^{\alpha-1}e^{-t} \le t^{\alpha-1}$, obtenemos que $t^{\alpha-1}e^{-t}$ es integrable en $[0,b]$. Por otro lado tenemos que
    \[\lim_{t \to \infty} \frac{t^{\alpha-1}e^{-t}}{e^{-t / 2}} = 0.\]
    Consecuentemente, para cierto $b > 0$ se verifica $t^{\alpha-1}e^{-t} \le e^{-t / 2}$ para todo $t \ge b$. Puesto que $e^{-t / 2}$ es integrable en $[b, \infty)$, deducimos que $t^{\alpha-1}e^{-t}$ también lo es, lo que termina la demostración.
\end{proof}

\begin{prop}[Propiedades de la función gamma]
    Sea $\alpha > 0$. Se verifica:
    \begin{enumerate}
        \item $\Gamma(1) = 1$;
	\item $\Gamma(\alpha+1) = \alpha\Gamma(\alpha)$;
    \item $\Gamma(n+1) = n!$ para cualquier $n \in \mathbb{N}$;
    \item (Fórmula de reflexión de Euler) si $0 < \alpha < 1$, entonces $\Gamma(\alpha) \Gamma(1- \alpha) = \frac{\pi}{\sin(\alpha \pi)}$;
    \item $\Gamma(1/2) = \sqrt{\pi}$;
    \item $\Gamma(\alpha) = \beta^\alpha\int_0^\infty t^{\alpha-1}e^{-\beta t} dt$ para todo $\beta > 0$.
    \end{enumerate}
\end{prop}
\begin{proof}
    \
    \begin{enumerate}
        \item Es fácil ver que $\int_{0}^\infty e^{-t} dt = 1$.
        \item Integrando por partes obtemos
        \[\Gamma(\alpha+1) = \int_0^{\infty}{t^{\alpha}e^{-t}dt} = \bigg[-e^{-t}t^\alpha\bigg]_0^{\infty} +
            \int_0^{\infty}{xt^{\alpha-1}e^{-t}dt} = \alpha\int_0^{\infty}{t^{\alpha-1}e^{-t}dt} = \alpha\Gamma(\alpha).\]
        \item Es consecuencia directa de los apartados a) y b).
        \item Se obtiene utilizando definiciones alternativas de la función gamma tras extenderla a $\mathbb{C} \setminus \mathbb{Z}^-_0$. Para más información véase \cite{gamma}. No desarrollamos esta demostración pues solo la necesitamos para el siguiente apartado.
        \item Se obtiene al evaluar la fórmula de reflexión en $\alpha = 1/2$.
        \item Se obtiene realizando el cambio de variable $t = \beta s$. \qedhere
    \end{enumerate}
\end{proof}

\begin{definition}
    Sean $\alpha, \beta > 0$. Definimos la distribución $Gamma(x | \alpha, \beta)$ como la distribución que tiene función de densidad
    \[f(x | \alpha, \beta) = \frac{\beta^\alpha}{\Gamma(\alpha)}x^{\alpha-1}e^{-\beta x}, x > 0.\]
\end{definition}

El parámetro $\alpha$ se conoce como parámetro de forma ya que influencia la forma de la distribución, como muestra el siguiente resultado.

\begin{prop}
    La función de densidad de la distribución $Gamma(\alpha, \beta)$ verifica las siguientes propiedades:
    \begin{itemize}
        \item Si $0< \alpha <1$, entonces $f(x | \alpha, \beta)$ es decreciente y $f(x) \to \infty$ para $x \to 0$.
        \item Si $\alpha = 1$, entonces $f(x | \alpha, \beta)$ es decreciente con $f(0) = 1$.
        \item Si $\alpha > 1$, entonces $f(x | \alpha, \beta)$ crece en $[0, (\alpha-1) / \beta]$ y decrece en $[(\alpha-1) / \beta,\infty]$.
        \item Si $0 < \alpha \le 1$, entonces $f(x | \alpha, \beta)$ es convexa.
        \item Si $1 < \alpha \le 2$, entonces $f(x | \alpha, \beta)$ es cóncava en $[0,(\alpha-1 + \sqrt{\alpha - 1}) / \beta]$ y convexa en $[(\alpha-1 + \sqrt{\alpha - 1}) / \beta, \infty]$.
        \item Si $2 < \alpha$, entonces $f(x | \alpha, \beta)$ es cóncava en $[(\alpha-1 - \sqrt{\alpha - 1}) / \beta,(\alpha-1 + \sqrt{\alpha - 1}) / \beta]$ y convexa en $[0, (\alpha-1 - \sqrt{\alpha - 1}) / \beta]$ y $[(\alpha-1 + \sqrt{\alpha - 1}) / \beta, \infty]$.
    \end{itemize}
\end{prop}
\begin{proof}
Los resultados se obtienen mediante las herramientas habituales del cálculo. Basta estudiar la derivada primera y la derivada segunda
\begin{align*}
f^\prime(x) &= \frac{\beta^\alpha}{\Gamma(\alpha)} x^{\alpha-2} e^{-\beta x}[(\alpha - 1) - \beta x]; \\
f^{\prime \prime}(x) &= \frac{\beta^\alpha}{\Gamma(\alpha)} x^{\alpha-3} e^{-\beta x} \left[(\alpha - 1)(\alpha - 2) - 2 \beta (\alpha - 1) x + \beta^2 x^2\right]. \qedhere
\end{align*}
\end{proof}

La Figura \ref{fig:gamma:alpha} muestra la función de densidad de la distribución gamma para distintos valores de $\alpha$.  El parámetro $\beta$ se denomina parámetro de escala debido a su influencia en la escala de la función de densidad. La Figura \ref{fig:gamma:beta} muestra la función de densidad de la distribución gamma para distintos valores de $\beta$.

\begin{figure}[H]
    \centering
    \begin{tikzpicture}[
        declare function={gamma(\z)=
        2.506628274631*sqrt(1/\z)+ 0.20888568*(1/\z)^(1.5)+ 0.00870357*(1/\z)^(2.5)- (174.2106599*(1/\z)^(3.5))/25920- (715.6423511*(1/\z)^(4.5))/1244160)*exp((-ln(1/\z)-1)*\z;},
        declare function={gammapdf(\x,\k,\theta) = 1/(\theta^\k)*1/(gamma(\k))*\x^(\k-1)*exp(-\x/\theta);}
        ]
        \begin{axis}[
            ylabel={$f(x | \alpha, \beta)$},
            domain=0.000001:10, samples=100,
            axis lines=left,
            every axis y label/.style={at=(current axis.above origin),anchor=east},
            every axis x label/.style={at=(current axis.right of origin),anchor=north},
            height=6cm, width=12cm,
            enlargelimits=false,
            clip=false,
            axis on top,
            grid = major,
            legend pos=outer north east
        ]
            \addplot [very thick,cyan!20!black] {gammapdf(x,0.98,2)};
            \addlegendentry{$\alpha = 0.9, \beta = 2$}
            \addplot [very thick,cyan!35!black] {gammapdf(x,1,2)};
            \addlegendentry{$\alpha = 1, \beta = 2$}
            \addplot [very thick,cyan!60!black] {gammapdf(x,2,2)};
            \addlegendentry{$\alpha = 2, \beta = 2$}
            \addplot [very thick,cyan] {gammapdf(x,3,2)};
            \addlegendentry{$\alpha = 3, \beta = 2$}
        \end{axis}
    \end{tikzpicture}
    \caption{Densidad de la distribución gamma con distintos valores de $\alpha$.}
    \label{fig:gamma:alpha}
\end{figure}

\begin{figure}[H]
    \centering
    \begin{tikzpicture}[
        declare function={gamma(\z)=
        2.506628274631*sqrt(1/\z)+ 0.20888568*(1/\z)^(1.5)+ 0.00870357*(1/\z)^(2.5)- (174.2106599*(1/\z)^(3.5))/25920- (715.6423511*(1/\z)^(4.5))/1244160)*exp((-ln(1/\z)-1)*\z;},
        declare function={gammapdf(\x,\k,\theta) = 1/(\theta^\k)*1/(gamma(\k))*\x^(\k-1)*exp(-\x/\theta);}
        ]
        \begin{axis}[
            ylabel={$f(x | \alpha, \beta)$},
            domain=0:10, samples=100,
            axis lines=left,
            every axis y label/.style={at=(current axis.above origin),anchor=east},
            every axis x label/.style={at=(current axis.right of origin),anchor=north},
            height=6cm, width=12cm,
            enlargelimits=false, clip=false, axis on top,
            grid = major,
            legend pos=outer north east
        ]
            \addplot [very thick,magenta!10!black] {gammapdf(x,2,0.5)};
            \addlegendentry{$\alpha = 2, \beta = 0.5$}
            \addplot [very thick,magenta!40!black] {gammapdf(x,2,1)};
            \addlegendentry{$\alpha = 2, \beta = 1$}
            \addplot [very thick,magenta!75!black] {gammapdf(x,2,2)};
            \addlegendentry{$\alpha = 2, \beta = 2$}
            \addplot [very thick,magenta] {gammapdf(x,2,3)};
            \addlegendentry{$\alpha = 2, \beta = 3$}
        \end{axis}
    \end{tikzpicture}
    \caption{Densidad de la distribución gamma con distintos valores de $\beta$.}
    \label{fig:gamma:beta}
\end{figure}

\begin{prop} \label{prop:gamma:cf}
    La función característica de la distribución $Gamma(x|\alpha, \beta)$ viene dada por $\varphi_X(t) = \left(\frac{\beta}{\beta -it}\right)^\alpha$.
\end{prop}
\begin{proof}
    Basta utilizar el cambio de variable $g(y) = y / (\beta - it)$ como sigue
    \[E[e^{itX}] = \frac{\beta^\alpha}{\Gamma(\alpha)} \int_{0}^{\infty} x^{\alpha-1}e^{-(\beta - it) x} dx = \frac{\beta^\alpha}{\Gamma(\alpha) (\beta -it)^\alpha} \int_{0}^{\infty} y^{\alpha-1}e^{-y} dy = \left(\frac{\beta}{\beta -it}\right)^\alpha.\]
    Nótese que a pesar de ser una integral de contorno compleja el cambio de variable afín es válido como se comentó en la Proposición \ref{prop:normal:cf}.
\end{proof}

\begin{cor}
    El momento $k$-ésimo de la distribución $Gamma(x|\alpha,\beta)$ es $\alpha (\alpha+1) \ldots (\alpha + k -1) / \beta^k$.
\end{cor}
\begin{proof}
    Tenemos que $i^k E[X^k]= \varphi_X^{(k)}(t) = i^k \alpha (\alpha+1) \ldots (\alpha + k -1) / \beta^k$.
\end{proof}

\begin{prop}
    La función generatriz de momentos de la distribución $Gamma(x|\alpha, \beta)$ viene dada por $\varphi_X(t) = \left(\frac{\beta}{\beta - t}\right)^\alpha$.
\end{prop}
\begin{proof}
    La demostración es análoga a la dada en la Proposición \ref{prop:gamma:cf}.
\end{proof}

\begin{cor}
    La distribución $Gamma(x|\alpha,\beta)$ tiene media $\alpha / \beta$ y varianza $\alpha / \beta^2$.
\end{cor}

\begin{prop}
    Sea $n \ge 1$. Consideremos $X_1, \ldots, X_n$ variables aleatorias independientes tales que $X_j$ sigue una distribución $Gamma(x|\alpha_i, \beta)$. Entonces, $\sum_{i=1}^n X_j$ sigue una distribución $Gamma(x|\sum_{i=1}^n \alpha_i, \beta)$.
\end{prop}
\begin{proof}
    En primer lugar, calculamos la función característica de $\sum_{i=1}^n X_j$ como sigue
    \[E[e^{i\sum X_j}] = E[\prod e^{iX_j}] = \prod E[e^{iX_j}] = \left(\frac{\beta}{\beta - it}\right)^{\sum \alpha_j},\]
    donde se ha utilizado que la esperanza del producto de dos variables aleatorias independientes es el producto de las esperanzas. Por último, nótese que la función característica de la variable $\sum X_j$ es la función característica de $Gamma(x|\sum_{i=1}^n \alpha_i, \beta)$. El hecho de que la función característica de una distribución la determina de forma unívoca finaliza la prueba.
\end{proof}

\begin{prop} \label{prop:normal-square}
    Sea $X \sim N(x|0,\sigma^2)$. La variable aleatoria $Y = X^2$ sigue una distribución \\ $Gamma(y,1/2,1/(2\sigma^2))$. En particular, para $\sigma = 1$, $Y = X^2$ sigue una distribución $\chi^2_1$.
\end{prop}
\begin{proof}
    Sean $F$ y $G$ las funciones de distribución de las variables $X$ e $Y$ respectivamente. Tenemos que $G(y) = P(X^2 \le y) = P(- \sqrt{y} \le X \le \sqrt{y}) = F(\sqrt{y}) - F(-\sqrt{y})$. Derivando, obtenemos
    \[G'(y) = \frac{F'(\sqrt{y}) + F'(-\sqrt{y})}{2\sqrt{y}} = \frac{1}{\sqrt{y}} \frac{1}{\sqrt{2\pi}\sigma} e^{-y/(2\sigma^2)} = \frac{(1/(2\sigma^2))^{1/2}}{\Gamma(1/2)} y^{-1/2} e^{-y/(2\sigma^2)}.\]
    Por último, basta darse cuenta de que $G'(y)$ es la función de densidad de $Gamma(y,1/2,1/(2\sigma^2))$.
\end{proof}

\subsubsection{Distribución beta}

La famila de distribuciones beta se encuentra definida sobre el intervalo $(0, 1)$. En su definición entra en juego la denominada función beta, de ahí su nombre.

\begin{definition}
    Se define la función beta como la aplicación $\beta : (0, \infty) \times (0, \infty) \to (0, \infty)$ dada por
    \[\beta(x, y) = \int_0^1 t^{x-1}(1-t)^{y-1}\,dt.\]
\end{definition}
\begin{prop} \label{prop:beta-gamma}
    Para cada $x, y > 0$ se tiene que $\frac{\Gamma(x)\Gamma(y)}{\Gamma(x+y)} = \beta(x,y)$. Como consecuencia, la función beta está bien definida.
\end{prop}
\begin{proof}
    En primer lugar escribimos $\Gamma(x)\Gamma(y)$ como una integral doble
    \begin{equation*}
        \Gamma(x)\Gamma(y) =\int_{0}^{\infty }\ e^{-u}u^{x-1}\,d u\int_{0}^{\infty }\ e^{-v}v^{y-1}\,d v
        =\int_{0}^{\infty }\int_{0}^{\infty }\ e^{-u-v}u^{x-1}v^{y-1}\,d u\,d v.
    \end{equation*}
    La expresión anterior nos sugiere utilizar el cambio de variable $(u, v) = J(t,s) = (st, (1-t)s)$. Nótese que $|J(t,s)| = s$. Aplicamos el cambio a continuación
    \begin{gather*}
        \Gamma(x)\Gamma(y) = \int_{0}^{\infty} \left( \int_{0}^{1}e^{-s}(st)^{x-1}(s(1-t))^{y-1}|J(t,s)|\, d t \right) ds \\
        = \int_{0}^{\infty }e^{-s}s^{x+y-2}s \left(\int_{0}^{1}t^{x-1}(1-t)^{y-1}\,dt \right) d s =\Gamma(x+y)\beta(x,y).  \qedhere
    \end{gather*}
\end{proof}

En la práctica siempre se utiliza la función gamma para evaluar la función beta. Ya podemos definir la distribución beta.

\begin{definition}
    Sean $p, q > 0$. Definimos la distribución $beta(x | p, q)$ como la distribución que tiene función de densidad
    \[f(x | p, q) = \frac{1}{\beta(p,q)}x^{p-1}(1-x)^{q-1}, 0 < x < 1.\]
\end{definition}

Claramente, la función de densidad integra $1$. Esta distribución asigna probabilidad $1$ al intervalo $(0,1)$. Por ello, es útil en modelos de proporciones. Las Figuras \ref{fig:beta:p} y \ref{fig:beta:q} muestran la distribución beta cambiando los valores $p$ y $q$ respectivamente. Podemos observar que las funciones de densidad de $beta(x|p,q)$ y $beta(x|q,p)$ son simétricas respecto del punto $1/2$. Esto se puede demostrar fácilmente a partir de la definición. La Figura \ref{fig:beta:pq} muestra la distribución beta con iguales valores de $p$ y $q$. Vemos que las densidades son simétricas en el eje $x = 1/2$, hecho que también puede demostrarse fácilmente a partir de la definición.

\begin{figure}[H]
    \centering
    \begin{tikzpicture}[
        declare function={gamma(\z)=
        2.506628274631*sqrt(1/\z)+ 0.20888568*(1/\z)^(1.5)+ 0.00870357*(1/\z)^(2.5)- (174.2106599*(1/\z)^(3.5))/25920- (715.6423511*(1/\z)^(4.5))/1244160)*exp((-ln(1/\z)-1)*\z;},
        declare function={betapdf(\x,\p,\q) = gamma(\p+\q)/(gamma(\p)*gamma(\q)) * \x^(\p-1)*(1-x)^(\q-1);}
        ]
        \begin{axis}[
            ylabel={$f(x | p, q)$},
            domain=0.05:1, samples=100,
            axis lines=left,
            every axis y label/.style={at=(current axis.above origin),anchor=east},
            every axis x label/.style={at=(current axis.right of origin),anchor=north},
            height=6cm, width=12cm,
            enlargelimits=false, clip=false, axis on top,
            grid = major,
            legend pos=outer north east
        ]
            \addplot [very thick,cyan!10!black] {betapdf(x,0.5,2)};
            \addlegendentry{$p = 0.5, q = 2$}
            \addplot [very thick,cyan!40!black] {betapdf(x,1,2)};
            \addlegendentry{$p = 1, q = 2$}
            \addplot [very thick,cyan!75!black] {betapdf(x,2,2)};
            \addlegendentry{$p = 2, q = 2$}
            \addplot [very thick,cyan] {betapdf(x,3,2)};
            \addlegendentry{$p = 3, q = 2$}
        \end{axis}
    \end{tikzpicture}
    \caption{Densidad de la distribución beta con distintos valores de $p$.}
    \label{fig:beta:p}
\end{figure}

\begin{figure}[H]
    \centering
    \begin{tikzpicture}[
        declare function={gamma(\z)=
        2.506628274631*sqrt(1/\z)+ 0.20888568*(1/\z)^(1.5)+ 0.00870357*(1/\z)^(2.5)- (174.2106599*(1/\z)^(3.5))/25920- (715.6423511*(1/\z)^(4.5))/1244160)*exp((-ln(1/\z)-1)*\z;},
        declare function={betapdf(\x,\p,\q) = gamma(\p+\q)/(gamma(\p)*gamma(\q)) * \x^(\p-1)*(1-x)^(\q-1);}
        ]
        \begin{axis}[
            ylabel={$f(x | p, q)$},
            domain=0:0.95, samples=100,
            axis lines=left,
            every axis y label/.style={at=(current axis.above origin),anchor=east},
            every axis x label/.style={at=(current axis.right of origin),anchor=north},
            height=6cm, width=12cm,
            enlargelimits=false, clip=false, axis on top,
            grid = major,
            legend pos=outer north east
        ]
            \addplot [very thick,magenta!10!black] {betapdf(x,2,0.5)};
            \addlegendentry{$p = 2, q = 0.5$}
            \addplot [very thick,magenta!40!black] {betapdf(x,2,1)};
            \addlegendentry{$p = 2, q = 1$}
            \addplot [very thick,magenta!75!black] {betapdf(x,2,2)};
            \addlegendentry{$p = 2, q = 2$}
            \addplot [very thick,magenta] {betapdf(x,2,3)};
            \addlegendentry{$p = 2, q = 3$}
        \end{axis}
    \end{tikzpicture}
    \caption{Densidad de la distribución beta con distintos valores de $q$.}
    \label{fig:beta:q}
\end{figure}

\begin{figure}[H]
    \centering
    \begin{tikzpicture}[
        declare function={gamma(\z)=
        2.506628274631*sqrt(1/\z)+ 0.20888568*(1/\z)^(1.5)+ 0.00870357*(1/\z)^(2.5)- (174.2106599*(1/\z)^(3.5))/25920- (715.6423511*(1/\z)^(4.5))/1244160)*exp((-ln(1/\z)-1)*\z;},
        declare function={betapdf(\x,\p,\q) = gamma(\p+\q)/(gamma(\p)*gamma(\q)) * \x^(\p-1)*(1-x)^(\q-1);}
        ]
        \begin{axis}[
            ylabel={$f(x | p, q)$},
            domain=0.05:0.95, samples=100,
            axis lines=left,
            every axis y label/.style={at=(current axis.above origin),anchor=east},
            every axis x label/.style={at=(current axis.right of origin),anchor=north},
            height=6cm, width=12cm,
            enlargelimits=false, clip=false, axis on top,
            grid = major,
            legend pos=outer north east
        ]
            \addplot [very thick,green!10!black] {betapdf(x,0.5,0.5)};
            \addlegendentry{$p = 0.5, q = 0.5$}
            \addplot [very thick,green!40!black] {betapdf(x,1,1)};
            \addlegendentry{$p = 1, q = 1$}
            \addplot [very thick,green!75!black] {betapdf(x,2,2)};
            \addlegendentry{$p = 2, q = 2$}
            \addplot [very thick,green] {betapdf(x,3,3)};
            \addlegendentry{$p = 3, q = 3$}
        \end{axis}
    \end{tikzpicture}
    \caption{Densidad de la distribución beta con $p = q$.}
    \label{fig:beta:pq}
\end{figure}

\begin{prop}
    La función característica de la distribución $beta(x|p,q)$ viene dada por
    \[\varphi_X(t) = 1 + \sum_{k = 1}^\infty \frac{(it)^k}{k!} \frac{\beta(p+k,q)}{\beta(p,q)}.\]
\end{prop}

\begin{proof}
Desarrollamos la función característica utilizando el desarrollo de la exponencial
\begin{gather*}
E[e^{itX}] = \frac{1}{\beta(p,q)}\int_0^1 e^{itx} x^{p-1}(1-x)^{q-1} dx = \frac{1}{\beta(p,q)}\int_0^1 \sum_{k = 0}^\infty \frac{(itx)^{k}}{k!} x^{p-1}(1-x)^{q-1} dx \\
= \frac{1}{\beta(p,q)}\sum_{k = 0}^\infty \frac{(it)^k}{k!} \int_0^1x^{p+k-1}(1-x)^{q-1} dx = \sum_{k = 0}^\infty \frac{(it)^k}{k!} \frac{\beta(p+k,q)}{\beta(p,q)} = 1 + \sum_{k = 1}^\infty \frac{(it)^k}{k!} \frac{\beta(p+k,q)}{\beta(p,q)}. \qedhere
\end{gather*}
\end{proof}

\begin{cor} \label{cor:beta:moments}
    Sea $X \sim beta(x|p,q)$. Entonces, para cada $k \ge 1$ se tiene que
    \[E[X^k] = \frac{\beta(p+k,q)}{\beta(p,q)}.\]
\end{cor}

\begin{cor} \label{cor:beta:esp}
    Sea $X \sim beta(x|p,q)$. Entonces, $E[X] = \frac{p}{p+q}$ y $Var(X) = \frac{pq}{(p+q)^2(p+q+1)}$.
\end{cor}
\begin{proof}
    Por el Corolario \ref{cor:beta:moments} y la Proposición \ref{prop:beta-gamma} tenemos que
    \[E[X] = \frac{\beta(p+1,q)}{\beta{p,q}} = \frac{\Gamma(p+1)\Gamma(q)}{\Gamma(p+q)} \frac{\Gamma(p+q)}{\Gamma(p)\Gamma(q)} = \frac{\Gamma(p+1)}{\Gamma(p)} \frac{\Gamma(p+q)}{\Gamma(p+q+1)} = \frac{p}{p+q}\]
    y
    \[E[X^2] = \frac{\beta(p+2,q)}{\beta{p,q}} = \frac{\Gamma(p+2)\Gamma(q)}{\Gamma(p+q+2)} \frac{\Gamma(p+q)}{\Gamma(p)\Gamma(q)} = \frac{\Gamma(p+2)}{\Gamma(p)} \frac{\Gamma(p+q)}{\Gamma(p+q+2)} = \frac{p(p+1)}{(p+q)(p+q+1)}.\]
    Por último, es directo calcular $Var(X) = E[X^2] - E[X]^2$.
\end{proof}

\subsubsection{Distribución de Cauchy}

\begin{definition}
    Sea $\mu \in \mathbb{R}$ y $\sigma > 0$. Definimos la distribución $Cauchy(x | \mu, \sigma)$ como la distribución que tiene función de densidad
    \[f(x | \mu, \sigma) = \frac{1}{\sigma \pi} \frac{1}{1+\left(\frac{x-\mu}{\sigma}\right)^2} = \frac{\sigma}{\pi(\sigma^2 + (x-\mu)^2)}, \ x \in \mathbb{R}.\]
\end{definition}

La distribución está bien definida. En efecto, utilizando el cambio de variable $x = \sigma y + \mu$ obtenemos
\[\int_{-\infty}^\infty \frac{1}{1+\left(\frac{x-\mu}{\sigma}\right)^2} dx = \int_{-\infty}^\infty \frac{\sigma}{1+y^2} dy = \sigma \pi.\]

\begin{prop}
    La función característica de la distribución $Cauchy(x|\mu,\sigma)$ viene dada por
    \[\varphi_X(t) = e^{i\mu t - \sigma |t|}.\]
\end{prop}
\begin{proof}
    En primer lugar, demostramos el resultado para $Cauchy(x|0,1)$. Tenemos que
    \[\varphi_X(t)  = \frac{1}{\pi}\int_{-\infty}^\infty \frac{e^{itz}}{1+z^2} dz = e^{-|t|},\]
    donde la última igualdad se explica en \cite{cauchy}. Ahora, si $X \sim Cauchy(x|\mu,\sigma)$, entonces $Y = (X - \mu) / \sigma \sim Cauchy(x|0,1)$ y, por tanto, obtenmos
    \[\varphi_X(t) = E[e^{itX}] = E[e^{it(\sigma Y+\mu)}] = e^{it\mu} \varphi_Y(\sigma t) = e^{i\mu t - \sigma |t|}. \qedhere\]
\end{proof}

Nótese que la función característica de la distribución de Cauchy no es diferenciable en $0$. Consecuentemente, esta distribución no tiene momentos de orden mayor o igual que 1. El recíproco no sería cierto, esto es, existen distribuciones que no tienen esperanza y su función característica es diferenciable en $0$ \cite{char}.

\begin{prop}
    Sean $X$ e $Y$ dos variables aleatorias independientes con distribuciones $Cauchy(x|\mu_1, \sigma_1)$ y $Cauchy(x|\mu_2, \sigma_2)$ respectivamente. Entonces, $X+Y \sim Cauchy(x|\mu_1+\mu_2,\sigma_1+\sigma_2)$.
\end{prop}
\begin{proof}
    Nótese que $\varphi_{X+Y}(t) = \varphi_{X}(t)\varphi_{Y}(t) = e^{it(\mu_1+\mu_2) - |t| (\sigma_1+\sigma_2)}$. La prueba finaliza al darse cuenta de que ésta es la función característica de $Cauchy(x|\mu_1+\mu_2,\sigma_1+\sigma_2)$.
\end{proof}

\begin{figure}[H]
    \begin{tikzpicture}[
            declare function={gauss(\x,\mu,\sigma)=1/(\sigma*sqrt(2*pi))*exp(-((\x-\mu)^2)/(2*\sigma^2));},
            declare function={cauchy(\x,\mu,\sigma) = 1/((\sigma*pi)*(1+((\x-\mu)/\sigma)^2));}
        ]
        \begin{axis}[
            domain=-5:5, samples=100,
            axis lines=left,
            every axis y label/.style={at=(current axis.above origin),anchor=east},
            every axis x label/.style={at=(current axis.right of origin),anchor=north},
            height=6cm, width=12cm,
            enlargelimits=false, clip=false, axis on top,
            grid = major,
            legend pos=outer north east
        ]
            \addplot [very thick,cyan!70!black] {cauchy(x,0,1)};
            \addlegendentry{Cauchy, $\mu = 0, \sigma = 1$}
            \addplot [very thick,cyan!40!black] {cauchy(x,0,2)};
            \addlegendentry{Cauchy, $\mu = 0, \sigma = 2$}
            \addplot [very thick,magenta!75!black] {gauss(x,0,1)};
            \addlegendentry{Normal, $\mu = 0, \sigma = 1$}
        \end{axis}
    \end{tikzpicture}
    \caption{Densidad de la distribución de Cauchy comparada con la distribución normal.}
    \label{fig:cauchy}
\end{figure}


\subsubsection{Distribución de Laplace}

\begin{definition}
    Sea $\mu \in \mathbb{R}$ y $\sigma > 0$. Definimos la distribución de Laplace, y la denotamos $Laplace(x | \mu, \sigma)$ como la distribución que tiene función de densidad
    \[f(x | \mu, \sigma) = \frac{1}{2 \sigma} e^{-|x - \mu| / \sigma}, \ x \in \mathbb{R}.\]
\end{definition}

\begin{figure}[H]
    \begin{tikzpicture}[
         declare function={gauss(\x,\mu,\sigma)=1/(\sigma*sqrt(2*pi))*exp(-((\x-\mu)^2)/(2*\sigma^2));},
         declare function={laplace(\x,\mu,\sigma) = exp(-abs(\x-\mu) / \sigma)/(2*\sigma);}
        ]
        \begin{axis}[
            domain=-5:5,
            samples=200,
            axis lines=left,
            every axis y label/.style={at=(current axis.above origin),anchor=east},
            every axis x label/.style={at=(current axis.right of origin),anchor=north},
            height=6cm, width=12cm,
            enlargelimits=false,
            clip=false,
            axis on top,
            grid = major,
            legend pos=outer north east
        ]
            \addplot [very thick,cyan!70!black] {laplace(x,0,1)};
            \addlegendentry{Laplace, $\mu = 0, q = 1$}
            \addplot [very thick,cyan!40!black] {laplace(x,0,2)};
            \addlegendentry{Laplace, $\mu = 0, q = 2$}
            \addplot [very thick,magenta!80!black] {gauss(x,0,1)};
            \addlegendentry{Normal, $\mu = 0, q = 1$}
        \end{axis}
    \end{tikzpicture}
    \caption{Densidad de la distribución de Laplace comparada con la densidad de la distribución normal.}
    \label{fig:laplace}
\end{figure}

\subsubsection{Distribución T de Student}

\subsubsection{Distribución de Dirichlet}

\section{Estimación de parámetros} \label{sec:estimacion}

Supongamos que estamos estudiando un fenómeno aleatorio que sabemos que sigue una distribución $f(X | \theta_0)$, donde $\theta_0 \in \Theta$ es un parámetro que no es conocido. Nuestro objetivo es estimar el parámetro $\theta_0$ a partir de una muestra $x_1, \ldots, x_n$. Para ello buscamos una función $T_n$ de manera que podamos decir $\theta_0 \approx T_n(x_1, \ldots, x_n)$.

\begin{definition}
    Un estimador puntual es una función medible $T_n(X_1, \ldots, X_n)$ que toma valores en $\Theta$, donde $\Theta$ es el dominio del parámetro a estimar. Una estimación es la evaluación obtenida por un estimador sobre una muestra $x_1, \ldots, x_n$, esto es, $T_n(x_1, \ldots, x_n)$.
\end{definition}

Nótese que la nomenclatura es ambigua. Para nosotros una muestra es una secuencia finita de variables aleatorias independientes e idénticamente distribuidas. Sin embargo, a los valores $x_1, \ldots, x_n$ obtenidos en la práctica también se le denomina muestra. Algunos autores evitan esta abigüedad denominando a $x_1, \ldots, x_n$ realización de la muestra. Nosotros distinguiremos entre ambos casos mediante el uso de mayúsculas para denotar variables aleatorias y el uso de minúsculas para denotar valores concretos.

En múltiples situaciones encontramos estimadores de calidad de forma natural. Por ejemplo, imaginemos que el parámetro $\theta_0$ se corresponde con la media de la distribución $f(X | \theta_0)$. En tal caso, parece claro que el mejor estimador para $\theta_0$ será la media muestral $\overline{x} = \frac{1}{n}\sum_{i = 1}^n x_i$. Sin embargo, en general no sabemos qué estimador hay que utilizar. Buscamos técnicas que nos proporcionen estimadores que sean razonables. En ocasiones querremos estimar $g(\theta_0)$, donde $g$ es determinada transformación de $\Theta$ en otro espacio más manejable.

\subsection{Método de los momentos}

El método de los momentos es, probablemente, el método más antiguo para estimar parámetros. Fue propuesto por Pearson al finales del siglo XIX. En muchos casos los resultados de este método son mejorables. Sin embargo, siempre es un último recurso en el caso de que no podamos aplicar otros métodos.

Sea $X_1, \ldots, X_n$ una muestra de un fenómeno con función de distribución $f(X |\theta)$ con $\theta = (\theta_1, \ldots, \theta_m) \in \Theta \subset \mathbb{R}^m$. Definimos los momentos de la muestra como $m_j = \frac{1}{n} \sum_{i = 1}^n X_i^j$. En media se debería cumplir que $m_j = E_\theta X^j$ para todo $j$ tal que $E_\theta X^j$ existe. Nótese que $E_\theta X^j = \mu_j(\theta_1, \ldots, \theta_m)$ es una función que depende de $\theta_1, \theta_2, \ldots, \theta_k$. El método de los momentos propone como estimador a una solución del sistema de ecuaciones
\begin{equation} \label{eq:sistema-momentos}
    \begin{matrix}
        m_1 = \mu_1(\theta_1, \ldots, \theta_k), \\
        m_2 = \mu_2(\theta_1, \ldots, \theta_k), \\
        \vdots \\
        m_k = \mu_k(\theta_1, \ldots, \theta_k). \\
    \end{matrix}
\end{equation}

\begin{ex}[Distribución normal]
    Supongamos que $X_1, \ldots, X_n$ son muestras de una distribución normal $N(\theta, \sigma^2)$. En el contexto anterior, los parámetros a estimar son $\theta_1 = \theta, \theta_2 = \sigma^2$. En este caso el sistema \eqref{eq:sistema-momentos} viene dado por las ecuaciones $\overline{X} = \theta$ y $m_2 = \theta^2 + \sigma^2$. La solución claramente es $\theta = \overline{X}$ y
    \[\sigma^2 = \frac{1}{n} \sum_{i = 1}^n X_i^2 - \overline{X}^2 = \frac{1}{n} \sum_{i = 1}^n (X_i - \overline{X})^2.\]
    En este caso, los estimadores obtenidos coinciden con nuestra intuición. Este método es más útil cuando no disponemos de un estimador intuitivo.
\end{ex}

\subsection{Método de la máxima verosimilitud de Fisher}

    El método de la máxima verosimilitud es una de las técnicas más utilizada para obtener estimadores de calidad.

    \begin{definition}
        Sea $x_1, \ldots, x_n$ una muestra de un fenómeno con función de distribución $f(X | \theta_0)$, donde $\theta_0 \in \Theta$. Se define la función de verosimilitud para cada $\theta \in \Theta$ como $L(\theta | x) = \prod_{i = 1}^n f(x_i| \theta)$.
    \end{definition}

    Para cada posible valor $\theta$ del parámetro a estimar, la verosimilitud proporciona la credibilidad que se le da a $\theta$ para los datos $x_1, \ldots, x_n$. Buscamos una aproximación $\hat{\theta}$ de $\theta_0$ en base a la muestra obtenida. Parece lógico que si asumimos que los datos son correctos, entonces una buena aproximación será aquella en la que los datos sean coherentes, esto es, la probabilidad de que se den datos similares a la muestra observada debe ser lo más alta posible.

    \begin{definition}
        Para cada elemento $x = (x_1, \ldots, x_n)$ del espacio muestral, definimos $\hat{\theta}(x) \in \Theta$ como un máximo global de $L(\theta | x)$. El estimador máximo verosímil (EMV) de una muestra $X$ se define como $\hat{\theta}(X)$.
    \end{definition}

    El estimador máximo verosímil presenta principalmente dos problemas.
    \begin{itemize}
        \item Cálculo del estimador. Para calcular $\hat{\theta}(X)$ es necesario maximizar una función. Muchas veces esto es complejo incluso para funciones de densidad comunes. Es más, puede suceder que la verosimilitud presente múltiples máximos globales y, por tanto, el estimador máximo verosímil no está bien definido. Necesitaremos condiciones sobre la distribución que nos permitan asegurar la buena definición del estimador máximo verosímil.
        \item Sensibilidad numérica. El valor $\hat{\theta}(x)$ puede cambiar considerablemente para pequeñas variaciones de $x$. Nos preguntamos qué condiciones debe verificar la función de distribución para evitar este comportamiento.
    \end{itemize}

    Para adentrarnos en el estudio de estos problemas necesitaremos teoría general de estimadores. Antes de desarrollarla realizaremos varios ejemplos de cálculo de estimadores máximo verosímiles.

    \begin{remark} \label{rem:emv:log}
        Los máximos globales de la función $L(\theta | x)$ se corresponden con los máximos globales de la función $\log L(\theta | x) = \sum_{i = 1}^n \log f(x_i | \theta)$. En múltiples ocasiones es más sencillo maximizar esta última expresión.
    \end{remark}

    \begin{ex}[Distribución normal]
        Consideremos una muestra $X_1, \ldots, X_n$ de un fenómeno con distribución $N(\theta,1)$. En primer lugar, calculamos la función de verosimilitud
        \[L(\theta | x) = \prod_{i = 1}^n \frac{1}{\sqrt{2\pi}} e^{-(x_i - \theta)^2 / 2} = \frac{1}{(2\pi)^{n/2}}e^{-\sum_{i = 1}^n (x_i - \theta)^2 / 2}.\]
        En virtud del Comentario \ref{rem:emv:log} maximizamos la función $-\sum_{i = 1}^n (x_i - \theta)^2 / 2 - n/2 \log(2\pi)$. Maximizar esta función equivale a minimizar $h(\theta) = \sum_{i = 1}^n (x_i - \theta)^2$. Derivando, obtenemos que $h'(\theta) = 0$ si, y solo si, $\theta = \overline{x}$. Además, es rutinario comprobar que $\overline{x}$ es el mínimo absoluto de $h$. Por tanto, $\overline{x}$ es el máximo absoluto de $L(\theta | x)$. Tenemos pues $\hat{\theta}(x) = \overline{x}$.
    \end{ex}

    A continuación pretendemos extender el método de la máxima verosimilitud para estimar $g(\theta)$, donde $g : \Theta \to \Theta'$ sobreyectiva. Si la aplicación $g$ fuese inyectiva, entonces podemos definir de norma natural la verosimilitud de $\eta \in \Theta'$ como $L^*(\eta | x) = L(g^{-1}(\eta) | x)$. Claramente, el valor que maximiza $L^*(\eta | x)$, que denotaremos $\hat{g}(x)$, es $g(\hat{\theta}(x))$. Sin embargo, los casos que presentan relevancia práctica son aquellos en los que $g$ no es inyectiva ya que de esta forma conseguimos reducir la dimensionalidad del espacio de parámetros. Necesitamos extender la definición de verosimilitud para abordar esta problemática.

    \begin{definition}
        En el contexto anterior, definimos la verosimilitud inducida por $g$ como
        \[L^*(\eta|x) = \sup\{L(\theta | x): \theta \in g^{-1}(\eta)\}.\]
        El valor $\hat{g}(x)$ que maximiza $L^*(\eta|x)$ se denomina estimador maximo verosímil de $g(\theta)$.
    \end{definition}

    La definición anterior es artificial en el sentido de que se realiza con el fin de poder mantener la propiedad de invarianza del estimador máximo verosímil, que se recoge en el siguiente teorema.

    \begin{thm}[Invarianza de Zehna]
        Para cualquier aplicación sobreyectiva $g: \Theta \to \Theta'$ se tiene que $\hat{g}(X) = g(\hat{\theta}(X))$.
    \end{thm}
    \begin{proof}
        En primer lugar, la definición de la verosimilitud inducida proporciona
        \[\sup_{\eta \in \Theta'} L^*(\eta|x) = \sup_{\eta \in \Theta'} \sup\{L(\theta | x): \theta \in g^{-1}(\eta)\} = \sup_{\theta \in \Theta} L(\theta | x).\]
        Por tanto, si la verosimilitud tiene un máximo global $\hat\theta(x)$, entonces lo tiene la verosimilitud inducida (el recíproco puede no ser cierto) y se alcanza en $g(\hat\theta(x))$.
    \end{proof}

    \subsection{Teoría general de estimadores}

    En esta sección introduciremos conceptos y definiciones relacionados con estimadores arbitrarios. El objetivo de esta teoría es dotarnos de herramientas que nos permitan abordar el estudio práctico de estimadores concretos, como el estimador máximo verosímil.

    \subsubsection{Estadísticos suficientes} \label{sec:estimacion:tge:sufi}

        Fijemos $\{f(x|\theta): \theta \in \Theta\}$ una familia de distribuciones. Sea $X = (X_1 , \ldots, X_n)$ una muestra de la distribución $f(x|\theta_0)$. Nuestro objetivo es inferir el parámetro $\theta_0$ a partir de la muestra. El concepto de estadístico suficiente nos permitirá separar la información contenida en $X$ en dos partes. Una parte contiene toda la información útil sobre $\theta_0$ mientras que la otra parte no dependerá del parámetro $\theta_0$. Consecuentemente, podemos ignorar esta última parte.

        Intuitivamente, un estadístico $T$ es suficiente para la familia de distribuciones considerada si $T(X)$ nos permite estimar $\theta_0$ tan bien como lo permite toda la muestra $X$. Procedemos a dar la definición matemática.

        \begin{definition}
            Un estadístico $T(X_1, X_2, \ldots, X_n)$ es suficiente si para cada $\theta \in \Theta$ y $t$ la distribución condicional de $X_1, X_2, \ldots, X_n$ respecto de $\theta$ y $T(X) = t$ no depende de $\theta$.
        \end{definition}

        El teorema de factorización de Neyman nos proporciona un criterio práctio para ver si un estadístico es suficiente.

        \begin{thm}
            Sea $\{f(x|\theta): \theta \in \Theta\}$ una familia de distribuciones. Sea $X = (X_1 , \ldots, X_n)$ una muestra de la distribución $f(x|\theta)$. Sea $T(X_1, X_2, \ldots, X_n)$ un estadístico. Entonces, $T$ es  suficiente si y solo si la función de verosimilitud puede factorizarse de la siguiente forma
            \[L(x_1, x_2, \ldots, x_n) = h(t;\theta) g(x_1, x_2, \ldots, x_n),\]
            donde $t = T(x_1, \ldots, x_n)$ y $g(x_1, x_2, \ldots, x_n)$ no depende de $\theta$.
        \end{thm}

        Si encontramos un estadístico suficiente, entonces podemos inferir el parámetro $\theta_0$ utilizando solamente la función $h(t;\theta)$. Interesa pues que el codominio del estadístico suficiente sea lo más simple posible. Los estadísticos suficientes son especialmente interesantes al aplicar el método de la máxima verosimilitud.

        \begin{ex}[Distribución normal, media desconocida]
            Sea $X = (X_1, \ldots, X_n)$ una muestra de $N(x |\mu, \sigma^2)$ donde solamente $\sigma^2$ es conocido ($\theta = \mu$). Es fácil verificar que
            \[\sum_{i = 1}^n (x_i - \mu)^2 = \sum_{i = 1}^n (x_i - \overline{x})^2 + n (\overline{x} - \mu)^2 = (n-1)S^2 + n(\overline{x} - \mu)^2.\]
            A partir de la igualdad anterior obtenemos
            \[f(x|\theta) = \frac{1}{(2\pi\sigma^2)^{n/2}} \exp(-\sum_{i = 1}^n (x_i - \mu)^2 / (2\sigma^2)) = \frac{1}{(2\pi\sigma^2)^{n/2}} \exp(-((n-1)S^2 + n(\overline{x} - \mu)^2) / (2\sigma^2)).\]
            Definimos
            \[g(x_1, x_2, \ldots, x_n) = \frac{1}{(2\pi\sigma^2)^{n/2}} \exp(-(n-1)S^2 / (2\sigma^2)) \text{ y } h(t|\mu) = \exp(n(t - \mu)^2) / (2\sigma^2)).\]
            Tenemos que $f(x|\theta) = h(\overline{x}|\theta) g(x_1, x_2, \ldots, x_n)$ y, por tanto, $T(x) = \overline{x}$ es suficiente.
        \end{ex}

        \begin{ex}[Distribución normal, ambos parámetros son desconocidos]
            Sea $X = (X_1, \ldots, X_n)$ una muestra de $N(x |\mu, \sigma^2)$ donde $\mu$ y $\sigma^2$ son desconocidos ($\theta = (\mu, \sigma^2)$).
            Tenemos que
            \[f(x|\theta) = \frac{1}{(2\pi\sigma^2)^{n/2}} \exp(-\sum_{i = 1}^n (x_i - \mu)^2 / (2\sigma^2)) = \frac{1}{(2\pi\sigma^2)^{n/2}} \exp(-(\sum_{i = 1}^n x_i^2 - 2\mu \sum_{i = 1}^n x_i + n \mu^2) / (2\sigma^2)). \]
            Por tanto, el estadístico $T(X_1, \ldots , X_n) = (\sum_{i = 1}^n X_i^2, \sum_{i = 1}^n X_i)$ es suficiente (tomamos $g(x_1, x_2, \ldots, x_n) = 1$). También podemos desarrollar $f(x|\theta)$ como sigue
            \[f(x|\theta) = \frac{1}{(2\pi\sigma^2)^{n/2}} \exp(-(\sum_{i = 1}^n (x_i - \overline{x})^2 + n(\overline{x} - \mu)^2) / (2\sigma^2)). \]
            Consecuentemente, el estadístico $T(X_1, \ldots , X_n) = (\overline{x}, S^2)$ también es suficiente. Nótese que el estadístico $T(X_1, \ldots , X_n) = (X_1, \ldots , X_n)$ es trivialmente suficiente, pero no aporta ninguna información.
        \end{ex}

    \subsubsection{Score, hipótesis de regularidad y función de información de Fisher}

    \begin{definition}
        Sea $\Theta \subset \mathbb{R}^m$ un abierto y sea $\{f(X|\theta): \theta \in \Theta\}$ una familia de funciones de densidad. Sea $X = (X_1, \ldots, X_n)$ una muestra que sigue una distribución con función de densidad $f(X|\theta_0)$. Si la función de verosimilitud para los valores $x = (x_1, \ldots, x_n)$ es diferenciable en $\theta \in \Theta$, entonces definimos el score de $\theta$ como el gradiente de la función $\log L(x; \theta)$ y lo denotamos $S(x; \theta)$.
    \end{definition}

    Intuitivamente el score indica la sensibilidad de la verosimilitud en un punto. Nos centraremos en el estudio del score cuando $\Theta$ es un abierto de $\mathbb{R}$. En tal caso
    \[S(x; \theta) = \frac{\partial}{\partial \theta} \log f(x | \theta) = \frac{\frac{\partial}{\partial \theta} f(x | \theta)}{f(x | \theta)}.\]

    Supongamos que el score de $\theta$ existe para cualesquiera valores de la muestra $x_1, \ldots, x_n$. En tal caso es natural considerar la función $E_{X|\theta}[S(X;\theta)]$, que depende solamente de $\theta$. Si $\theta$ fuese el parámetro a estimar, entonces $E_{X|\theta}[S(X;\theta)]$ mide la sensibilidad media de la verosimilitud en $\theta$.

    \begin{lem} \label{lem:score:esp}
        Sea $\Theta \subset \mathbb{R}$ un abierto y sea $\{f(X|\theta): \theta \in \Theta\}$ una familia de funciones de densidad que verifican las siguientes condiciones de regularidad:
        \begin{enumerate}
            \item Para cualesquier muestra $x=(x_1, \ldots, x_n)$ la función $L(x; \theta)$ es diferenciable para todo $\theta \in \Theta$.
            \item Se verifica
            \[\frac{\partial}{\partial \theta}\int_{X} f(x| \theta) dx = \int_{X} \frac{\partial}{\partial \theta} f(x| \theta) dx.\]
        \end{enumerate}
        Entonces, $E_{X|\theta}[S(X;\theta)] = 0$.
    \end{lem}
    \begin{proof}
        Tenemos que
        \[E_{X|\theta}[S(X;\theta)] = \int_{X} \frac{\frac{\partial}{\partial \theta} f(x | \theta)}{f(x | \theta)}  f(x | \theta) dx = \int_{X} \frac{\partial}{\partial \theta} f(x | \theta) dx = \frac{\partial}{\partial \theta} \int_{X} f(x; \theta) dx = \frac{\partial}{\partial \theta} 1 = 0. \qedhere\]
    \end{proof}

    En el resultado anterior aparecen por primera vez hipótesis de regularidad sobre las distribuciones a estudiar. Nótese que en la práctica normalmente vamos a trabajar con distribuciones que satisfagan estas hipótesis. La hipótesis b) se verifica si la derivada de la verosimilitud es continua \cite{leibniz}. Consecuentemente, todas las distribuciones continuas estudiadas, exceptuando la distribución de Laplace, cumplen estas hipótesis de regularidad (su función de densidad es de clase infinito con respecto a $\theta$).

    \begin{definition}
        Sea $\Theta \subset \mathbb{R}$ un abierto y sea $\{f(X|\theta): \theta \in \Theta\}$ una familia de funciones de densidad para la cual siempre existe el score. Dado $\theta \in \Theta$, definimos la función de información de Fisher en $\theta$ como el segundo momento de la variable aleatoria $S(X;\theta)$, donde $X = (X_1, \ldots, X_n)$ es una muestra de la distribución con función de densidad $f(X|\theta)$. Se denota $\mathcal{I}(\theta) := E_{X|\theta}[S(X;\theta)^2] \ge 0.$
    \end{definition}

    Si en determinado contexto no está clara la muestra $X$ para la cual calculamos la información de Fisher, entonces la denotamos $\mathcal{I}_X$ o $\mathcal{I}^X$.

    El siguiente resultado nos permite explicar por qué se define de esta forma la información de Fisher.

    \begin{cor}
        Bajo las hipótesis de regularidad del Lema \ref{lem:score:esp}, tenemos que  $\mathcal{I}(\theta) = Var_{X|\theta}(S(X;\theta))$.
    \end{cor}
    \begin{proof}
        Nótese que $Var_{X|\theta}(S(X;\theta)) = \mathcal{I}(\theta) - E_{X|\theta}[S(X; \theta)]^2$. El Lema \ref{lem:score:esp} nos indica que $E_{X|\theta}[S(X; \theta)] = 0$.
    \end{proof}

    Como consecuencia, la información de Fisher nos informa de cómo varía la sensibilidad de la verosimilitud en $\theta$. Si la información de Fisher es pequeña, entonces la sensibilidad de la verosimilitud en $\theta$ no depende prácticamente de la muestra utilizada y, por tanto, siempre será cercana a cero. Si por el contrario la información de Fisher es muy grande, entonces la sensibilidad de la verosimilitud en $\theta$ varía mucho en función de la muestra con la que se trabaje. Si utilizamos el estimador máximo verosímil, entonces estamos maximizando el logaritmo de la verosimilitud. Buscamos pues aquellos $\theta$ que sean extremos relativos de $\log L(x; \theta)$ y, por tanto, verifiquen $S(x; \theta) = 0$. Consecuentemente, nos interesa que $\mathcal{I}(\theta)$ sea grande para todo $\theta$ ya que de esta forma podremos discriminar aquellos $\theta$ que tengan score no nulo (no son extremos relativos de $\log L(x; \theta)$). Si en determinado $\theta$ la información de Fisher es muy pequeña, obtendremos que $\theta$ es un candidato a estimador máximo verosímil para casi cualquier muestra, incluso para muestras poco probables bajo ese parámetro, lo cual dificulta el correcto cómputo del estimador.

    En lo que sigue habitualmente exigiremos unas hipótesis de regularidad más fuertes, denominadas hipótesis o condiciones de regularidad de Cramer-Rao. Estas hipótesis son las siguientes:

    \begin{enumerate}[label=\roman*)]
        \item $\Theta$ es un abierto de $\mathbb{R}$.
        \item Para cualquier muestra $x = (x_1, \ldots, x_n)$, la verosimilitud $L(x | \theta)$ es dos veces derivable en $\Theta$.
        \item $\frac{\partial^i}{\partial\theta^i} \int_X f(x | \theta) dx = \int_X \frac{\partial^i}{\partial\theta^i} f(x | \theta) dx$ para $i=1,2$.
        \item Para cada $\theta \in \Theta$ se tiene $0 < \mathcal{I}(\theta) < +\infty$.
        %\item La función $\Psi(\theta) = \mathbb{E}_{\theta_0} \frac{\partial}{\partial \theta} f(x | \theta)$ es continua en $\theta_0$.
    \end{enumerate}

    Todas las distribuciones continuas estudiadas, exceptuando la distribuciónd de Laplace, verifican estas hipótesis de regularidad.

    El siguiente lema profundiza en nuestro entendimiento de la función de información de Fisher.

    \begin{lem} \label{lem:fisher:2dev}
        Bajo hipótesis de regularidad de Cramer-Rao tenemos que
        \[\mathcal{I}(\theta) = E_{X|\theta} \left[-\frac{\partial^2}{\partial\theta^2} \log f(X;\theta) \right].\]
    \end{lem}
    \begin{proof}
        En primer lugar, podemos escribir
        \[\frac{\partial^2}{\partial\theta^2} \log f(X|\theta)=\frac{\frac{\partial^2}{\partial\theta^2} f(X|\theta)}{f(X| \theta)}\;-\;\left( \frac{\frac{\partial}{\partial\theta} f(X|\theta)}{f(X| \theta)} \right)^2=\frac{\frac{\partial^2}{\partial\theta^2} f(X|\theta)}{f(X| \theta)}\;-\;\left( \frac{\partial}{\partial\theta} \log f(X|\theta)\right)^2.\]
        La demostración finaliza al tomar esperanzas en la expresión anterior y darse cuenta de que
        \[E_{X|\theta}\left[\frac{\frac{\partial^2}{\partial\theta^2} f(X|\theta)}{f(X| \theta)}\right] = \int_X \frac{\partial^2}{\partial\theta^2} f(x | \theta)\, dx = \frac{\partial^2}{\partial\theta^2} \int_X f(x | \theta)\, dx = \frac{\partial^2}{\partial\theta^2} \; 1 = 0. \qedhere\]
    \end{proof}

    Como consecuencia, la información de Fisher también indica cuál es la curvatura media de la función $\log L(x; \theta)$, que como vemos, en media es negativa ($\mathcal{I}(\theta) \ge 0$). Para calcular el estimador máximo verosímil intentamos maximizar $\log L(x; \theta)$. Si la función de información de Fisher es habitualmente grande, entonces en media tendremos máximos relativos muy claros.

    El Lema \ref{lem:fisher:2dev} nos permite calcular la información de Fisher de forma más sencilla, como muestran los siguientes ejemplos.

    \begin{ex} \label{ex:fisher:binom}
        Calculamos la función de información de Fisher de $X \sim B(x | n, \theta)$ donde $n$ es conocido. Recordemos que $\log f(X | n, \theta) = \log \binom{n}{X} + X \log \theta + (n - X) \log (1 - \theta)$. Derivando dos veces respecto de $\theta$ obtenemos
        \[\frac{\partial^2}{\partial \theta^2} f(X | n, \theta) = \frac{-X}{\theta^2} + \frac{-(n-X)}{(1 - \theta)^2}.\]
        Por tanto, la función de información de Fisher responde a
        \[\mathcal{I}_X(\theta) =  E_{X|\theta} \left[\frac{X}{\theta^2} + \frac{(n-X)}{(1 - \theta)^2}\right] = n \left(\frac{1}{\theta} + \frac{1}{1 - \theta}\right) = \frac{n}{\theta (1 - \theta)}. \qedhere\]
    \end{ex}

    \begin{ex}
        Calculamos la función de información de Fisher de $X \sim N(x|\mu, \sigma^2)$ para varias configuraciones de la distribución normal.
        \begin{itemize}
            \item El parámetro $\sigma^2$ es conocido. Tenemos que $\log f(X|\mu, \sigma^2) = - (X-\mu)^2 / (2\sigma^2) - \log(\sqrt{2\pi}) - \log(\sigma^2)/2$. Consecuentemente, deducimos que \[\frac{\partial^2}{\partial\mu^2} f(X|\mu,\sigma^2) = \frac{-1}{\sigma^2}.\]
            Por tanto, $\mathcal{I}(\mu) = 1 / \sigma^2$.
            \item El parámetro $\mu$ es conocido. Obtenemos que
            \[\frac{\partial^2}{\partial(\sigma^2)^2} f(X|\mu,\sigma^2) = -\frac{(X-\mu)^2}{\sigma^6} + \frac{1}{2\sigma^4}.\]
            Por tanto, podemos calcular $\mathcal{I}(\sigma^2)$ utilizando que $E[(X-\mu)^2] = Var(X) = \sigma^2$. Obtenemos que
            \[\mathcal{I}(\sigma^2) = E_{X|\sigma^2}[\frac{(X-\mu)^2}{\sigma^6} - \frac{1}{2\sigma^4}] = \frac{1}{\sigma^6} Var((X-\mu)^2) - \frac{1}{2\sigma^4} = \frac{1}{2\sigma^4}. \qedhere\]
        \end{itemize}
    \end{ex}

    \begin{remark}
        Bajo hipótesis de regularidad de Cramer-Rao, si $X=(X_1, \ldots, X_n)$ es una muestra de $f(X|\theta)$, entonces tenemos que
        \[\frac{\partial^2}{\partial \theta^2}\log(f(X;\theta)) = \frac{\partial^2}{\partial \theta^2} \left(\sum_{i = 1}^n \log(f(X_i;\theta))\right) = \sum_{i = 1}^n \frac{\partial^2}{\partial \theta^2}\log(f(X_i;\theta)).\]
        Consecuentemente, $\mathcal{I}^{X_1, \ldots, X_n}(\theta) = \sum_{i = 1}^n \mathcal{I}^{X_i}(\theta) = n \mathcal{I}^{X_i}(\theta)$.
    \end{remark}

    \begin{lem}
        Bajo hipótesis de regularidad de Cramer-Rao, sea $T(X_1, \ldots, X_n)$ un estadístico tal que su ditribución inducida también verifica las hipótesis de regularidad de Cramer-Rao. Entonces, para cualquier $\theta \in \Theta$ se tiene
        \[\mathcal{I}_{T(X)}(\theta) \le \mathcal{I}_{X}(\theta).\]
        Además, la igualdad se da para todo $\theta \in \Theta$ si, y solo si, $T$ es suficiente.
    \end{lem}

    En lo que sigue necesitaremos el siguiente lema.

    \begin{lem}[Desigualdad de Jenssen]
        Sean $X$ una variable aleatoria cuya imagen está contenida en un intervalo $I$. Sea $g: I \to \mathbb{R}$ una función.
        \begin{enumerate}
            \item Si $g$ es convexa, entonces $E[g(X)] \ge g(E[X])$.
            \item Si $g$ es cóncava, entonces $E[g(X)] \le g(E[X])$
        \end{enumerate}
    \end{lem}

    \begin{prop} \label{prop:desigualdad}
        Sea $X = (X_1, \ldots, X_n)$ una muestra de $f(X;\theta_0)$. Entonces, para cada $\theta_1 \in \Theta$ se tiene
        \[E_{X|\theta_0} \log f(X|\theta_0) \ge E_{X|\theta_0} \log f(X|\theta_1).\]
    \end{prop}
    \begin{proof}
        Por la desigualdad de Jenssen obtenemos
        \[E_{X|\theta_0} \log \frac{f(X|\theta_1)}{f(X|\theta_0)} \le \log \int_X \frac{f(X|\theta_1)}{f(X|\theta_0)} f(X|\theta_0) \, dx = \log \int_X f(X|\theta_1) \, dx = 0,\]
        de donde se deduce el resultado.
    \end{proof}

    Cabe mencionar que la información de Fisher puede definirse cuando $\Theta \subset \mathbb{R}^m$. Incluimos la definición por complitud, aunque no entraremos en ella a fondo.

    \begin{definition}
        Sea $\Theta \subset \mathbb{R}^m$ un abierto y sea $\{f(X|\theta): \theta \in \Theta\}$ una familia de funciones de densidad para la cual siempre existe el score. Dado $\theta \in \Theta$, definimos la función de información de Fisher en $\theta$ como
        \[{\left(\mathcal{I} \left(\theta \right) \right)}_{i, j} = E_{X|\theta} \left[
          \left(\frac{\partial}{\partial\theta_i} \log f(X;\theta)\right)
          \left(\frac{\partial}{\partial\theta_j} \log f(X;\theta)\right)\right], \ 1 \le i,j \le m,\]
        donde $X = (X_1, \ldots, X_n)$ es una muestra de la distribución con función de densidad $f(X|\theta)$.
    \end{definition}

    Bajo determinadas hipótesis de regularidad se puede probar que para cada $1 \le i, j \le n$ se verifica

    \[{\left(\mathcal{I} \left(\theta \right) \right)}_{i, j} =
      -E_{X|\theta}\left[\frac{\partial^2}{\partial\theta_i \, \partial\theta_j} \log f(X;\theta)
    \right].\]

    \begin{ex}
        Calculamos la función de información de Fisher de una variable aleatoria $X$ que siga una distribución $Beta(x | p, q)$, donde $\theta = (p, q)$. Recordemos que
        \begin{equation} \label{eq:log-beta}
            \log f(x | p, q) = (p-1) \log x + (q-1) \log (1-x) - \log \beta(p, q).
        \end{equation}
        Recordemos que $\beta(p,q) = \Gamma(p) \Gamma(q) / \Gamma(p+q)$. Habitualmente al cociente $\Gamma'(p)/\Gamma(p)$ se le llama función digamma y se denota $\psi(p)$. Las derivadas parciales de \eqref{eq:log-beta} son las siguientes:
        \begin{enumerate}
            \item $\frac{\partial}{\partial p} \log f(x | p, q) = \log x + \psi(p+q) - \psi(p).$
            \item $\frac{\partial}{\partial q} \log f(x | p, q) = \log(1-x) + \psi(p+q) - \psi(q).$
            \item $\frac{\partial^2}{\partial p^2} \log f(x | p, q) = \psi'(p+q) - \psi'(p).$
            \item $\frac{\partial^2}{\partial q^2} \log f(x | p, q) = \psi'(p+q) - \psi'(q).$
            \item $\frac{\partial^2}{\partial p\partial q} \log f(x | p, q) = \psi'(p+q).$
        \end{enumerate}

        Nótese que las derivadas obtenidas no dependen de $x$ y, por tanto, al tomar esperanzas obtenemos las mismas derivadas. Consecuentemente, la función de información de Fisher de $X$ viene dada por
        \[\mathcal{I}_X(p, q) = - \left(\begin{matrix} \psi'(p+q) - \psi'(p) & \psi'(p+q) \\ \psi'(p+q) & \psi'(p+q) - \psi'(q) \end{matrix}\right). \qedhere\]

    \end{ex}

    \subsubsection{Estimadores insesgados}

    Para comprobar cómo de bueno es un estimador $T$ podemos definir una función de pérdida $L(\theta,T(X))$ que indique la pérdida asociada a estimar un parámetro mediante $T$ si su verdadero valor es $\theta$. A partir de la función de pérdida definimos la función de riesgo, que asocia a cada posible valor del parámetro la pérdida media producida por el estimador. La función de riesgo viene dada por
    \[ R^L_T(\theta) = E_{X|\theta} [L(\theta,T(X))].\]
    Un estimador $T$ que ``minimice uniformemente'' la función de riesgo hará mejores estimaciones en media. Con minimizar uniformemente queremos decir que para cada estimador $T'$ se tiene que
    \[ R^L_T(\theta) \leq R^L_{T'}(\theta) \ \forall \ \theta \in \Theta.\]

    En esta sección introducimos un tipo particular de estimadores que minimizan determinada función de riesgo.

    \begin{definition}
        Se denomina sesgo de un estimador $T$ de $g(\theta)$ a la diferencia entre la esperanza del estimador y el verdadero valor del parámetro a estimar. Diremos que un estimador es insesgado si para cualquier posible valor del parámetro a estimar su sesgo es nulo.
    \end{definition}

    Nótese que el sesgo de un estimador es la función de riesgo asociada a la pérdida $L(\theta,T(X)) = g(\theta) - T(X)$. Un estimador insesgado verifica $0 = g(\theta) - E_{X|\theta} T(X)$ y, por tanto, minimiza uniformemente la función de riesgo. Aunque esta propiedad puede parecer a priori interesante, puede suceder que en la práctica el estimador insesgado no proporcione estimaciones de calidad si la varianza $Var_{X|\theta}(T(X))$ es muy alta.

    Claramente, la media muestral es un estimador insesgado de la media de la distribución. El siguiente resultado nos muestra otro ejemplo de un estimador insesgado.

    \begin{prop}
        Sea $X_1, \ldots, X_n$ una muestra de alguna población con función de densidad $f(X | \theta_0)$. Definimos la varianza muestral como
        \[S^2 = \frac{1}{n-1}\sum_{i = 1}^n(X_i - \overline{X})^2.\]
        Entonces, $S^2$ es un estimador insesgado de la varianza de la distribución.
    \end{prop}
    \begin{proof}
        Nótese que $\sum_{i = 1}^n(X_i - \overline{X})^2 = \sum_{i = 1}^nX_i^2 - n\overline{X}^2$. Consecuentemente tenemos
        \[E\left[\sum_{i = 1}^n(X_i - \overline{X})^2\right] = \sum_{i = 1}^nE\left[X_i^2\right] -     nE[\overline{X}^2] = n(E\left[X_i^2\right] - E[\overline{X}^2]).\]
        Utilizando que $Var(\overline{X}) = Var(X_i) / n$ y $E[\overline{X}] = E[X_i]$ obtenemos
        \[E[X_i^2] - E[\overline{X}^2] = Var(X_i) + E[X_i]^2 - Var(\overline{X}) - E[\overline{X}]^2 = \frac{n-1}{n} Var(X_i).\]
        Por tanto, $E[S^2] = Var(X_i)$ como se quería.
    \end{proof}

    La función de información de Fisher juega un papel importante en el estudio de los estimadores insesgados como muestra el siguiente Teorema.

    \begin{thm}[Cota de Cramer-Rao]
        Supongamos que se verifican las hipótesis de regularidad de Cramer-Rao. Sea $g: \Theta \to  \mathbb{R}$ de clase 1. Sea $\hat{\theta}$ un estimador insesgado de $g(\theta)$ tal que
        \[\int_X \left|\hat{\theta}(x) \frac{\partial}{\partial \theta}f(x | \theta)\right| dx < \infty.\]
        Entonces, para todo $\theta \in \Theta$
        \[Var_{X|\theta}(\hat{\theta}) \ge \frac{g'(\theta)^2}{\mathcal{I}(\theta)}.\]
    \end{thm}
    \begin{proof}
        Puesto que $\hat{\theta}$ es insesgado tenemos que
        \[g(\theta) = \int_X \hat{\theta}(x) f(x|\theta)\,dx.\]
        Podemos derivar respecto de $\theta$ la expresión anterior y utilizar que $\int_X \frac{\partial}{\partial \theta}f(x|\theta) \, dx = 0$, obteniendo
        \begin{equation}\label{eq:proof-cr}
        g'(\theta) = \int_X \hat{\theta}(x)\frac{\partial}{\partial \theta}f(x|\theta)\,dx = \int_X \left(\hat{\theta}(x) - g(\theta)\right)\frac{\partial}{\partial \theta}f(x|\theta)\,dx.
        \end{equation}
        Aplicamos la desigualdad de Cauchy-Schwarz al miembro de la derecha de \eqref{eq:proof-cr}, obteniendo
        \[g'(\theta)^2  \le \left( \int \left(\hat\theta(x) - g(\theta)\right)^2 f(x|\theta) \, dx \right) \left(\int \left( \frac{\partial}{\partial\theta} (\log f(x|\theta)) \right)^2 f(x|\theta) \, dx \right) = Var_{X|\theta}(\hat\theta) \, \mathcal{I}(\theta). \qedhere\]
    \end{proof}

    \begin{cor}
        Supongamos que se verifican las hipótesis de regularidad de Cramer-Rao. Sea $\hat{\theta}$ un estimador insesgado de $\theta$ tal que
        \[\int_X \left|\hat{\theta}(x) \frac{\partial}{\partial \theta} f(x | \theta)\right| dx < \infty.\]
        Entonces, para todo $\theta \in \Theta$
        \[Var_{X|\theta}(\hat{\theta}) \ge \frac{1}{\mathcal{I}_X(\theta)} = \frac{1}{n\mathcal{I}_{X_i}(\theta)}.\]
    \end{cor}

    La cota de Cramer-Rao nos dice que si la información de Ficher en $\theta$ es pequeña, entonces cualquier estimador insesgado tendrá una gran varianza y, por tanto, será inestable ante pequeños cambios en la muestra.

    \begin{definition}
        Un estimador se dice eficiente si alcanza la cota de Cramer-Rao para todo $\theta \in \Theta$.
    \end{definition}

    \subsubsection{Consistencia de sucesiones de estimadores}

    Nos interesa que los estimadores tiendan al parámetro de la distribución cuando el tamaño de la muestra diverge. En tal caso, podemos mejorar el resultado del estimador recurriendo a una mayor muestra de la población.

    \begin{definition}
        Consideremos una familia de densidades $\{f(x | \theta) : \theta \in \Theta\}$. Una sucesión de estimadores $\hat\theta_n$ de $g(\theta)$ es consistente para $\theta_0 \in \Theta$ si toda sucesión $X_n$ de variables aleatorias independientes e idénticamente distribuidas con función de distribución $f(x | \theta_0)$ la sucesión $\hat\theta_n(X_1, \ldots, X_n)$ converge en probabilidad ($P_{\theta_0}$) a $g(\theta_0)$. Si $\hat\theta_n$ es consitente para todo $\theta_0 \in \Theta$, entonces decimos que es consistente.
    \end{definition}

    \begin{thm}
        Sea $\hat\theta_n$ una sucesión de estimadores de $g(\theta)$ verificando
        \begin{enumerate}
            \item $\lim_{n \to \infty} E_\theta[\hat\theta_n] = g(\theta)$;
            \item $\lim_{n \to \infty} Var_\theta(\hat\theta_n) = 0$.
        \end{enumerate}
        Entonces, $\hat\theta_n$ es consistente para $\theta$.
    \end{thm}
    \begin{proof}
        Sea $\varepsilon > 0$. La desigualdad de Markov nos proporciona
        \[P_\theta[|\hat\theta_n - g(\theta)| \ge \varepsilon] \le \varepsilon^{-2} E[(\hat\theta_n - g(\theta))^2] = \varepsilon^{-2}  \left(Var(\hat\theta_n) + (E\hat\theta_n -g(\theta))^2\right).\]
        La prueba finaliza al recordar que el último término converge a $0$.
    \end{proof}

    Otra propiedad interesante de un estimador cuando la muestra tiene a infinito es la siguiente.

    \begin{definition}
        Consideremos una familia de densidades $\{f(x | \theta) : \theta \in \Theta\}$. Una sucesión de estimadores $\hat\theta_n$ de $g(\theta)$ es asintóticamente normal para $\theta_0 \in \Theta$ si toda sucesión $X_n$ de variables aleatorias independientes e idénticamente distribuidas con función de distribución $f(x | \theta_0)$ la sucesión $\sqrt{n}(\hat\theta_n(X_1, \ldots, X_n) - g(\theta_0))$ converge en ley a una distribución $N(X|0,\sigma^2)$ para cierto $\sigma^2 > 0$.
    \end{definition}

    \begin{prop}
        Todo estimador asintóticamente normal es consistente.
    \end{prop}
    \begin{proof}
        Sea $\hat\theta_n$ un estimador asintóticamente normal de $g(\theta)$. Tenemos que $n (\hat\theta_n - g(\theta))^2$ converge en ley a $Gamma(X|1/2, 1/(2\sigma^2))$ por la Proposición \ref{prop:normal-square}. Sea $F$ la función de distribución de $Gamma(X|1/2, 1/(2\sigma^2))$ y sea $\varepsilon > 0$. Vamos a probar que $P[(\hat\theta_n - g(\theta))^2 < \varepsilon] \to 1$. En efecto, sea $1 > \alpha \ge 0$. Tomamos $y$ tal que $F(y) = \alpha$. Tenemos que
        \[P[n(\hat\theta_n - g(\theta))^2 < y] \to F(y) = \alpha.\]
        Por tanto, para cada $\delta > 0$ existe $n_0$ tal que $\varepsilon > y /n$ y para cada $n \ge n_0$ tenemos
        \[P[(\hat\theta_n - g(\theta))^2 < \varepsilon] \ge P[n(\hat\theta_n - g(\theta))^2 < y] \ge \alpha - \delta.\]
        Deducimos que $\liminf P[(\hat\theta_n - g(\theta))^2 < \varepsilon] \ge \alpha - \delta.$ De la arbitrariedad de $\delta$ y $\alpha$ se deduce el resultado.
    \end{proof}

    Como es natural, el recíproco del anterior no es cierto.

    \subsection{Estudio teórico del estimador máximo verosímil}

    En este punto nos preguntamos cuándo está bien definido el estimador máximo verosímil. En tal caso nos interesa saber si el método de la máxima verosimilitud nos proporciona un estimador consistente. Para ello aplicamos los resultados teóricos vistos en la sección anterior.

    \begin{prop}
        Si existe un estadístico suficiente $T$ par ala familia de distribuciones $\{f(X|\theta): \theta \in \Theta\}$ y $\hat\theta$ es un estimador máximo verosímil, entonces $\hat\theta$ depende solamente de $T(X)$.
    \end{prop}
    \begin{proof}
        Por el teorema de factorización de estimadores suficientes podemos escribir $f(X|\theta) = h(X) g(T(X), \theta)$. Maximizar $L(x; \theta) = h(x) g(T(x); \theta)$ equivale a maximizar $g(T(x); \theta)$. Por tanto, $\hat\theta$ depende solamente de $T(x)$.
    \end{proof}

    \begin{thm}
        Bajo las hipótesis de regularidad de Cramer-Rao se verifican las siguientes afirmaciones:
        \begin{enumerate}
            \item Existe $n_0$ tal que para cada $n \ge n_0$ la ecuación en probabilidad $0 =\sum_{i=0}^n \frac{\partial}{\partial \theta} \log f(X_i | \theta)$ tiene solución única. A esta solución se le llama $\hat{\theta}_n(X_1, \ldots, X_n)$. En dicho punto se maximiza la verosimilitud.
            \item $\hat{\theta}(X_1, \ldots, X_n)$ es consistente. De hecho, se puede probar que la convergencia a $\theta_0$ es casi segura.
        \end{enumerate}
    \end{thm}
    \begin{proof}
        Para cualquier muestra $X$ de $f(X|\theta_0)$ tenemos que
        \[0 > -\mathcal{I}(\theta_0)=E_{X|\theta_0}\left[ \frac{\partial^2}{\partial\theta^2} \log f(X;\theta_0)\right] = \frac{\partial}{\partial\theta} E_{X|\theta_0} \left[ \frac{\partial}{\partial\theta} \log f(X;\theta_0) \right].\]
        Consecuentemente, $E_{X|\theta_0} \left[ S(X; \theta) \right]$ es decreciente en un entorno de $\theta_0$. Recordemos que $E_{X|\theta_0} \left[ S(X; \theta_0) \right] = 0$ por el Lema \ref{lem:score:esp}. Por tanto, existe $\varepsilon > 0$ tal que
        \begin{itemize}
            \item $E_{X|\theta_0} \left[ S(X; \theta) \right] > 0$ para todo $\theta \in (\theta_0 - \varepsilon, \theta_0)$;
            \item $E_{X|\theta_0} \left[ S(X; \theta) \right] < 0$ para todo $\theta \in (\theta_0, \theta_0 + \varepsilon)$.
        \end{itemize}
        Esto implica que $\theta_0$ es un máximo relativo de $E_{X|\theta_0}[\log f(X|\theta)]$.
        INCOMPLETO.
    \end{proof}

    \begin{thm}
        Bajo hipótesis de regularidad de Cramer-Rao, si $\hat{\theta}(X_1, \ldots, X_n)$ es un estimador máximo verosímil consistente, entonces es asintóticamente normal. Además, la varianza de la distribución normal asociada es $1/\mathcal{I}_{X_1}(\theta_0)$.
    \end{thm}
    \begin{proof}
        Escribimos $L_n(\theta) = \frac{1}{n}\sum_{i = 1}^n \log f(X_i|\theta)$. Tenemos que $L_n'(\hat\theta_n) = 0$. El teorema del valor medio nos proporciona un $\theta_n$ entre $\theta_0$ y $\hat\theta_n$ tal que $0 = L_n'(\hat\theta) = L_n'(\theta_0) + L_n''(\theta_n)(\hat\theta_n - \theta_0)$. Por tanto,
        \begin{equation} \label{eq:seq-normal}
            \sqrt{n}(\theta_0 - \hat\theta_n) = \frac{\sqrt{n}L_n'(\theta_0)}{L_n''(\theta_n)}.
        \end{equation}
        Por el teorema central del límite tenemos que
        \[\sqrt{n}L_n'(\theta_0) = \sqrt{n}(L_n'(\theta_0) - E_{\theta_0}[S(X_1; \theta_0)]) \to N(0, \mathcal{I}_{X_1}(\theta_0)),\]
        donde hemos utilizado que $Var_{\theta_0}(S(X_1; \theta_0)) = \mathcal{I}_{X_1}(\theta_0)$.
        Estudiamos ahora el denominador de \eqref{eq:seq-normal}. Puesto que $\hat\theta_n$ converge en probabilidad a $\theta_0$ y $\theta_n$ se encuentra entre ambos, tenemos que $\theta_n$ converge en probabilidad a $\theta_0$. Además, la ley uniforme de los grandes números nos garantiza que
        \[L_n''(\theta) \xrightarrow{P_{\theta_0}} E_{\theta_0}\left[\frac{\partial^2}{\partial \theta^2}f(X_1|\theta)\right],\]
        siendo la convergencia uniforme en espacios de parámetros compactos. Por tanto, tomando un compacto que contenga una cola de $\theta_n$ obtenemos la mencionada convergencia uniforme. De esta convergencia uniforme se desprende que
        \[L_n''(\theta_n) \xrightarrow{P_{\theta_0}} E_{\theta_0}\left[\frac{\partial^2}{\partial \theta^2}f(X_1|\theta_0)\right] = - \mathcal{I}_{X_1}(\theta_0).\]
        Hemos obtenido pues
        \[\sqrt{n}(\theta_0 - \hat\theta_n) = \frac{\sqrt{n}L_n'(\theta_0)}{L_n''(\theta_n)} \rightarrow N(0, \frac{1}{\mathcal{I}_{X_1}(\theta_0)}). \qedhere\]
    \end{proof}

\section{La familia exponencial}

    En esta sección estudiamos una amplia familia de distribuciones, denominada la familia exponencial. Veremos que gran parte de las distribuciones que hemos estudiado hasta el momento pertenecen a esta familia.

    \begin{definition}
        Una variable aleatoria se distribuye respecto de una \emph{familia exponencial} si su función de distribución es de la forma
        \begin{equation} \label{eq:exponencial}
            f(x | \theta) = h(x) \exp\left(\sum^k_{i=1} \theta_i T_i(x)  + \Psi (\theta)\right),
        \end{equation}
        donde $\theta = (\theta_1, \ldots, \theta_k)$ y $h(x) \ge 0$, $\Psi(\theta)$, $T_1(x), \ldots, T_k(x)$ son funciones reales.
    \end{definition}

    Las familias exponenciales presentan características matemáticas y estadísticas muy convenientes. De estas características cabe destacar el siguiente resultado, que utiliza estadísticos suficientes introducidos en la Sección \ref{sec:estimacion:tge:sufi}.

    \begin{prop} \label{prop:exp:sufi}
        Sea $\{f(X | \theta): \theta \in \Theta\}$ una familia exponencial y sea una muestra $x = (x_1, \ldots, x_n)$. Entonces, $T(x) = (\sum_{j = 1}^n T_i(x_j))_{j = 1, \ldots, k}$ es un estadístico suficiente de dimensión $k$.
    \end{prop}
    \begin{proof}
        En efecto, utilizando \eqref{eq:exponencial} basta escribir $f(x | \theta)$ como sigue
        \begin{equation*}
            f(x | \theta) = \prod_{i=1}^n l(x_i) \exp\left(\sum^n_{i=1}\sum^k_{j=1} \theta_i T_i(x_j)  + n\Psi(\theta)\right) = \prod_{i=1}^{n}l(x_i) \exp\left(\sum_{i=1}^{n}\theta_i \sum^k_{j=1} T_i(x_j)  + n\Psi(\theta)\right). \qedhere
        \end{equation*}
    \end{proof}

    Nótese que la dimensión del estadístico suficiente encontrado no depende de la muestra. A continuación mostramos algunos ejemplos de familiais exponenciales.

    \begin{ex}[Distribución binomial] \label{ex:exp:binom}
        La función masa de probabilidad de una distribución binomial con $n$ fijo puede escribirse como sigue
        \[f(x|p) = \binom{n}{x} p^x (1-p)^{n-x} = \binom{n}{x} \exp(x\log(p) + (n-x) \log(1-p)) = \binom{n}{x} \exp(x\log(\frac{p}{1-p}) + n \log(1-p)). \]
        La aplicación  $f(p) = \log(\frac{p}{1-p}) = \log(\frac{1}{1-p} - 1)$ es una biyección de $(0,1)$ a $\mathbb{R}$. En este punto hacemos el cambio de variable $\theta = f(p)$.
        Hemos obtenido que la distribución binomial es una familia exponencial de parámetro $\theta$ con $h(x) = \binom{n}{x}$, $T_1(x) = x$ y $\Psi(\theta) = n \log(1 - f^{-1}(\theta))$. Según la Proposición \ref{prop:exp:sufi} un estadístico suficiente es $T(x) = \sum_{i = 1}^n x_i$ y, por tanto, la media muestral, $T(x) = \overline{x}$, es otro estadístico suficiente.
    \end{ex}

    En el ejemplo anterior hemos tenido que realizar un cambio de variable del espacio paramétrico para poder escribir la distribución de Bernoulli como una familia exponencial. El nuevo espacio paramétrico obtenido es el \emph{espacio paramétrico natural} de la familia. Para evitar trabajar con cambios de variables algunos autores definen las familias exponenciales como aquellas cuya función de desidad se puede escribir de la forma

    \begin{equation} \label{eq:exponencial:2}
        f(x | \theta) = h(x) \exp\left(\sum^k_{i=1} w_i(\theta) T_i(x)  + \Psi(\theta)\right),
    \end{equation}
    donde  $h(x) \ge 0$, $\Psi(\theta)$, $w_1(\theta), \ldots, w_k(\theta)$ y $t_1(x), \ldots, T_k(x)$ son funciones reales. En el Ejemplo \ref{ex:exp:binom} se tendría $w_1(p) = \log(\frac{p}{1-p})$ y $\Psi(p) = n \log(1-p)$.

    \begin{ex}[Distribución normal] \label{ex:exp:normal}
        La función de densidad de la distribución normal se puede escribir de la forma
        \[f(x|\mu,\sigma^2) = \frac{1}{\sqrt{2 \pi} \sigma} \exp\left(- \frac{(x-\mu)^2}{2\sigma^2}\right) = \frac{1}{\sqrt{2 \pi} \sigma} \exp\left(- \frac{x^2}{2\sigma^2} + \frac{x\mu}{\sigma^2} - \frac{\mu^2}{2\sigma^2}\right)\]
        y, por tanto, es una familia exponencial con $h(x) = 1$, $\Psi(\mu, \sigma^2) = -\frac{\mu^2}{2\sigma^2} - \log(\sqrt{2\pi} \sigma)$, $T_1(x) = x$ y $T_2(x) = - x^2 / 2$.
        El espacio paramétrico natural se corresponde con $(1/\sigma^2, \mu/\sigma^2)$. No obstante, utilizamos los parámetros $(\mu, \sigma^2)$ debido a la interpretación estadística de los mismos.

        Como consecuencia de la Proposición \ref{prop:exp:sufi} obtenemos que cualquier variable aleatoria siguiendo una distribución $N(x|\mu, \sigma^2)$ verifica que $T(x) = (\sum x_i, \sum x_i^2)$ es un estadístico suficiente.
    \end{ex}

    La mayoría de las distribuciones estudiadas hasta el momento forman una familia exponencial. La Tabla \ref{table:exponencial} muestra una lista de ejemplos. No obstante, no toda familia de distribuciones es exponencial, como sucede con las distribuciones uniformes.

    \begin{table}[H]
    	\begin{center}
    		\begin{tabular}{|l|l|l|}
    			\hline
    			DENSIDAD & NOTACIÓN & SOPORTE\\
    			\hline \hline
                $\frac{1}{\sqrt{2 \pi} \sigma} \exp\left(- \frac{(x-\mu)^2}{2\sigma^2}\right)$ & $N(x|\mu, \sigma^2)$ & $\mathbb{R}$ \\
                \hline
                $\frac{1}{\Gamma(\alpha)\beta^\alpha}x^{\alpha - 1}e^{\frac{-x}{\beta}}$ & $ gamma(x|\alpha,\beta)$ &   $(0,\infty)  $\\ \hline
    			$\frac{1}{\Gamma(\frac{p}{2})2^{\frac{p}{2}}}x^{\frac{p}{2} - 1}e^{\frac{-x}{2}}$ & $\chi^2$ con p grados de libertad  & $(0,\infty)  $ \\ \hline
    			$\frac{\Gamma(\alpha + \beta)}{\Gamma(\alpha) \Gamma(\beta)} x ^{\alpha - 1}(1-x)^{\beta - 1}$ &  $beta(x| \alpha, \beta)$   &   $(0,1)$   \\ \hline
    			$\binom{n}{x} \theta ^x (1-\theta)^{n-x}$ &  $B(x|\theta, n)$ & ${0,1, \ldots , n}$   \\ \hline
    			$\frac{e^{-\lambda} \lambda^x}{x!}$ & $P(x|\lambda)$ & ${0,1, \ldots , n}$   \\ \hline
    		\end{tabular}
    	\end{center}
        \caption{Ejemplos de familias exponenciales.}
        \label{table:exponencial}
    \end{table}

\section{Tests de hipótesis}

En la Sección \ref{sec:estimacion} se estudió el problema de estimación de parámetros. En este capítulo se estudiará otro problema clásico de la inferencia, los tests de hipótesis.

\begin{definition}
    Sea $\{f(X|\theta): \theta \in \Theta\}$ una familia de distribuciones. Supongamos que una variable aleatoria sigue una distribución $f(X|\theta)$ con $\theta \in \Theta$. Una hipótesis es una afirmación $\theta \in \Theta_0$, donde $\Theta_0$ es un subconjunto de $\Theta$. Una hipótesis es simple si es de la forma  $\theta \in \{\theta_0\}$ para cierto $\theta_0 \in \Theta$. En tal caso se escribe $\theta = \theta_0$. Si una hipótesis no es simple, entonces decimos que es compuesta. Además, dos hipótesis $\theta \in \Theta_0$ y $\theta \in \Theta_1$ son excluyentes si $\Theta_0$ y $\Theta_1$ son disjuntos.
\end{definition}

El objetivo de los tests de hipótesis es, dadas dos hipótesis excluyentes, aceptar una de las dos hipótesis como verdadera tras observar una muestra. Esta acción se denomina contrastar dos hipótesis. Generalmente el tratamiento de las dos hipótesis no es simétrico, esto es, una de las dos hipótesis tiene preferencia sobre la otra y solo será rechazada cuando la evidencia en su contra sea muy clara. Esta hipótesis se llama hipótesis nula, mientras que la otra hipótesis se denomina hipótesis alternativa. Las denotamos $H_0: \theta \in \Theta_0$ y $H_1: \theta \in \Theta_1$ respectivamente (recordemos que $\Theta_0 \cap \Theta_1 = \emptyset$).

El contraste de hipótesis surge de forma natural en multitud de ciencias e ingenierías. Por ejemplo, supongamos que estamos estudiando cómo afecta un determinado medicamento a la presión sanguínea de los pacientes. Queremos corroborar que el medicamento no tiene ningún efecto sobre la presión, hecho que es nuestra hipótesis nula. Sea $\theta$ la variación media de la presión de los pacientes al tomar el medicamento. Esta situación puede representarse como $H_0: \theta = 0$ y $H_1: \theta \ne 0$. Tras tomar muestras en varios pacientes tendremos que decidir cuál de las dos hipótesis aceptamos. Otro ejemplo puede ser el estudio de la proporción media de piezas defectuosas que fabrica una empresa, que denotamos $\theta$. Sea $\theta_0$ el máximo valor aceptable que puede alcanzar esa proporción. Queremos combrobar si la proporción de piezas defectuosas es menor o igual que $\theta_0$, lo que se puede representar mediante las hipótesis $H_0: \theta \le \theta_0$ y $H_1: \theta > \theta_0$. Habitualmente no es factible propar cada una de las piezas y, por tanto, tenemos que hacer inferencia a partir de una muestra. En ambos problemas $\theta$ debe ser el parámetro de una distribución. Por ejemplo, podríamos utilizar una distribución normal de media $\theta$.

Nótese que en los dos ejemplos anteriores el espacio paramétrico es unión disjunta de $\Theta_0$ y $\Theta_1$. Podemos suponer que siempre se da esta situación. En efecto, el espacio paramétrico siempre se puede restringir a $\Theta_0 \cup \Theta_1$. En este contexto la hipótesis alternativa es la hipótesis complementaria a la hipótesis nula.

\begin{definition}
    Un test de hipótesis es una regla que permite dividir el espacio muestral en dos subconjuntos medibles disjuntos. Estos conjuntos se corresponden con aquellas muestras que aceptan la hipótesis nula como cierta (región de aceptación) y aquellas muestras que rechazan la hipótesis nula y aceptan la hipótesis alternativa (región crítica).
\end{definition}

Supondremos que la región crítica se corresponde con la imagen inversa de un boreliano, esto es, es de la forma $X \in R$, donde $X = (X_1, \ldots, X_n)$ es la muestra y $R \in \mathcal{B}^n$. En ocasiones podremos expresar la región crítica en términos de un estadístico $T(X_1, \ldots, X_n)$. Por ejemplo, este estadístico puede ser la media de la muestra y la región crítica aquellas muestras con media mayor que un determinado valor.

\subsection{Errores de los tests de hipótesis}


Consideremos un test de hipótesis con región crítica $X \in R$. En dicho test podemos cometer dos tipos de errores:

\begin{itemize}
    \item \textbf{Error de tipo 1.} Rechazar la hipótesis nula cuando es cierta. Si $\theta \in \Theta_0$, entonces la probabilidad de cometer un error de tipo 1 es $P_\theta(X \in R)$.
    \item \textbf{Error de tipo 2.} Rechazar la hipótesis alternativa cuando es cierta. Si $\theta \in \Theta_0^c$, entonces la probabilidad de cometer un error de tipo 2 es $P_\theta(X \in R^c)$.
\end{itemize}

\begin{definition}
    En el contexto actual, se define la función de potencia del test de hipótesis como
    \[\eta(\theta) = P_\theta(X \in R) = \begin{cases} \text{Probabilidad de cometer un error de tipo 1} & \text{ si } \theta \in \Theta_0; \\ \text{1 - Probabilidad de cometer un error de tipo 2} & \text{ si } \theta \in \Theta_0^c. \\ \end{cases}\]
\end{definition}

Nuestro objetivo es desarrollar tests de hipótesis tales que la probabilidad de cometer errores de tipo 1 y tipo 2 sea lo más pequeña posible para cualquier valor de $\theta$. Esto es, queremos minimizar $\eta$ em $\Theta_0$ y maximizar $\eta$ en $\Theta_0^c$. Por tanto, la función de potencia ideal es aquella que toma el valor $0$ en $\Theta_0$ y el valor $1$ en $\Theta_0^c$. Esta función solo se obtiene en situaciones triviales. Generalmente obtendremos funciones potencia mucho más complejas. Nótese que si reducimos la región crítica de un test entonces disminuye la probabilidad de cometer un error de tipo 1 pero aumenta la probabilidad de cometer un error de tipo 2. Consecuentemente, encontrar una región crítica apropiada no es una tarea sencilla.

Recordemos que la hipótesis nula generalmente tiene preferencia frente a la hipótesis alternativa. Por tanto, el error de tipo 1 es más grave que el error de tipo 2. Para asegurarnos que se respeta esta preferencia podemos buscar tests de hipótesis que garanticen que la probabilidad de cometer un error de tipo 1 es menor que un valor $\alpha$ fijado de antemano.

\begin{definition}
    Un test de hipótesis es de tamaño $\alpha \in [0,1]$ si $\sup\{\eta(\theta): \theta \in \Theta_0\} = \alpha$.
\end{definition}

\begin{definition}
    Un test de hipótesis tiene nivel de significación $\alpha \in [0,1]$ si $\eta(\theta) \le \alpha$ para todo $\theta \in \Theta_0$.
\end{definition}

Si nos restringimos a estudiar los tests de tamaño $\alpha$, entonces nuestro objetivo se reduce a buscar entre todos estos tests aquel que minimice la probabilidad de cometer un error de tipo 2. Este hecho queda formalizado en la siguiente definición.

\begin{definition}
    Considérese un test con función potencia $\eta$. Decimos que es un test más potente de tamaño $\alpha$ (resp. de significancia $\alpha$)  si para cualquier otro test de tamaño $\alpha$ (resp. de significancia $\alpha$) con función potencia $\eta'$ se verifica $\eta(\theta) \ge \eta'(\theta)$ para todo $\theta \in \Theta_0^c$.
\end{definition}

En lo que sigue buscaremos obtener los tests más potentes de tamaño $\alpha$ en caso de que sea posible. Cabe decir que en la práctica se utilizan valores de $\alpha$ pequeños, como $0.01$, $0.05$ o $0.1$.

\subsection{Tests de Neyman-Pearson}

En esta sección estudiamos cuáles son los tests más potentes para contrastar dos hipótesis simples. Esta cuestión es resuelta por el Lema de Neyman-Pearson. Además, estudiamos si es posible generalizar el resultado para contrastar hipótesis compuestas.

\begin{thm}[Lema de Neyman-Pearson]
    Supóngase que se desea contrastar dos hipótesis simples, $H_0 : \theta = \theta_0$ y $H_1 : \theta = \theta_1$. Para $k \in [0,1]$ consideramos la región crítica
    \[C_k = \{x: \frac{L(\theta_1;x)}{L(\theta_0;x)} \ge k\}.\]
    Sea $\alpha = P_{\theta_0}(C_k)$. Entonces, el test de región crítica $C_k$ es un test más potente de significancia $\alpha$. Además, cualquier test más potente de significancia $\alpha$ es de tamaño $\alpha$ y su región crítica difiere de $C_k$ en a lo sumo un subconjunto de probabilidad nula.
\end{thm}
\begin{proof}
    En primer lugar, nótese que el test dado por $C_k$ es de tamaño $\alpha$ ya que $\sup\{ \eta(\theta) : \theta \in \Theta_0\} = \eta(\theta_0) = \alpha$. Consideremos otro test de significancia $\alpha$ cuya función potencia es $\eta'$ y su región crítica es $C'$ y veamos que $\eta'(\theta_1) \le \eta(\theta_1)$. Por distinción de casos es fácil razonar que para cualquier muestra $x$ se verifica
    \begin{equation} \label{eq:np:desigualdad}
        (1_{C_k}(x)-1_C(x)) (L(\theta_1; x) - k L(\theta_0; x)) \ge 0,
    \end{equation}
    donde $1_A$ denota la función indicadora del conjunto $A$. Integrando la desigualdad \eqref{eq:np:desigualdad} obtenemos
    \begin{equation} \label{eq:np:desigualdad:2}
        0 \le \int(1_{C_k}(x)-1_C(x)) (L(\theta_1; x) - k L(\theta_0; x)) \, dx = \eta(\theta_1) - \eta'(\theta_1) - k (\eta(\theta_0) - \eta'(\theta_0)).
    \end{equation}
    Aplicando $\eta(\theta_0) - \eta'(\theta_0) = \alpha  - \eta'(\theta_0) \ge 0$ a \eqref{eq:np:desigualdad:2} deducimos que $0 \le \eta(\theta_1) - \eta'(\theta_1)$ como se quería. Por último, si el test de significancia $\alpha$, función potenci $\eta'$ y región crítica $C'$ es más potente, entonces
\end{proof}

\begin{cor}[Lema de Neyman-Pearson para distribuciones continuas]
    Supóngase que se desea contrastar dos hipótesis simples, $H_0 : \theta = \theta_0$ y $H_1 : \theta = \theta_1$. Entonces, para cada $\alpha \in [0,1]$ existe $k \in [0,1]$ tal que el test dado por la región crítica
    \[C_k = \{x: \frac{L(\theta_1;x)}{L(\theta_0;x)} \ge k\}\]
    es más potente de tamaño $\alpha$. Además, cualquier test más potente de significancia $\alpha$ es de tamaño $\alpha$ y su región crítica difiere de $C_k$ en a lo sumo un subconjunto de probabilidad nula.
\end{cor}

\subsection{Tests de la razón de verosimilitud}

Los tests de la razón de la verosimilitud están íntimamente ligado con los estimadores máximo verosímiles como veremos a continuación. Fijemos una familia de distribuciones $\{f(X|\theta): \theta \in \Theta\}$. Sea $X$ una variable aleatoria que sigue una distribución con función de densidad $f(X | \theta_0)$. Dada una muestra $x = (x_1, \ldots, x_n)$, recordemos que la verosimilitud se define como la función
$L(\theta;x_1, \ldots, x_n)$.

\begin{definition}
    Considérese una hipótesis $H_0 : \theta_0 \in \Theta_0$. El ratio de verosimilitud de $H_0$ para la muestra $x$ es
    \[\lambda(x) = \frac{\sup\{L(\theta; x): \theta \in \Theta_0\}}{\sup\{L(\theta; x): \theta \in \Theta\}}. \]
     Un test de la razón de verositud es cualquier test que cuya región crítica sea de la forma $\{x : \lambda(x) \le c\}$, donde $0 \le c \le 1$.
\end{definition}

Para motivar la definición de este tipo de test, supongamos que estamos trabajando con distribuciones discretas. En tal caso, tanto el numerador como el denominador de $\lambda(x)$ se corresponden con la máxima probabilidad posible de la muestra $x$ si variaos $\theta$ en $\Theta_0$ y $\Theta$ respectivamente. Si el cociente de ambos valores es pequeño ($\lambda(x) \le c$), entonces es razonable rechazar la hipótesis nula puesto que hay elementos de $\Theta_0^c$ para los cuales la muestra es más probable.

Supongamos ahora que existen los estimadores máximo verosímiles de $\theta_0$ en $\Theta_0$ y $\Theta$. Denotamos a estos estimadores $\hat{\theta_0}(x)$ y $\hat{\theta}(x)$ respectivamente. Entonces, $\lambda(x) = L(\hat{\theta_0}(x);x) / L(\hat{\theta}(x);x)$. El test de la razón de verosimilitud nos dice que rechazaremos la hipótesis nula cuando $\hat{\theta}(x)$  tenga una credibilidad considerablemente mayor que la de $\hat{\theta_0}(x)$, esto es, $\hat{\theta}(x)$ es mucho mejor estimador. Como caso particular, si a partir de una muestra $x$ hemos realizado una estimación de $\theta_0$ mediante un estimador máximo verosímil $\hat{\theta}$, entonces todo test de la razón de verosimilitud acepta la hipóteiss $H_0: \theta_0 = \hat{\theta}$ para la muestra $x$. Esto es, la filosofía de los tests de la razón de verosimilitud es coherente con la filosofía de los estimadores máximo verosímiles.

\section{Estadística bayesiana}

En esta sección estudiaremos el modelo de inferencia estadística desde el punto de vista bayesiano.

\subsection{Introducción}

Empecemos recordando uno de los teoremas clásicos de la probabilidad, el teorema de Bayes, que va a ser la base del modelo que queremos establecer. Supongamos que en el espacio de probabilidad $(\Omega,\mathcal{A},P)$ tenemos la partición de $\Omega$ dada por los sucesos $A_1,\dots,A_n$, todos ellos con probabilidad no nula. Y sea $B$ un suceso no nulo del que conocemos sus probabilidades condicionadas a cada suceso $A_i$. Entonces, podemos obtener la probabilidad de cada $A_i$ condicionada al suceso $B$ de la siguiente forma:

\begin{equation*}
	P(A_i|B)=\frac{P(A_i\cap B)}{P(B)}=\frac{P(B|A_i)P(A_i)}{P(B)}
\end{equation*}

A su vez, la ley de la probabilidad total establece que $P(B)=\sum_{i=1}^n{P(B|A_i)P(A_i)}$, luego:

\begin{equation*}
	P(A_i|B)=\frac{P(B|A_i)P(A_i)}{\sum_{i=1}^n{P(B|A_i)P(A_i)}}
\end{equation*}

Donde los valores $P(B|A_i)$ son conocidos. Los valores $P(A_i)$ son lo que llamaremos probabilidades a priori, mientras que los $P(A_i|B)$ serán las probabilidades a posteriori.

La estadística bayesiana sigue esta idea, basándose en la interpretación subjetiva de la probabilidad. Para ello, utiliza la percepción existente, por parte del investigador, como una variable modificadora (distribución a priori) de los datos muestrales, que dan lugar a una distribución (distribución a posteriori), con la que formular inferencias con respecto al parámetro de interés.

Consideremos un problema de inferencia estadística en el que las observaciones se toman de una variable aleatoria $X$ que sigue una distribución $f(x|\theta)$, con $\theta\in\Theta$. Disponemos de información previa sobre $\theta$, que podemos recoger definiendo una distribución de probabilidad sobre el espacio $\Theta$, la distribución a priori, dando así a $\theta$ el carácter de variable aleatoria, con la peculiaridad de que no es observable. Lo que sí observamos es la variable aleatoria $X$ condicionada al verdadero valor que toma $\theta$, que llamaremos $\theta_0$. Así, el estudio de las observaciones de $X$ aporta información sobre el valor de $\theta$, información que debemos combinar con la distribución a priori para modificarla. El resultado de esta modificación es de nuevo una distribución sobre $\Theta$, que llamaremos la distribución a posteriori de $\theta$, una vez observada la variable aleatoria $X$. Estos son los planteamientos básicos que conforman el enfoque bayesiano de la estadística.

Pasemos a definir formalmente estas distribuciones:

\newcommand{\vx}[1]{\underset{\sim}{#1}}

\begin{definition}
	Sea $X$ una variable aleatoria que sigue una distribución $f(x|\theta)$, con $\theta \in \Theta$. A una distribución $\pi(\theta)$ sobre el espacio $\Theta $ establecida con información previa conocida sobre $\theta$ se le llama distribución a priori de la variable aleatoria $\theta$.
	%Sea $X_{\sim}=(X_1,\dots,X_n)$ un vector de variables aleatorias de la familia de densidades $\{f(x_{\sim}|\theta)|\theta = (\theta_1,\dots,\theta_k) \in \Theta \subset \mathbb{R}^k \}$. A una distribución $\pi(\theta)$ sobre el espacio $\Theta $ establecida con información previa conocida sobre $\theta$ se le llama distribución a priori de la variable aleatoria $\theta$.

	%Dada una muestra $x_{\sim} = (x_1,\dots,x_n)$ de $X_{\sim}$, se define la probabilidad a posteriori de $\theta$ condicionada a la muestra como:

	Dada una muestra aleatoria simple $X_1,\dots,X_n$ de $X$, y el vector de observaciones de la muestra $\vx{x}=(x_1,\dots,x_n)$, se define la distribución a posteriori de $\theta$ como la ley condicional de $\theta$ dadas las observaciones $\vx{x}$ de $\vx{X}$.
	\begin{equation*}
		\pi(\theta|\vx{x})= \frac{f(\vx{x}|\theta)\pi(\theta)}{\int_{\Theta}{f(\vx{x}|\theta)\pi(\theta)d\theta}}
	\end{equation*}
\end{definition}

La distribución a posteriori, como acabamos de indicar, es una distribución condicional a las observaciones dadas. Como tal, es el cociente entre una densidad conjunta y una marginal. Además, se evalúa sobre la ley conjunta o verosimilitud de la muestra. Pasamos a recordar estos conceptos.

\begin{definition}
	Dada una muestra aleatoria simple $X_1,\dots,X_n$ de $X$, y el vector de observaciones de la muestra $\vx{x}=(x_1,\dots,x_n)$, a la ley conjunta de $X_1,\dots,X_n$ se denomina distribución muestral o verosimilitud de la muestra, dado el valor del parámetro $\theta$, a:

	\[f(\vx{x}|\theta)=\prod_{i=1}^{n}{f(x_i|\theta)}\]

	La densidad conjunta de $\vx{X}$ y $\theta$ es:

	\[f(\vx{x},\theta)=f(\vx{x}|\theta)\pi(\theta)\]

	La densidad marginal de $\vx{X}$ es:

	\[m(\vx{x}) = \int_{\Theta}{f(\vx{x},\theta)d\theta} = \int_{\Theta}{f(\vx{x}|\theta)\pi(\theta)d\theta}\]

	En consecuencia, podemos expresar la distribución a posteriori de esta forma:

	\[\pi(\theta|\vx{x}) = \frac{f(\vx{x},\theta)}{m(\vx{x})}\]
\end{definition}

Para finalizar esta sección, cabe destacar es que es posible no exigirle a la distribución de probabilidad a priori que integre, es decir, podrían distribuir una probabilidad infinita sobre $\Theta$. En tal caso se dice que la distribución es \textit{impropia}. Pese a su carácter impropio estas distribuciones nos pueden permitir hacer inferencias correctas.


\subsection{Estadística clásica vs bayesiana}

Veamos ahora las diferencias entre la inferencia clásica y la bayesiana. En la inferencia clásica destacan las siguientes características:

\begin{itemize}
	\item El concepto de probabilidad está limitado a aquellos sucesos en los que se pueden definir frecuencias relativas.
	\item $\theta$ es un valor fijo, pero desconocido.
	\item Se usa el concepto de intervalo de confianza.% (AÑADIR SI ESO)
	\item El método de muestreo es muy importante.
	\item Se pueden usar estimadores de máxima verosimilitud o estimadores insesgados.
\end{itemize}

Por su parte, en la inferencia bayesiana destacan:

\begin{itemize}
	\item Podemos establecer probabilidades previas para cualquier suceso.
	\item $\theta$ es una variable que sigue una distribución de probabilidad.
	\item Se usa el concepto de intervalo de credibilidad para $\theta$.% (AÑADIR SI ESO)
	\item El método de muestreo no importa; solo importan los datos.
	\item Se utilizan estimadores diferentes según la utilidad; la estimación es un problema de decisión.
\end{itemize}

Uno de los aspectos más criticados de la estadística bayesiana es el grado de subjetividad a la que se expone la inferencia por el hecho de que es el experimentador quien define la distribución a priori. En cualquier caso, en lo que hay coincidencia es en que si hay información sobre $\theta$ esta tiene que ser utilizada en la inferencia.

$ $ \newline
Como acabamos de decir, una parte muy importante en la inferencia bayesiana es la selección de la distribución a priori. En muchos casos, si no disponemos de una distribución clara para modelar $\theta$ es posible considerar distribuciones específicas que permitan simplificar los cálculos de la distribución a posteriori. A continuación estudiaremos distintos medios para seleccionar estas distribuciones.

\subsection{Familias conjugadas}

La principal dificultad que surge en los problemas de inferencia bajo la perspectiva bayesiana es tanto la confianza que se pueda esperar de la distribución a priori como el cálculo de la distribución a posteriori. La primera cuestión es importante ya que la inferencia que se realice posteriormente puede depender de la elección hecha de la distribución inicial, razón por la cual en muchos casos se recurre a distribuciones no informativas, que no imponen unas condiciones muy fuertes sobre el parámetro, o bien se puede aprovechar parte de la información muestral para mejorar la distribución inicial.% dando origen a las denominadas distribuciones intrínsecas a priori, de gran auge en la actualidad.

En cuanto a la segunda opción, el cálculo de la distribución a posteriori no tiene por qué conducir a una distribución tratable y, en ocasiones, hay que recurrir a métodos numéricos para poder trabajar con ellas. Centrándonos en esta última parte, interesa considerar familias de distribuciones a priori cuyas distribuciones a posteriori asociadas sean de fácil cálculo. En este sentido surge el concepto de familias a priori conjugadas.

\begin{definition}
	Sea $\mathcal{F} = \{\pi_i(\theta)|i\in I\}$ una familia de distribuciones a priori. $\mathcal{F}$ se dice que es conjugada respecto de la familia de densidades $\mathcal{P} = \{f(x|\theta)|\theta\in\Theta\}$, si para cualquier $\pi(\theta)\in\mathcal{F}$ y $f(x|\theta)\in \mathcal{P}$ se verifica que $f(x|\theta)\pi(\theta)\propto\Pi(\theta)\in\mathcal{F}$.

	Es decir, el producto de cualesquiera dos distribuciones de ambas familias vuelve a ser, salvo constante, una distribución de la familia de distribuciones a priori.
\end{definition}

Tener una familia de distribuciones conjugadas a priori nos permite simplificar en gran medida el cálculo de la distribución a posteriori, pues para el cálculo de la distribución marginal de $x$ (la integral) en el denominador del cociente, tenemos, salvo constantes una integral de una función de la familia $\mathcal{F}$, que sabemos que integra 1. Las constantes son las mismas en numerador y denominador, dando lugar así a una distribución a posteriori de la misma familia, como se explica en la siguiente observación.

\begin{remark}
	Se tienen las siguientes condiciones equivalentes:
	\begin{enumerate}[label=\roman*]
		\item $f(x|\theta)\pi(\theta) \propto \Pi(\theta) \in \mathcal{F}$
		\item $\pi(\theta|x)=\Pi(\theta)\in\mathcal{F}$
	\end{enumerate}
	Es decir, una familia $\mathcal{F}$ de distribuciones a priori es conjugada respecto a la familia dada si y solo si las distribuciones a posteriori pertenecen de nuevo a $\mathcal{F}$.
\end{remark}

Es posible calcular las distribuciones conjugadas para las familias de distribuciones clásicas, obteniendo de nuevo otras distribuciones clásicas.

\begin{prop}
	$ $ \newline
	\begin{itemize}
		\item La familia de distribuciones Beta es una familia de distribuciones conjugada para las distribuciones de Bernouilli, binomiales y binomiales negativas.

		\item La familia de distribuciones Gamma es una familia de distribuciones conjugada para las distribuciones de Poisson.% y (exponenciales (definir)).

		\item La familia de distribuciones normales es una familia de distribuciones conjugada para la familia de distribuciones normales con varianza conocida.
	\end{itemize}
\end{prop}

Veamos algún ejemplo de los proporcionados por la proposición anterior.

\begin{ex}
	Veamos algún ejemplo de los proporcionados por la proposición anterior. Vamos a considerar una distribución de Poisson de parámetro $\lambda > 0$, y consideramos como distribución a priori una Gamma de parámetros $\alpha$ y $\beta$. Dada una muestra $x_1,\dots,x_n$ se tiene:

	\[f(\vx{x}|\lambda) = \frac{e^{-n\lambda}\lambda^{\sum{x_i}}}{\prod{x_i!}}\]
	\[\pi(\lambda)=\frac{\beta^{\alpha}\lambda^{\alpha-1}e^{-\beta\lambda}}{\Gamma(\alpha)}\]

	\[f(\vx{x}|\lambda)\pi(\lambda)=\frac{e^{-n\lambda}\lambda^{\sum{x_i}}\lambda^{\alpha-1}e^{-\beta\lambda}\beta^\alpha}{\prod{x_i!}\Gamma(\alpha)}\]
	\[=\frac{\beta^{\alpha}}{\prod{x_i!}\Gamma(\alpha)}\lambda^{\sum{x_i}+\alpha-1}e^{-(\beta+n)}\propto Gamma\left(\lambda|\alpha+\sum{x_i},\beta+n\right) \]

	Es decir, para variables aleatorias de Poisson de parámetro $\lambda$, escogiendo una distribución a priori Gamma de parámetros $\alpha$ y $\beta$, y para los datos observados, obtenemos como distribución de $\lambda$ a posteriori una nueva Gamma, esta vez de parámetros $\alpha + \sum{x_i}$ y $\beta+n$.

	Notemos que la conjugación nos ha permitido evitar el cálculo de la distribución marginal de $x$. Si optamos por calcularla, obtendríamos:

	[PROXIMAMENTE]

	Llegando de nuevo al mismo resultado.
\end{ex}




\begin{definition}
    Sea $\{f(x | \theta): \theta \in \Theta \}$ una familia de distribuciones con parámetro $\theta \in \Theta$. La distribución a priori de Jeffreys se define como $\pi^{J}(\theta) \propto \sqrt{\mathcal{I}_X(\theta)}$.
\end{definition}

\begin{ex}
    Vamos a estudiar la distribución a priori de Jeffreys para la distribución binomial. En el ejemplo \ref{ex:fisher:binom} se calculó la función de información de Fisher para la distribución binomial. A partir de los resultados obtenidos tenemos que $\pi^J(\theta) \propto \theta^{-1/2} (1 - \theta)\theta^{-1/2}$. Por tanto, $\pi^J(\theta)$ sigue una distribución $beta(1/2,1/2)$. La distribución a posteriori para $x = (x_1, \ldots, x_k)$ viene dada por
    \[\pi(\theta; x) \propto \pi(\theta) \prod_{i = 1}^k f(x_i; \theta) \propto \theta^{\sum x_i -1/2} (1 - \theta)^{\sum (n-x_i) -1/2},\]
    esto es, $\pi(\theta;x)$ sigue una distribución $beta(k\overline{x} -1/2, k(n - \overline{x}) - 1/2)$. Recordando el Corolario \ref{cor:beta:esp} podemos calcular la esperanza y la varianza de la distribución a posteriori:
    \[E[\pi(\theta; x)] = \frac{k\overline{x} -1/2}{kn - 1} = \frac{\overline{x} -1/(2k)}{n - 1/k};\]
    \[Var(\pi(\theta; x)) = \frac{(k\overline{x} -1/2)(k(n - \overline{x}) - 1/2)}{(kn - 1)^2(kn)} = \frac{(\overline{x} -1/{2k})((n - \overline{x}) - 1/(2k))}{kn(n - 1/k)^2}.\]
    Para $k$ lo suficientemente grande $E[\pi(\theta; x)] \approx \overline{x} / n$, que es el estimador máximo verosímil. Por tanto, cuando $k \to \infty$ obtenemos que $Var(\pi(\theta; x)) \to 0$ y $E[\pi(\theta; x)] \to \theta_0$.
\end{ex}

Notemos que la distribución de Jeffreys podría ser impropia, es decir, no integrable. Veámoslo con el siguiente ejemplo:

\begin{ex}
	Poisson.
\end{ex}

\subsection{Convergencia de distribuciones a posteriori}

Nos planteamos en este punto la convergencia de las distribuciones a posteriori cuando el tamaño de las muestras crece.

Supongamos que queremos hacer inferencia sobre un fenómeno que sigue una distribución $f(x|\theta_0)$, con una medida de probabilidad $P_{\theta_0}$. Conocemos la familia de distribuciones $\{f(x|\theta)|\theta\in\Theta\}$. Queremos ver, dada una distribución a priori $\pi(\theta)$, si converge la distribución a posteriori $\pi(\theta|X_1,\dots,X_n)$, para las variables i.i.d. $X_1,\dots,X_n \sim f(x|\theta_0)$,  cuando $n \to \infty$.

En general, si el espacio paramétrico no es discreto, el estudio de la convergencia de la distribución marginal no es sencillo. Estudiaremos la convergencia de la distribución a posteriori cuando el espacio paramétrico es $\Theta = \{\theta_1,\dots,\theta_k\}$ discreto.


\begin{thm}
	En las condiciones anteriores, se tiene que:

	\[\pi(\theta|X_1,\dots,X_n) \xrightarrow[n\to\infty]{P_{\theta_0}} \theta_0\]
\end{thm}

\begin{proof}

	Supongamos $\Theta = \{\theta_1,\dots,\theta_k\}$. La distribución a priori viene determinada por las probabilidades de cada $\theta_j$ y vendrá dada por $\pi(\theta_j)=p_j$, con $p_j\in[0,1]$ y $\sum_{i=1}^{k}{p_i}=1$, para $j=1,\dots,k$. Podemos suponer también que $t\in\{1,\dots,k\}$ es el índice del parámetro que corresponde al verdadero valor, es decir, $\theta_t = \theta_0$.


	Consideramos las variables aleatorias i.i.d. $X_1,\dots,X_n$ con distribución $f(x|\theta_t)$ y una muestra $x_1,\dots,x_n$. La distribución a posteriori en este caso discreto vendrá dada, para cada $\theta_i$, por:

	\[\pi(\theta_i|x_1,\dots,x_n) = \frac{p_i \prod_{j=1}^n{f(x_j|\theta_i)}}{\sum_{r=1}^k{p_r\prod_{j=1}^n{f(x_j|\theta_r)}}}\]

	Multiplicando numerador y denominador por $\left(\prod_{j=1}^n{f(x_j|\theta_t)}\right)^{-1}$ obtenemos:

	\begin{equation} \label{eq:tcfd1}
		\pi(\theta_i|x_1,\dots,x_n) = \frac{p_i \prod_{j=1}^n{\frac{f(x_j|\theta_i)}{f(x_j|\theta_t)}}}{\sum_{r=1}^k{p_r\prod_{j=1}^n{\frac{f(x_j|\theta_r)}{f(x_j|\theta_t)}}}}
	\end{equation}


	Estudiemos la convergencia de $\prod_{j=1}^n{\frac{f(x_j|\theta_i)}{f(x_j|\theta_t)}}$. Si $i=t$, claramente tenemos que el resultado es 1. En caso contrario, tomando logaritmos, obtenemos:

	\begin{equation} \label{eq:tcfd2}
	\log{\prod_{j=1}^n{\frac{f(x_j|\theta_i)}{f(x_j|\theta_t)}}} = \sum_{j=1}^{n}{log{\frac{f(x_j|\theta_i)}{f(x_j|\theta_t)}}} = n\left(\frac{1}{n}\sum_{j=1}^{n}{log{\frac{f(x_j|\theta_i)}{f(x_j|\theta_t)}}}\right)
	\end{equation}

	Ahora, las variables aleatorias $Z_i \sim \log{\frac{f(x_j|\theta_i)}{f(x_j|\theta_t)}}$ son i.i.d, luego por las leyes de los grandes números el término $\frac{1}{n}\sum_{j=1}^{n}{log{\frac{f(x_j|\theta_i)}{f(x_j|\theta_t)}}}$ converge en probabilidad $P_{\theta_t}$ a la esperanza de cualquiera de ellas, $E\left[\log{\frac{f(x_j|\theta_i)}{f(x_j|\theta_t)}}\right]$. Además, como consecuencia de la desigualdad de la información (proposición \ref{prop:desigualdad}), dicha esperanza es un valor estrictamente negativo. En consecuencia, a partir de la expresión obtenida en (\ref{eq:tcfd2}), podemos concluir que:

	\[\log{\prod_{j=1}^n{\frac{f(x_j|\theta_i)}{f(x_j|\theta_t)}}} \xrightarrow[n\to\infty]{P_{\theta_t}} -\infty\]

	La continuidad del logaritmo nos asegura que $\prod_{j=1}^n{\frac{f(x_j|\theta_i)}{f(x_j|\theta_t)}} \xrightarrow[n\to\infty]{P_{\theta_t}} 0$.

	Finalmente, aplicando lo que acabamos de obtener a (\ref{eq:tcfd1}), concluimos que:

	\begin{itemize}
		\item Si $i \ne t$, $\pi(\theta_i|X_1,\dots,X_n) \xrightarrow[n\to\infty]{P_{\theta_t}} \frac{0}{p_t+\sum{0}} = 0$
		\item Si $i = t$, $\pi(\theta_i|X_1,\dots,X_n) \xrightarrow[n\to\infty]{P_{\theta_t}} \frac{p_t}{p_t+\sum{0}} = 1$
	\end{itemize}

	Pero esto es equivalente a decir que la distribución a posteriori degenera a $\theta_t$ en probabilidad $P_{\theta_t}$.
\end{proof}

\begin{remark}
	Observemos que en la convergencia de la distribución a posteriori no ha influido la distribución a priori escogida. Esto nos indica que cuando el tamaño de la muestra es grande, la distribución que hayamos elegido a priori no va a tener mucha influencia sobre la distribución que utilizaremos para realizar inferencia.
\end{remark}

\begin{thebibliography}{99}
\bibitem{gamma} Proof Wiki, Euler's Reflection Formula, \url{https://proofwiki.org/wiki/Euler%27s_Reflection_Formula}.
\bibitem{cauchy} Wikipedia, Residue theorem, \url{https://en.wikipedia.org/wiki/Residue_theorem#Example}.
\bibitem{leibniz} Wikipedia, Leibniz integral rule, \url{https://en.wikipedia.org/wiki/Leibniz_integral_rule}.
\bibitem{char} Davide Giraudo, No first moment and differentiable characteristic function, \url{http://math.stackexchange.com/questions/793788/continuous-probability-distribution-with-no-first-moment-but-the-characteristic}
\end{thebibliography}
\end{document}
