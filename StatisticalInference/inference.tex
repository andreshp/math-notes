%%%%%%%%%%%%%%%%%%%%%%%%%%%%%%%%%%%%%%%%%%%%%%%%%%%%%%%%%%%%%%%%%%%%%%%%%%%%%%%%%%%%%%%%%%%%%%%%%%%%%%
% Plantilla básica de Latex en Español.
%
% Autor: Andrés Herrera Poyatos (https://github.com/andreshp)
%
% Es una plantilla básica para redactar documentos. Utiliza el paquete fancyhdr para darle un
% estilo moderno pero serio.
%
% La plantilla se encuentra adaptada al español.
%
%%%%%%%%%%%%%%%%%%%%%%%%%%%%%%%%%%%%%%%%%%%%%%%%%%%%%%%%%%%%%%%%%%%%%%%%%%%%%%%%%%%%%%%%%%%%%%%%%%%%%%

%-----------------------------------------------------------------------------------------------------
%	TEX OPTIONS
%-----------------------------------------------------------------------------------------------------

% !TEX program = pdflatex
% !TEX option = -shell-escape
% !TEX language = es

%-----------------------------------------------------------------------------------------------------
%	INCLUSIÓN DE PAQUETES BÁSICOS
%-----------------------------------------------------------------------------------------------------

\documentclass{article}

%-----------------------------------------------------------------------------------------------------
%	SELECCIÓN DEL LENGUAJE
%-----------------------------------------------------------------------------------------------------

% Paquetes para adaptar Látex al Español:
\usepackage[spanish,es-noquoting, es-tabla, es-lcroman]{babel} % Cambia
\usepackage[utf8]{inputenc}                                    % Permite los acentos.
\selectlanguage{spanish}                                       % Selecciono como lenguaje el Español.

%-----------------------------------------------------------------------------------------------------
%	SELECCIÓN DE LA FUENTE
%-----------------------------------------------------------------------------------------------------

% Fuente utilizada.
\usepackage{courier}                    % Fuente Courier.
\usepackage{microtype}                  % Mejora la letra final de cara al lector.


%-----------------------------------------------------------------------------------------------------
%	LICENCIA
%-----------------------------------------------------------------------------------------------------

\usepackage[
    type={CC},
    modifier={by},
    version={4.0},
]{doclicense}

%-----------------------------------------------------------------------------------------------------
%	ESTILO DE PÁGINA
%-----------------------------------------------------------------------------------------------------

% Estílo de capítulos (usar book en lugar de article).
%\usepackage[Lenny]{fncychap}

% Paquetes para el diseño de página:
\usepackage{fancyhdr}               % Utilizado para hacer títulos propios.
\usepackage{lastpage}               % Referencia a la última página. Utilizado para el pie de página.
\usepackage{extramarks}             % Marcas extras. Utilizado en pie de página y cabecera.
\usepackage[parfill]{parskip}       % Crea una nueva línea entre párrafos.
\usepackage{geometry}               % Asigna la "geometría" de las páginas.

% Se elige el estilo fancy y márgenes de 3 centímetros.
\pagestyle{fancy}
\geometry{left=3cm,right=3cm,top=3cm,bottom=3cm,headheight=1cm,headsep=0.5cm} % Márgenes y cabecera.
% Se limpia la cabecera y el pie de página para poder rehacerlos luego.
%\fancyhfb{}

% Espacios en el documento:
\linespread{1.1}                        % Espacio entre líneas.
\setlength\parindent{0pt}               % Selecciona la indentación para cada inicio de párrafo.

% Cabecera del documento. Se ajusta la línea de la cabecera.
\renewcommand\headrule{
	\begin{minipage}{1\textwidth}
	    \hrule width \hsize
	\end{minipage}
}

% Texto de la cabecera:
\lhead{A. Herrera, N. Rodríguez, J. Poyatos, M. Ruiz, J.L. Suárez}                          % Parte izquierda.
\chead{}                                    % Centro.
\rhead{\subject \ - \doctitle}              % Parte derecha.

% Pie de página del documento. Se ajusta la línea del pie de página.
\renewcommand\footrule{
\begin{minipage}{1\textwidth}
    \hrule width \hsize
\end{minipage}\par
}

\lfoot{}                                                 % Parte izquierda.
\cfoot{}                                                 % Centro.
\rfoot{Página\ \thepage\ de\ \protect\pageref{LastPage}} % Parte derecha.

%-----------------------------------------------------------------------------------------------------
%	Secciones
%-----------------------------------------------------------------------------------------------------

\newcommand{\importsection}[1]{\input{./Sections/#1}}           % Include sections from sections directory.

%-----------------------------------------------------------------------------------------------------
%	PORTADA
%-----------------------------------------------------------------------------------------------------

% Elija uno de los siguientes formatos.
% No olvide incluir los archivos .sty asociados en el directorio del documento.
\usepackage{title1}
%\usepackage{title2}
%\usepackage{title3}

%----------------------------------------------------------------------------------------
%   MATEMÁTICAS
%----------------------------------------------------------------------------------------

\usepackage{mathematics} % Llama al paquete de matemáticas que se adjunta con la plantilla

% Pone una tilde debajo de la palabra.
\def\utilde#1{\mathord{\vtop{\ialign{##\crcr
$\hfil\displaystyle{#1}\hfil$\crcr\noalign{\kern1pt\nointerlineskip}
$\hfil\widetilde{}\hfil$\crcr\noalign{\kern-5pt\nointerlineskip}}}}}

\usepackage{comment}

%----------------------------------------------------------------------------------------
%   GRÁFICOS
%----------------------------------------------------------------------------------------

\usepackage{pgfplots}
\usepackage{tikz}
% Load the library (Descomentar para SOs distintos de Windows)
\usetikzlibrary{external}
% Enable the library !!!>>> MUST be in the preamble <<<!!!!
%\tikzexternalize
\pgfplotsset{compat=newest}

\usepackage{float}

%----------------------------------------------------------------------------------------
%   ENLACES
%----------------------------------------------------------------------------------------
\usepackage{hyperref}
\hypersetup{
    colorlinks   = true,   % Quita las cajas y añade un color al texto.
    % Tipos de enlaces cuyo color se puede configurar:
    linkcolor    = [rgb]{0,0.2,0.5},        % Por defecto red
    anchorcolor  = gray,        % Por defecto black
    citecolor    = magenta,     % Por defecto green
    filecolor    = red,         % Por defecto cyan
    menucolor    = green,       % Por defecto red
    runcolor     = red,         % Por defecto cyan
    urlcolor     = cyan        % Por defecto magenta
}

%-----------------------------------------------------------------------------------------------------
%	TÍTULO, AUTOR Y OTROS DATOS DEL DOCUMENTO
%-----------------------------------------------------------------------------------------------------

% Título del documento.
\newcommand{\doctitle}{Apuntes}
% Subtítulo.
\newcommand{\docsubtitle}{}
% Fecha.
\newcommand{\docdate}{\date}
% Asignatura.
\newcommand{\subject}{Inferencia Estadística}
% Autor.
\newcommand{\docauthor}{Andrés Herrera Poyatos \\ Nuria Rodríguez Barroso \\ Javier Poyatos Amador \\ María del Mar Ruiz Martín \\ Juan Luis Suárez Díaz}
\newcommand{\docaddress}{Universidad de Granada}
\newcommand{\docemail}{}%andreshp9@gmail.com}


%-----------------------------------------------------------------------------------------------------
%	RESUMEN
%-----------------------------------------------------------------------------------------------------

% Resumen del documento. Va en la portada.
% Puedes también dejarlo vacío, en cuyo caso no aparece en la portada.
\newcommand{\docabstract}{}
%\newcommand{\docabstract}{En este texto puedes incluir un resumen del documento. Este informa al lector sobre el contenido del texto, indicando el objetivo del mismo y qué se puede aprender de él.}

\begin{document}

\hypersetup{pageanchor=false}
\maketitle
\hypersetup{pageanchor=true}

%-----------------------------------------------------------------------------------------------------
%	ÍNDICE
%-----------------------------------------------------------------------------------------------------

% Profundidad del Índice:
\setcounter{tocdepth}{2}

\newpage
\tableofcontents
\vspace*{\fill}
\doclicenseThis
\newpage

%-----------------------------------------------------------------------------------------------------
%	SECCIÓN 1
%-----------------------------------------------------------------------------------------------------

\importsection{Intro.tex}
\pagebreak
\importsection{Familias.tex}
\pagebreak
\importsection{Estimacion.tex}
\pagebreak
\importsection{Exponencial.tex}
\pagebreak
\importsection{Hipotesis.tex}

\pagebreak


\pagebreak
\importsection{Bayesiana.tex}
\pagebreak

\section{Probabilidades subjetivas}

   Tras haber realizado un estudio de ambos enfoques, hemos llegado a la conclusión de que el enfoque bayesiano presenta un buen funcionamiento para muestras de tamaño pequeño. Sin embargo, en numerosas ocasiones no habrá apenas observaciones del experimento, o incluso no existirá ninguna. Un ejemplo de esto podría ser un asesor financiero que debe desarrollar un informe de riesgo de inversión en un nuevo producto. No podrá aplicar ninguno de los dos enfoques estudiados pues no cuenta con ninguna información previa del producto.

    De la necesidad de asignar probabilidades a este tipo de sucesos surgen las probabilidades subjetivas. Para el desarrollo de de dichas probabilidades el experto hará uso de sus propias opiniones, creencias o similitud con problemas de la misma índole. Continuando con el ejemplo anterior, el asesor financiero desarrollará una probabilidad en base a información sobre la empresa que lanza el producto, la competencia actual en el mercado, tendencias actuales de la bolsa, o cualquier otro factor que estime influyente.

    Para la formalización de dicho método necesitamos estar trabajando realmente con una probabilidad, por lo que se realiza una formalización a partir de axiomas para asegurar que trabajamos con una verdadera probabilidad.

    \begin{definition}
    Dado un espacio probabilístico $(\Theta , \mathcal{A})$, y $A,B \in A$, se define la relación binaria $\preceq $ como sigue:

    $A \preceq B$ A es menor o igual de creíble que B\\
    $A \prec B$ A es menos creíble que B\\
    $A \sim B$ A es igual de creíble que B\\
    \end{definition}


\subsection{Axiomas de la probabilidad subjetiva}
    Para trabajar con las probabilidades subjetivas se considera el espacio $(\Theta, \mathcal{A}, \preceq )$, sobre el que se desarrollan los siguientes axiomas:


    \begin{axiom}
        Exhaustividad de la relación binaria
        $$A\preceq B \vee B\preceq A\ \ \forall A,B\in \mathcal{A} $$
    \end{axiom}

    \begin{axiom}
        Sean $A_1,A_2,B_1,B_2 \in \mathcal{A}$ verificando $A_1A_2 = B_1B_2 = \emptyset$ y $A_i \preceq B_i, i = 1,2$. Entonces se verifica que
        $$A_1+A_2 \preceq B_1+B_2$$
    \end{axiom}


    \begin{axiom}
        Si $A\in \mathcal{A}$ entonces $\emptyset \preceq A$. Además $\emptyset \prec \Theta$
    \end{axiom}

    \begin{axiom}
        Si $B\in \mathcal{A}$, $A_i \in \mathcal{A}\ \forall i \in \mathbb{N}$ con $B\preceq ... \subset A_2 \subset A_1$, entonces se verifica que
        $$B\preceq \bigcap\limits_{i = 1}^{\infty}A_i$$
    \end{axiom}

    \begin{axiom}
        Existe una variable aleatoria uniformemente distribuida, esto es, $X:\Theta \longrightarrow [0,1]$ variable aleatoria verificando que
        $$X^{-1}({I_1}) \preceq X^{-1}({I_2}) \Longleftrightarrow \lambda(I_1)\preceq \lambda(I_2)$$
        Donde se está considerando el espacio probabilístico $([0,1],\mathcal{B}_{[0,1]},\lambda)$, siendo $\lambda $ la medida de Lebesgue en [0,1] y $I_1,I_2\in [0,1]$
    \end{axiom}


    A partir de dichos axiomas se construye la probabilidad subjetiva como una $X^{-1}(a,b)=G(a,b)$, donde G es una función verificando
    $$G(a,b) \sim G[a,b) \sim G(a,b] \sim G[a,b]$$
    $$ G(a_1,b_1)\preceq G(a_2,b_2) \Longleftrightarrow b_1-a_1 \leq b_2 - a_2$$
    Los siguientes resultados son inmediatos a partir de los axiomas anteriores.

    \begin{lem}
        Sean $A,B,D\in\mathcal{A}$ con $AD = BD = \emptyset$. Entonces
        $$ A\cup D \preceq B\cup D  \Longleftrightarrow A \preceq B$$
    \end{lem}

    \begin{lem}
        Sean $A,B,D\in\mathcal{A}$ con $A \preceq B, B \preceq D$. Entonces $A \preceq D$
    \end{lem}

    \begin{lem}
        Sean $A_1,A_2,...,A_n \in \mathcal{A} $ todos ellos distintos, y $ B_1,B_2,..,B_n \in \mathcal{B}$ distintos y verificando $A_i \preceq B_i \forall i = 1,2,...,n$. Entonces
        $$ \bigcup\limits_{i = 1}^{n}A_i \preceq \bigcup\limits_{i = 1}^{n}B_i$$
    \end{lem}

    \begin{lem}
        Sean $A,B \in \mathcal{A}$. Entonces
        $A \preceq B \Longleftrightarrow B^c \preceq A^c$
    \end{lem}

    \begin{lem}
        Sean $A,B\in \mathcal{A}$ tal que $A\subset B$. Entonces $A \preceq B$
    \end{lem}

    \begin{lem}
        Sean $B \in \mathcal{A}, \{A_i\}_{i \in \mathbb{N}}$ sucesión creciente de elementos de $\mathcal{A}$ tal que $A_i \preceq B\ \forall i \in \mathbb{N}$. Entonces
        $$\bigcup\limits_{i = 1}^{\infty}A_i \preceq B$$
    \end{lem}

    \begin{thm}
        Sean $\{A_i\}_{i\in \mathbb{N}} $ disjuntos, $\{B_i\}_{i\in \mathbb{N}} $ disjuntos con $A_i,B_i \in \mathcal{A} \ \forall i \in \mathbb{N}$ y $A_i \preceq B_i \ \forall i \in \mathbb{N}$ Entonces

        $$ \bigcup\limits_{i = 1}^{\infty}A_i \preceq \bigcup\limits_{i = 1}^{\infty}B_i $$
    \end{thm}

    El siguiente teorema garantiza que la probabilidad construida es una verdadera probabilidad.

    \begin{thm}\textbf{Existencia de la probabilidad subjetiva}\\
        Consideramos el espacio $(\Theta, \mathcal{A}, \preceq )$ bajo los axiomas anteriores. Entonces $\forall A \in \mathcal{A} \ \exists^1a^*\in[0,1]$ tal que $A \sim G[0,a^*]$.
        A ese punto $a^*$ se le llama probabilidad subjetiva de A, y notaremos $P(A)=a^*$.
    \end{thm}

    \begin{lem}
        Sean $A_i \in \mathcal{A} \ \forall i \in \mathbb{N}$ disjuntos. Entonces $P(\sum\limits_{i = 1}^{\infty}A_i) = \sum\limits_{i = 1}^{\infty}P(A_i)$. Además $P(\emptyset) = 0 $ y $P(\Theta) = 1$
    \end{lem}

\pagebreak
\begin{thebibliography}{99}
\bibitem{loeve} Probability theory, M. Loève, 1977, Springer-Verlag.
\bibitem{casella} Statistical Inference, G. Casella, R. L. Berger, segunda edicón (2002), Duxbury Advanced Series.
\bibitem{garthwaite} Statistical Inference, Garthwaite, Jollife y Jones, 1995, Oxford University Press on Demand.
\bibitem{gamma} Proof Wiki, Euler's Reflection Formula, \url{https://proofwiki.org/wiki/Euler%27s_Reflection_Formula}.
\bibitem{cauchy} Wikipedia, Residue theorem, \url{https://en.wikipedia.org/wiki/Residue_theorem#Example}.
\bibitem{leibniz} Wikipedia, Leibniz integral rule, \url{https://en.wikipedia.org/wiki/Leibniz_integral_rule}.
\bibitem{char} Davide Giraudo, No first moment and differentiable characteristic function, \url{http://math.stackexchange.com/questions/793788/continuous-probability-distribution-with-no-first-moment-but-the-characteristic}
\end{thebibliography}

\end{document}
