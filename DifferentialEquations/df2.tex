%%%%%%%%%%%%%%%%%%%%%%%%%%%%%%%%%%%%%%%%%%%%%%%%%%%%%%%%%%%%%%%%%%%%%%%%%%%%%%%%%%%%%%%%%%%%%%%%%%%%%%
% Apuntes de la asignatura ecuaciones diferenciales 2.
%
% Autores: Andrés Herrera Poyatos (https://github.com/andreshp)
%          Paco Luque Sánchez (https://github.com/pacron)
%%%%%%%%%%%%%%%%%%%%%%%%%%%%%%%%%%%%%%%%%%%%%%%%%%%%%%%%%%%%%%%%%%%%%%%%%%%%%%%%%%%%%%%%%%%%%%%%%%%%%%

%-----------------------------------------------------------------------------------------------------
%	INCLUSIÓN DE PAQUETES BÁSICOS
%-----------------------------------------------------------------------------------------------------

\documentclass{article}

% Utiliza el paquete de español.
\usepackage{spanish}
% Utiliza la plantilla para reports.
\usepackage{template}
% Utiliza una portada (elija una de las portadas disponibles y comente el resto).
\usepackage{title1}
%\usepackage{title2}

%-----------------------------------------------------------------------------------------------------
%	OTROS PAQUETES
%-----------------------------------------------------------------------------------------------------

\usepackage{mathematics}

%-----------------------------------------------------------------------------------------------------
%	DATOS DEL DOCUMENTO
%-----------------------------------------------------------------------------------------------------

\newcommand{\doctitle}{Apuntes}
\newcommand{\docsubtitle}{}
\newcommand{\docdate}{\date}
\newcommand{\subject}{Ecuaciones Diferenciales 2}
\newcommand{\docauthor}{Andrés Herrera Poyatos \\ Paco Luque Sánchez}
\newcommand{\docaddress}{Universidad de Granada}
\newcommand{\docemail}{andreshp9@gmail.com}
\newcommand{\docabstract}{}

\begin{document}

\maketitle

\section{Introducción}

En esta sección introducimos los conceptos básicos de la teoría de ecuaciones diferenciales ordinarias.

\subsection{Notación y terminología}

En esta sección haremos una pequeña introducción a los problemas que abordaremos en esta
asignatura. El objetivo es obtener propiedades de las soluciones de una ecuación diferencial que no
podemos resolver explícitamente. Concretamente, estudiaremos ecuaciones diferenciales ordinarias
(que abreviaremos EDO). Una EDO es una ecuación de la forma
\begin{equation}
  \label{eq:edo}
  x'(t) = f(t,x(t)) \text{ con } (t,x(t)) \in D,
  \tag{E}
\end{equation}

donde $D \subset \R \times \R^d$ es un conjunto abierto y arcoconexo y $f: D \to \R^d$ es una
función continua. Los elementos que intervienen en una EDO tienen una terminología específica que
introducimos a continuación:

\begin{itemize}
\item $D$: dominio de la ecuación.
\item $f$: campo de la ecuación.
\item $t$: tiempo o variable independiente.
\item $x(t)$: función incógnita o variable dependiente. En lo que sigue, por comodidad, la notaremos
  simplemente por $x$, sin especificar la variable de la que depende.
\end{itemize}

\begin{ex} \label{ex:ex} Veamos algunos ejemplos de ecuaciones diferenciales ordinarias.
  
  \begin{itemize}
  \item EDO escalar:
    \[x' = \frac{x}{t} \text{ con } t > 0,\] donde $D = (0, +\infty) \times \R$.
  \item Sistema de EDOs:
    \[
      \left.
        \begin{array}{r}
          x_1' = x_1x_2 + t \\
          x_2' = x_1 + x_2 + t^2
        \end{array}
      \right\}
    \]
    donde $D = \R \times \R^2$ y $f(t, x_1, x_2) = (x_1x_2 + t, x_1 + x_2 + t^2)$. \qedhere
  \end{itemize}
\end{ex}

  \begin{definition}[Concepto de solución]
    Una \emph{solución} de una EDO es una función $\varphi: I \to \R^d$ donde $I$ es un intervalo
    abierto no vacío, tal que:
    \begin{enumerate}
    \item $\varphi \in \mathcal{C}^1(I, \R^d)$;
    \item $(t, \varphi(t)) \in D$;
    \item $\varphi'(t) = f(t, \phi(t))$ para todo $t \in I$.
    \end{enumerate}
  \end{definition}

  \begin{ex} Sea $I \subset \R^+$ un intervalo y $a \in \R$. La función $\varphi: I \to \R$ dada por
    $\varphi(t) = at$ es una soluciónn de la EDO escalar propuesta en el Ejemplo \ref{ex:ex}. Esto
    prueba que una EDO puede tener infinitas soluciones.
  \end{ex}

  
  \begin{definition}[Problema de valores iniciales]
    Un \emph{problema de valores iniciales} o \emph{problema de Cauchy} (de aquí en adelante lo
    abreviaremos PVI) consiste en añadir a una EDO un dato o condición inicial, esto es, un par
    $(t_0, x_0) \in D$. Un PVI se escribe
    \begin{equation}
      \label{eq:pvi}
      \left\{
        \begin{array}{l}
          x' = f(t,x); \\
          x(t_0) = x_0.
        \end{array}
      \right.
      \tag{P}
    \end{equation}
    Una solución del PVI es una solución de la EDO que cumple
    \begin{enumerate}
    \item $t_0 \in I$;
    \item $\varphi(t_0) = x_0$.
    \end{enumerate}
  \end{definition}

  En lo que sigue referiremos a \eqref{eq:pvi} cuando trabajemos con un PVI genérico.

  \begin{ex} Sea $t_0 = 1$ y $x_0 \in \R$.  La función $\varphi: \R^+ \to \R$ dada por
    $\varphi(t) = x_0t$ es una soluciónn de la EDO escalar propuesta en el Ejemplo \ref{ex:ex} y
    verifica $\varphi(t_0) = x_0$. Nótese que la restricción de $\varphi$ a cualquier intervalo que
    contenga a $1$ es una solución del PVI dado por $(1, x_0)$. De todas las soluciones, nos
    interesan aquellas con el mayor dominio posible. La única solución maximal resultará ser
    $\varphi$. El concepto de maximalidad se estudia en la siguiente sección.
  \end{ex}

  \subsection{Unicidad de solución y soluciones maximales} \label{sec:unicidad-maximal}

  Consideremos un PVI
  \begin{equation}
    \left\{
      \begin{array}{l}
        x' = f(t,x); \\
        x(t_0) = x_0.
      \end{array}
    \right.
    \tag{P}
  \end{equation}

  En primer lugar introducimos los diferentes conceptos de unicidad de la teoría de ecuaciones
  diferenciales.
  
  \begin{definition}[Conceptos de unicidad] \
    \begin{itemize}
    \item Diremos que \eqref{eq:pvi} verifica la \emph{propiedad de unicidad local} si dadas
      $(I_1, \varphi_1), (I_2, \varphi_2) \in \Sigma$, existe $H$ intervalo abierto con $t_0 \in H$
      tal que $\varphi_1(t) = \varphi_2(t)$ para todo $t \in I_1 \cap I_2 \cap H$.

    \item Diremos que \eqref{eq:pvi} verifica la \emph{propiedad de unicidad en $H$}, con $H$
      intervalo abierto y $t_0 \in H$ si dadas $(I_1, \varphi_1), (I_2, \varphi_2) \in \Sigma$, se
      cumple que $\varphi_1(t) = \varphi_2(t)$ para todo $t \in I_1 \cap I_2 \cap H$.

    \item Diremos que \eqref{eq:pvi} verifica la \emph{propiedad de unicidad global} si dadas
      $(I_1, \varphi_1), (I_2, \varphi_2) \in \Sigma$, se cumple que $\varphi_1(t) = \varphi_2(t)$
      para todo $t \in I_1 \cap I_2$.
    \end{itemize}
  \end{definition}

  El siguiente resultado se deduce fácilmente de las definiciones de unicidad.

\begin{proposition}
  En el contexto actual se cumplen las siguientes afirmaciones.
  
  \begin{enumerate}
  \item El PVI \eqref{eq:pvi} verifica la propiedad de unicidad global si, y solo si, verifica la
    propiedad de unicidad en $\R$.
  \item El PVI \eqref{eq:pvi} verifica la propiedad de unicidad local si, y solo si, existe un
    intervalo abierto $H$ con $t_0$ tal que \eqref{eq:pvi} verifica la propiedad de unicidad en $H$.
  \end{enumerate}
\end{proposition}

\begin{corollary} \label{cor:global-local} Si el PVI \eqref{eq:pvi} verifica la propiedad de
  unicidad global, entonces verifica la propiedad de unicidad global.
\end{corollary}

El recíproco del Corolario \ref{cor:global-local} no es cierto como cabe esperar. No obstante, sí se
puede demostrar el resultado bajo una condición más restrictiva.

\begin{proposition} \label{prop:local-global} La EDO \eqref{eq:edo} verifica la propiedad de
  unicidad local para cualquier condición inicial $(t_0, x_0) \in D$ si, y solo si, verifica la
  propiedad de unicidad global para cualquier condición inicial $(t_0, x_0) \in D$.
\end{proposition}
\begin{proof}
  La implicación de derecha a izquierda es trivial. Veamos que se da el recíproco. Sean
  $(I_1, \varphi_1), (I_2, \varphi_2) \in \Sigma \eqref{eq:pvi}$, donde \eqref{eq:pvi} es el PVI
  asociado a $(t_0, x_0) \in D$. Veamos que coinciden en $I_1 \cap I_2$. Definimos
  $H = \{ t \in I_1 \cap I_2; \varphi_1(t) = \varphi_2(t)\}$. Por hipótesis tenemos que $t_0 \in H$
  y, por tanto, $H \neq \emptyset$. Además, $H$ es un cerrado relativo de $I_1 \cap I_2$ por ser el
  conjunto donde coinciden dos funciones continuas. Por último, razonamos que $H$ es abierto. En
  efecto, fijado $\tau \in H$, consideremos el PVI (\~P) asociado a $(\tau,
  \varphi_1(\tau))$. Nótese que $\varphi_1$ y $\varphi_2$ son soluciones de (\~P). Por la propiedad
  de unicidad local existe un intervalo abierto $\tilde{H}$ con $\tau \in \tilde{H}$ tal que
  $\varphi_1(t) = \varphi_2(t)$ para todo $t \in \tilde{H} \cap I_1 \cap I_2$.  Consecuentemente,
  $\tau \in \tilde{H} \cap I_1 \cap I_2 \subset H$. La prueba finaliza al recordar que
  $I_1 \cap I_2$ es conexo y, por tanto, el único abierto y cerrado no trivial de $I_1 \cap I_2$ es
  el propio $I_1 \cap I_2$.
\end{proof}

Es claro que un PVI admite infinitas soluciones ya que dada una solución cualquier restricción suya
a un intervalo que contenga a $t_0$ sigue siendo solución del PVI. En tal caso estamos tratando
esencialmente con la misma solución. Por tanto, interesa considerar solamente aquellas soluciones
con el mayor dominio posible. Para ello introducimos los conceptos de prolongación y solución
maximal.

\begin{definition}
  Sea
  $\Sigma = \Sigma\eqref{eq:pvi} = \{(I, \varphi) \ | \ \varphi: I \to \R^d \text{ es solución de
    \eqref{eq:pvi}}\}$. El par $(\widetilde{I}, \widetilde{\varphi}) \in \Sigma$ es una
  \emph{prolongación} de $(I, \varphi) \in \Sigma$ si $I \subsetneq \widetilde{I}$ y
  $\widetilde{\varphi}_{|I} = \varphi$ y lo denotaremos
  $(I, \varphi) >> (\widetilde{I}, \tilde{\varphi})$. Un elemento de $\Sigma$ es \emph{prolongable}
  si admite una prolongación.  El par $(I, \varphi) \in \Sigma$ es \emph{maximal} si no es
  prolongable.
\end{definition}

\begin{lemma} \label{lem:pro-max} Toda solución no maximal de \eqref{eq:pvi} puede prolongarse a una
  solución maximal.
\end{lemma}
\begin{proof}
  La demostración es consecuencia del Lemma de Zorn. Sea $\varphi_1: I_1 \to \R^d$ solución no
  maximal de \eqref{eq:pvi}. Basta aplicar este resultado sobre el conjunto
  \[ \{(I, \varphi) \ | \ \varphi: I \to \R^d \text{ es solución de \eqref{eq:pvi}} \text{ y } I_1
    \subsetneq I\} \] con la relación de orden $<<$ definida previamente. Nótese que este conjunto
  es no vació ya que $\varphi_1$ es prolongable. Encontramos pues una solución maximal que es
  prolongación de $\varphi_1$ como se quería.
\end{proof}

Transladamos ahora el concepto de unicidad global al ámbito de las soluciones maximales.

\begin{prop} \label{prop:unicidad-maximal} Consideremos el PVI \eqref{eq:pvi}. Éste verifica la
  propiedad de unicidad global si, y solo si, tiene una única solución maximal.
\end{prop}
\begin{proof}
  Supongamos que \eqref{eq:pvi} verifica la propiedad de unicidad global. Definimos
  $J = \cup_{(I, \varphi) \in \Sigma} I$, que es un intervalo ya que es unión de conexos de
  $\mathbb{R}$ que tienen un punto en común. Además, $J$ es abierto porque es unión de
  abiertos. Definimos la función $\Psi : J \to \mathbb{R}^d$ como $\Psi(t) = \varphi(t)$ para algún
  $(\varphi, I) \in \Sigma(P)$ con $t \in I$. Nótese que la función está bien definida gracias a la
  unicidad global. Además, $\Psi$ es solución de \eqref{eq:sol-maximal:pvi}. Es claro por
  construcción que $\Psi$ es la única solución maximal.

  Veamos que se verifica el recíproco. Sean $\varphi_1: I_1 \to \R^d$ y $\varphi_2: I_2 \to \R^d$
  dos soluciones de \eqref{eq:pvi}. Por el Lema \ref{lem:pro-max} podemos prolongarlas a dos
  soluciones maximales, que deben ser iguales por hipótesis. Por tanto, $\varphi_1$ y $\varphi_2$
  coinciden en $I_1 \cap I_2$ como se quería.
\end{proof}


\begin{ex} \label{ex:exp}
  Consideramos la EDO $x' = x$. En un curso básico de cálculo se demuestra
  que la derivada de la función exponencial es ella misma, esto es, la función exponencial verifica
  la ecuación diferencial. Es claro que cualquier constante por la función exponencial también
  verificará la ecuación diferencial. Vamos a demostrar que estas son las únicas soluciones
  maximales de la ecuación diferencial. Para ello vamos a probar que se verifica la propiedad de
  unicidad local para todo $(t_0, x_0) \in \R^2$. En primer lugar, $x_0 \ne 0$, tenemos que
  $\varphi(t)=x_0\exp(t-t_0)$ es solución (maximal) del PVI. Sea $x: I \to \R$ es solución del
  PVI. Como $x(t_0) = x_0 \ne 0$, encontramos un intervalo abierto $t_0 \subset J \subset I$ donde
  $x$ no se anula. En ese intervalo se tiene la igualdad $x'/x = 1$. Podemos integrar esta expresión
  para cada $t \in J$, obteniendo
  \[t - t_0 = \int_{t_0}^{t} \diff s = \int_{t_0}^{t} \frac{x'(s)}{x(s)}\diff s = \log(x(t)) -
    \log(x(t_0)).\] Aplicando la función exponencial en ambos lados de la igualdad deducimos que
  $x(t) = x(t_0) \exp(t-t_0)$ para todo $t \in J$ como se quería. El método que hemos utilizado para
  obtener la expresión de $x$ en un entorno de $t_0$ se denomina \emph{método de las variables
    separadas}, que generalizaremos más adelante. Por último, sabemos que la función $x = 0$ es
  solución de la ecuación diferencial. Tenemos que demostrar que si $x_0 = 0$, entonces cualquier
  solución es constantemente $0$ es un entorno de $t_0$. Para ello, razonamos por reducción al
  absurdo y utilizamos que $x$ se comporta como la exponencial cuando no se anula y, por tanto, no
  puede tomar el valor $0$ y ser continua al mismo tiempo. Se dejan rellenar los detalles de este
  último razonamiento al lector.
\end{ex}

El siguiente lema cuya demostración es trivial será utilizado a menudo para crear crear soluciones
que muestren que no se verifica la unicidad global.

\begin{lemma}
  Sean $\varphi_1: I_1 \to \R^d$ y $\varphi_2: I_2 \to \R^d$ soluciones del PVI \eqref{eq:pvi} y sea
  $\tau \in I_1 \cap I_2$ con $\varphi_1(\tau) = \varphi_2(\tau)$.  Entonces la función
  $\Psi: I_1 \cap I_2 \to \R^d$ dada por
  \[
    \psi(t) = \left\{
      \begin{array}{l}
        \varphi_1(t) \quad t \leq \tau \\
        \varphi_2(t) \quad t > \tau
      \end{array}
    \right.
  \]
  es solución del PVI.
\end{lemma}

\subsection{Ecuación integral de Volterra}

Sea $x: I \to \R$ una solución del PVI \eqref{eq:pvi}.  Reclordemos que $x$ verifica
$x'(s) = f(s,x(s))$ para todo $s \in I$. Si integramos la expresión, obtenemos para cada $t \in I$
la igualdad
\begin{equation}
  \label{eq:volterra}
  x(t) = x_0 + \int_{t_0}^t x'(s) \diff s = x_0 + \int_{t_0}^tf(s,x(s)) \diff s
  \tag{EV}
\end{equation}

A esta ecuación se le conoce como ecuación integral de Volterra, que abreviaremos EIV.

\begin{definition}
  Sea $\varphi: I \to \R^d$ con $I$ intervalo abierto. Diremos que $\varphi$ es solución de la
  ecuación integral de Volterra si

  \begin{enumerate}
  \item $\varphi \in \mathcal{C}(I, \R^d)$;
  \item $(t, \varphi(t)) \in D$ para todo $t \in I$;
  \item $t_0 \in I$;
  \item para cada $t \in I$ se verifica
    \[\varphi(t) = x_0 + \displaystyle\int_{t_0}^t f(s, \varphi(s))ds.\]
  \end{enumerate}
\end{definition}

Hemos visto que toda solución de \eqref{eq:pvi} es solución de la ecuación integral de Volterra
asociada. El recíproco también es cierto y se recoge en la siguiente proposición.

\begin{proposition}
  Sea $I$ un intervalo abierto y $\varphi: I \to \R^d$. Son equivalentes:
  \begin{enumerate}
  \item $\varphi$ es solución del PVI \eqref{eq:pvi};
  \item $\varphi$ es solución de la ecuación integral de Volterra \eqref{eq:volterra}.
  \end{enumerate}
\end{proposition}
\begin{proof}
  Ya sabemos que a) implica b). Veamos el recíproco. Sea $\varphi: I \to \R^d$ solución de
  \eqref{eq:volterra}.  Por el Teorema fundamental del cálculo aplicado a \eqref{eq:volterra}
  obtenemos que $\varphi$ es derivable y su derivada es $\varphi'(t) = f(t, \varphi(t))$. Puesto que
  $f$ es continua, obtenemos que $\varphi \in \mathcal{C}^1(I, \R^d)$. Por último, es claro que
  $\varphi(t_0) = x_0$.
\end{proof}

Vamos a ver un ejemplo práctico de cómo se relacionan los dos conceptos que hemos estado tratando.

\begin{ex}
  Resolver la EIV
  \[x(t) = 8 + \displaystyle\int_0^t x(s)^2 \diff s. \]
  

  \textbf{Solución:} El PVI asociado es
  \[
    \left\{
      \begin{array}{l}
        x'(t) = x(t)^2; \\
        x(0) = 8.
      \end{array}
    \right.
  \]

  Discutimos el caso de las soluciones constantes primero. Si la solución es constante tenemos que
  $0 = x^2$, que sólo ocurre si $x = 0$.  Esto no es posible ya que $x(0) = 8$. Por tanto, no hay
  soluciones constantes. Discutimos ahora el caso de las soluciones no constantes, aplicando de
  nuevo el método de variables separadas. Nótese que si $x: I \to \R$ es solución, donde $I$ es un
  intervalo abierto y $t_0 = 0 \in I$, tal que $x$ no se anula, entonces para cada $t \in I$ se
  tiene
  \[ t = \int_0^t \diff s = \int_{t_0}^t \frac{x'(s)}{x^2(s)}\diff t = \frac{-1}{x(t)} -
    \frac{-1}{x(t_0)} = \frac{-1}{x(t)} + \frac{1}{8}. \]

  Deducimos que debe verificarse
  \[x(t) = \frac{8}{1 - 8t}.\]
    
  Definimos $\varphi: (-\infty, \frac{1}{8}) \to \R$ con $\varphi(t) = \frac{8}{1-8t}$ y comprobamos
  que sea solución, Efectivamente tenemos que
  \[ \varphi(0) = 8 \quad \text{y} \quad \varphi'(t) = \frac{64}{(1-8t)^2} = \varphi^2(t). \]

  Por tanto, $\varphi$ es solución del PVI, y toda solución es igual a esta última en el mayor
  intervalo que contenga a $0$ donde no se anule. No es difícil ver que esto implica que $\varphi$
  es la única solución del PVI y, por tanto, de la EIV. No obstante, veremos este hecho formalmente
  en la siguiente sección.
\end{ex}

\section{Existencia y unicidad de solución} \label{sec:eu}

En este apartado proporcionamos los primeros resultados de existencia y unicidad para PVIs. En
primer lugar consideraremos EDOs en variables separadas, que ya han surgido en los ejemplos
anteriores. Posteriormente estudiaremos dos teoremas clásicos de la teoría de ecuaciones
diferenciales, el Teorema de unicidad Peano y el Teorema de Picard-Lindelöf.

\subsection{Variables separadas}

Hemos mencionado el método de variables separadas con anterioridad. La idea es obtener una igualdad
donde en un lado solamente intervenga la variable independiente mientras que en el otro solamente
intervenga la variable dependiente. De esta forma podemos integrar y aplicar el teorema del cambio
de variables para obtener una igualdad donde no intervenga la variable independiente. Este método
solo se puede aplicar a casos muy concretos. Introducimos estos casos en la siguiente definición.

\begin{definition}
  Una EDO \eqref{eq:edo} es de variables separadas si existen $J_1$, $J_2$ intervalos abiertos y
  $a: J_1 \to \R$, $g: J_2 \to \R$ funciones continuas tales que $D = J_1 \times J_2$ y
  $f(t,x) = a(t) g(x)$ para todo $(t,x) \in J_1 \times J_2$.  Un PVI es de variables separadas si la
  EDO asociada lo es. En tal caso escribimos
  \begin{equation}
    \label{eq:vs}
    \begin{cases}
      x' = a(t)g(x), \quad (t,x) \in J_1 \times J_2; \\
      x(t_0) = x_0.
    \end{cases}
    \tag{VS}
  \end{equation}
\end{definition}

\begin{thm}[Existencia y unicidad en variables separadas]
  Consideremos un PVI de variables separadas \eqref{eq:vs}. Se verifican las siguientes
  afirmaciones.
  \begin{enumerate}
  \item El PVI \eqref{eq:vs} tiene solución.
  \item Si $g(x_0) \neq 0$, entonces \eqref{eq:vs} verifica la propiedad de unicidad local.
  \item Si $g(x_0) = 0$ y existe $ G \in \mathcal{C} (J_2)$ tal que $G(x_0) = 0$ y existe
    $\delta > 0: G'(x) = \frac{1}{g(x)} \forall x \in J_2 \cap ]x_0, x_0 + \delta[$ y además
    $a(t_0) \neq 0$, entonces \eqref{eq:vs} no verifica la propiedad de unicidad local.
  \end{enumerate}
\end{thm}

\begin{proof}
  Comenzamos demostrando \textbf{i)}. Distinguimos dos casos:\\

  \textbf{Caso 1:} $g(x_0) = 0$. Entonces, $\varphi: \R \to \R$ tal que
  $\varphi(t) = x_0$ es una solución de $(P)$.\\

  \textbf{Caso 2:} $g(x_0) \neq 0$. Entonces, existe $I_1$ abierto tal que
  $x_0 \in I_1, g(x) \neq 0 \quad \forall x \in I_1$. Tomamos entonces la aplicación que asigna a
  $x$ el valor $\frac{1}{g(x)} \in \mathcal{C}(I_1)$.  Sea ahora la aplicación $ G: I_1 \to \R$ tal
  que $G(x_0) = 0, G'(x) = \frac{1}{g(x)}$.  Tenemos que $G$ es estrictamente monótona. Tomamos
  $I_2 = G(I_1)$ intervalo abierto. Entonces, $\exists G^{-1}: I_2 \to \R$ con $G^{-1}(I_2) =
  I_1$. Además, tenemos que $G^{-1} \in \mathcal{C}(I_2)$, con
  $(G^{-1})' (x) = \frac{1}{G'(G^{-1}(x))} = g(G^{-1}(x)) \forall x \in I_2$.  Sea ahora $A = A(t)$
  una primitiva de $a(t)$ tal que $A(t_0) = 0 \in I_2$.  Por continuidad, $\exists I_3 \subset J_1$
  abierto tal que $t_0 \in I_3$, $A(I_3) \subset I_2$. Definimos:
  \begin{align*}
    \varphi : & I_3 \to \R \in \mathcal{C}^1(I_3) \\
              & t \to G^{-1}(A(t))
  \end{align*}

  Tenemos que $t_0 \in I_3$, $\varphi(t_0) = G^{-1}(a(t_0)) = G^{-1}(0) = x_0$. Tenemos que
  $(t, \varphi(t)) \in J_1 \times J_2$ y
  $\varphi'(t) = g(G^{-1}(A(t))a(t) = a(t)g(\varphi(t)) \quad \forall
  t \in I_3$ \\

  Pasamos ahora a demostrar \textbf{ii)}. Sean $\varphi_1: H_1 \to \R$, $\varphi_2: H_2 \to \R$ dos
  soluciones de $(P)$.
  % TODO: Terminar demostración
\end{proof}

\subsection{Unicidad en el futuro y en el pasado. Teorema de unicidad de Peano}

Si $d = 1$, en algunos casos podemos obtener ``gratuitamente'' la propiedad de unicidad a la derecha
o a la izquierda de $t_0$ mediante el Teorema de unicidad de Peano. Introducimos a continuación la
nomenclatura pertinente.

\begin{definition}
  \label{def:unicidad-futuro}
  Diremos que el PVI \eqref{eq:pvi} verifica la propiedad de unicidad en el futuro si verifica la
  propiedad de unicidad en el intervalo $[t_0, +\infty[$. Análogamente, diremos que el PVI
  \eqref{eq:pvi} verifica la propiedad de unicidad en el pasado si verifica la propiedad de unicidad
  en el intervalo $]-\infty, t_0]$.
\end{definition}

\begin{thm}[Teorema de unicidad de Peano] \label{thm:peano} Consideramos un PVI \eqref{eq:pvi} con
  $d = 1$.
  \begin{enumerate}
  \item\label{item:peano:a} Si para cada $t \ge t_0$ la función $g(x) = f(t,x)$ es decreciente,
    entonces \eqref{eq:pvi} verifica la propiedad de unicidad en el futuro.
  \item\label{item:peano:b} Si para cada $t \le t_0$ la función $g(x) = f(t,x)$ es creciente,
    entonces \eqref{eq:pvi} verifica la propiedad de unicidad en el pasado.
  \end{enumerate}
\end{thm}
\begin{proof}
  Vamos a demostrar el apartado \ref{item:peano:a} ya que el otro apartado se demuestra de forma
  análoga. Sean $\varphi_1 \colon I_1 \to \R$ y $\varphi_1 \colon I_1 \to \R$. Definimos la función
  $h \colon I_1 \cap I_2 \to \R$ dada por $h(t) = (\varphi_1(t) - \varphi_2(t))^2$. La función $h$
  es derivable y su derivada viene dada por
  \[ h'(t) = 2(\varphi_1(t) - \varphi_2(t))(\varphi_1'(t) - \varphi_2'(t)) = 2(\varphi_1(t) -
    \varphi_2(t))(f(t, \varphi_1(t)) - f(t, \varphi_2(t))), \] que es menor o igual que $0$ para
  todo $t \in I_1 \cap I_2 \cap [t_0, +\infty[$. Por tanto, $h$ es decreciente en
  $I_1 \cap I_2 \cap [t_0, +\infty[$. Nótese que $h(t_0) = 0$ y $h \ge 0$. Por tanto, $h$ es
  constantemente $0$ en $I_1 \cap I_2 \cap [t_0, +\infty[$, esto es, $\varphi_1$ y $\varphi_2$
  coinciden en este intervalo como se quería.
\end{proof}

En este punto interesa introducir la ecuación dual en el tiempo, que nos servirá para trasladar los
resultados que conciernan al intervalo $[t_0, +\infty[$ al intervalo $]-\infty, t_0]$. Dado el PVI
\eqref{eq:pvi} y una solución $x : I \to \R^d$ de éste, nos preguntamos de qué PVI es solución la
función $y(t) = x(-t)$, definida en el intervalo $-I$. Claramente tenemos que $y(-t_0) =
x_0$. Además, para cada $t \in -I$ se verifica
\[ y'(t) = - x'(-t) = - f(-t, x(-t)) = - f(-t, y(t)).\]

En resumen, $y$ es solución del PVI

\begin{equation}
  \label{eq:pvi:dual}
  \begin{cases}
    y' = f(-t, y); \\
    y(-t_0) = x_0;
  \end{cases}
  \tag{D}
\end{equation}

que se conoce como \emph{problema dual} o \emph{ecuación dual en el tiempo}. Nótese que la
aplicación $\Lambda \colon \Sigma\eqref{eq:pvi} \to \Sigma\eqref{eq:pvi:dual}$ dada por
$\Lambda(\varphi(t)) = \varphi(-s)$ es biyectiva. La siguiente proposición, cuya demostración es
sencilla y se deja para el lector muestra la relación existente entre un PVI y su dual.

\begin{proposition} \label{prop:dual} El PVI \eqref{eq:pvi} verifica la propiedad de unicidad en el
  intervalo $I$ si, y solo si, el PVI dual \eqref{eq:pvi:dual} verifica la propiedad de unicidad en
  el intervalo $-I$. Como consecuencia se verifican las siguientes afirmaciones:
  
  \begin{enumerate}
  \item El PVI \eqref{eq:pvi} verifica la propiedad de unicidad local si, y solo si, el PVI dual
    \eqref{eq:pvi:dual} verifica la propiedad de unicidad local.
  \item El PVI \eqref{eq:pvi} verifica la propiedad de unicidad global si, y solo si, el PVI dual
    \eqref{eq:pvi:dual} verifica la propiedad de unicidad global.
  \end{enumerate}
\end{proposition}

Podemos utilizar la Proposición \ref{prop:dual} para demostrar el apartado \ref{item:peano:b} del
Teorema \ref{thm:peano} a partir del apartado \ref{item:peano:a}.

\begin{ex}
    Estudia la unicidad de solución del PVI
    \begin{equation} \label{eq:6:pvi}
      \begin{cases}
        x' = -t \sqrt[3]{x}; \\
        x(0) = 0.
      \end{cases}
    \end{equation}
    Claramente la función $x = 0$ es una solución de \eqref{eq:6:pvi}. Tenemos que
    $f(t,x) = - t \sqrt[3] x$. Para $t \ge 0$ fijo la función $\varphi(x) = f(t, x)$ es decreciente
    mientras que para $t \le 0$ fijo la función $\varphi(x) = f(t, x)$ creciente. Por tanto, el
    Teorema de unicidad de Peano nos dice que \eqref{eq:6:pvi} verifica la propiedad de unicidad en
    el futuro y en el pasado, esto es, verifica la propiedad de unicidad global. Por tanto, la única
    solución de \eqref{eq:6:pvi} es la trivial.
\end{ex}

\subsection{Teorema de Picard-Lindelöf} \label{sec:eu:pl}

En esta sección estudiamos el Teorema de Picard-Lindelöf, que también se conoce como Teorema de
Picard, Teorema de Cauchy - Lipschitz o Teorema de Existencia y Unicidad. Es el principal resultado
de existencia y unicidad de solución para PVIs. Existen numerosas versiones del resultado y, además,
múltiples demostraciones. Enunciamos el teorema a continuación.

\begin{thm}[Teorema de Picard-Lindelöf]
  Sean $D \subset \mathbb{R} \times \mathbb{R}^d$, $f: D \to \mathbb{R}^d$ continua y localmente
  lipschitziana respecto de la segunda variable. Dado $(t_0, x_0) \in D$, el PVI \eqref{eq:pvi}
  tiene solución y verifica la propiedad de unicidad global.
\end{thm}

En primer lugar, tenemos que introducir el concepto de ser lipschitziana respecto de la segunda
variable (Sección \ref{sec:eu:pl:lips}). Posteriormente desarrollaremos el denominado operador
integral de Volterra (\ref{sec:eu:pl:volterra}), que es la herramienta que se utiliza para demostrar
el Teorema de Picard-Lindelöf.

\subsubsection{Funciones lipschitzianas} \label{sec:eu:pl:lips}

Se asume que el lector tiene una cierta familiaridad con el concepto de función lipschitizana. Por
tanto, solo se enuncian los resultados que sean relevantes en lo que sigue, dejando las pruebas para
el lector.

\begin{definition}
  Sea $D \subset \R^d$ y sea $f \colon D \to \R^k$. La función $f$ es \emph{localmente
    lipschitziana}, que abreviaremos LL, si para cada $x \in D$ existe un entorno abierto
  $U \subset D$ de $x$ tal que $f$ es lipschitziana en $U$.
\end{definition}

\begin{proposition}
  Sea $f \in \mathcal{C}^1(A, \R^k)$, donde $A \subset \R^d$ es abierto convexo. Si
  $\frac{\partial}{\partial x}f$ está acotada en $A$, entonces $f$ es lipschitziana.
\end{proposition}

\begin{corollary}
  Sea $f \in \mathcal{C}^1(A, \R^k)$, donde $A \subset \R^d$ es abierto. Entonces $f$ es localmente
  lipschitziana.
\end{corollary}

\begin{definition}
  Sea $D \subset \R^d \times \R^m$ y sea $f \colon D \to \R^k$. La función $f(x,y)$ es
  \emph{lipschitziana respecto de la variable} $x \in \R^d$ si existe $M \ge 0$ tal que para cada
  $x,x' \in \R^d$, $y \in \R^m$ con $(x,y) \in D$ y $(x', y) \in D$ se tiene
  \[ ||f(x,y)-f(x',y)|| \le M ||x-x'||. \]
\end{definition}

\begin{definition}
  Sea $D \subset \R^d \times \R^m$ y sea $f \colon D \to \R^k$. La función $f(x,y)$ es
  \emph{localmente lipschitziana respecto de la variable} $x \in \R^d$ o \emph{LL respecto de la
    variable} $x \in \R^d$ si para cada $(p,q) \in D$ existe un entorno abierto $U \subset D$ de
  $(p,q)$ tal que $f$ es lipschitziana respecto de la variable $x$ en $U$.
\end{definition}

\begin{proposition}
  Sea $D \subset \R^d \times \R^m$ y sea $f \colon D \to \R^k$ continua y derivable respecto de la
  variable $x \in \R^{d}$. Si la función
  $\frac{\partial}{\partial x}f \colon D \to \mathcal{M}_{d,k}(\R)$ es continua, entonces $f$ es LL
  respecto de la variable $x$.
\end{proposition}

\subsubsection{Operador integral de Volterra} \label{sec:eu:pl:volterra}

Fijado una EIV \eqref{eq:volterra} queremos definir un operador $V \colon E \to E$ tal que a la
función $y\colon I \to \R^d$ le asigne la función $V(y)\colon I \to \R^d$ dada por
\begin{equation}
  \label{eq:1}
  V(y)(t) = x_0 + \displaystyle\int_{t_0}^t f(s,y(s))ds.
\end{equation}

Nótese que $y$ es solución de \eqref{eq:volterra} si, y solo si, $V(y) = y$. Queremos aplicar el
Teorema del punto fijo de Banach para asegurar la existencia y la unicidad de tal $y$. Para ello
necesitamos encontrar un conjunto $E \subset \{f\colon I\to\R^d \mid f \text{ es continua}\}$
apropiado que sea un espacio métrico completo. Además, $V$ debe ser una contracción en tal
espacio. En tal caso, a $V$ se le denomina \emph{operador integral de Volterra}.

En este contexto introducimos la siguiente notación. Dados $a,b > 0$ denotamos
\[R_{a,b}(t_0, x_0) = [t_0+a, t_0-a] \times \overline{B}(x_0, b).\]

Existen $a,b > 0$ tales que $R_{a,b}(t_0,x_0) \subset D$.  Consideramos el conjunto
$E = \overline{B}(x_0, b) \subset \mathcal{C}([t_0-a, t_0+a]; \mathbb{R}^d)$, que es un espacio
métrico completo con la norma infinito.

\begin{lemma} \label{lem:v:1} Sea $\varphi \in E$ y
  $M = \max \{||f(t,x)|| \mid (t,x) \in R_{a,b}(t_0, x_0)\}$. Si $aM \le b$, entonces
  $V(\varphi) \in E$.
\end{lemma}
\begin{proof}
  La comprobación es sencilla. Sea $t \in [t_0-a, t_0+a]$, tenemos que
  \[ ||V(\varphi)(t)-x_0|| \le \int_{t_0}^t ||f(s, \varphi(s))|| \diff s \le M(t-t_0) \le M a \le
    b.  \]

  De la arbitrariedad de $t$ se obtiene que $||V(\varphi)-x_0|| \le b$.
\end{proof}

\begin{remark}
  Si se verifican las hipótesis del Lema \ref{lem:v:1} y $\varphi: [t_0-a, t_0+a] \to \mathbb{R}^d$
  es solución de \eqref{eq:volterra}, entonces $(t, \varphi(t)) \in R_{a,b}$ para todo
  $t \in [t_0-a, t_0+a]$.
\end{remark}

\begin{lemma}
  Si la función $f$ es lipschitziana respecto de la segunda variable en $R_{a.b}(t_0, x_0)$ con
  constante de Lipschitz $L \ge 0$, entonces
  \[||V(\varphi) - V(\Psi)||_{\infty} \le La||\varphi - \Psi||_\infty,\]
\end{lemma}

\begin{proof}
  Sea $t \in [t_0-a, t_0+a]$. Tenemos que
  \[ (V(\varphi) - V(\Psi))(t) = \int_{t_{0}}^t \left[f(s, \varphi(s)) - f(s, \Psi(s))\right] \diff
    s.\] Tomamos normas y acotamos la integral
  \[|(V(\varphi) - V(\Psi))(t)| \le \left|\int_{t_{0}}^t \left|f(s, \varphi(s)) - f(s,
        \Psi(s))\right| \diff s\right| \le \left|\int_{t_{0}}^t L |\varphi(s) - \Psi(s)| \diff s
    \right| \le L a ||\varphi - \Psi||_{\infty}.\]

  De la arbitrariedad de $t$ se obtiene el resultado.
\end{proof}

\subsubsection{Demostración del Teorema de Picard-Lindelöf}

Gracias al operador integral de Volterra podemos dar una demostración sencilla del Teorema de
Picard-Lindelöf. Presentamos la demostración a continuación.

\begin{proof}
  Existe $U \subset D$ entorno de $(t_0, x_0)$ y $L \ge 0$ tal que $f$ es lipschitziana respecto de
  la segunda variable en $U$ con constante de Lipschitz $L \ge 0$. Existen
  $\overline{a}, b \in \mathbb{R}^+$ tal que $R_{\overline{a},b}(t_0, x_0) \subset U$. Por el
  teorema de Weierstrass, existe $M = \max \{||f(t,x)|| : (t,x) \in R_{\overline{a},b}(t_0,
  x_0)\}$. Tomamos $0 < a < \overline{a}$ tal que $a M < b$ y $a L < 1$. En el nuevo recinto
  $R_{a,b}(t_0, x_0)$ se cumplen todas las propiedades que se requieren para finalizar la
  demostración. Consideramos el espacio métrico completo
  \[E = \{\varphi \in \mathcal{C}([t_0-a, t_0+a], \mathbb{R}^d): ||\varphi(t)-x_0|| \le b \ \forall
    t \in [t_0-a, t_0+a]\}.\]

  Por los lemas anteriores tenemos que $V(E) \subset E$ y $V$ es contractiva. La existencia de
  solución se deduce del Teorema del punto fijo de Banach.

  Estudiemos ahora la unicidad local de \eqref{eq:pvi}. Consideremos dos soluciones
  $\varphi_1: I_1 \to \mathbb{R}$ y $\varphi_2: I_2 \to \mathbb{R}$. Podemos tomar $0 < a' < a$ tal
  que $[t_0-a', t_0+a'] \subset I_1 \cap I_2$ y, además, $||\varphi_i(t)-x_0|| < b$ para todo
  $t\in [t_0-a', t_0+a']$, $i \in \{1,2\}$. Nótese que se sigue verificando $a'M < b$ y $a'L <
  1$. Tomamos el espacio métrico completo
  \[E' = \{\varphi \in \mathcal{C}([t_0-a', t_0+a'], \mathbb{R}^d): ||\varphi(t)-x_0|| \le b \
    \forall t \in [t_0-a', t_0+a']\}.\]

  Tenemos que $\varphi_1, \varphi_2 \in E'$ son puntos fijos del operador integral de Volterra
  definido sobre $E'$. Como consecuencia del Teorema del punto fijo de Banach, $\varphi_1$ y
  $\varphi_2$ coinciden en $[t_0-a', t_0+a']$, lo que demuestra la unicidad local.

  Por último, de la arbitrariedad de $(t_0, x_0)$ deducimos que la EDO verifica la propiedad de
  unicidad local en cualquier punto $(t_0, x_0) \in D$. Podemos aplicar pues la Proposición
  \ref{prop:local-global} para completar la demostración.
\end{proof}

Recuérdese en este punto que el Teorema del punto fijo de Banach nos dice además que dado un
elemento $\varphi_0 \in E$, la sucesión $\varphi_n = V^n(\varphi)$ converge en norma a la solución
de \ref{eq:pl:pvi}. Puesto que estamos usando la norma infinito, esta convergencia equivale a la
convergencia uniforme. A la sucesión $\varphi_n$ se le denomina iterantes de Picard.

\begin{ex}[Iterantes de Picard]
  Consideramos el PVI
  \begin{equation}
    \begin{cases}
      x' = x^2; \\ x(0) = 1.
    \end{cases}
  \end{equation}
  Para este PVI obtenemos $f(t,x) = x^2$. En el entorno $U = \mathbb{R} \times [0,2]$ de $(0,1)$
  tenemos que $f$ es lipschitziana respecto de la segunda variable con constante de Lipschitz
  $L = 4$ y está acotada por $M = 4$. Tomamos pues $a = 1/8$. Podemos definir en $R_{a,b}(0,1)$ las
  iterantes de Picard, que nos permiten encontrar una aproximación numérica de la solución en caso
  de no conocer ésta.
\end{ex}

\subsection{Solución general}

Como consecuencia del Teorema de Picard-Lindelöf y los resultados de la Sección
\ref{sec:unicidad-maximal} obtenemos el siguiente corolario.

\begin{cor} \label{cor:picard:maximal} Sea $D \subset \mathbb{R} \times \mathbb{R}^d$ un conjunto
  abierto y $f : D \to \mathbb{R}^d$ continua y localmente lipschitziana respecto de la segunda
  variable. Entonces, para cualquier $(t_0, x_0) \in D$ el PVI \eqref{eq:pvi} tiene una única
  solución maximal.
\end{cor}

\begin{definition}
  Sea $D \subset \mathbb{R} \times \mathbb{R}^d$ y $f \colon D \to \mathbb{R}^d$ continua y localmente
  lipschitziana. Fijado $(t_0, x_0) \in D$ denotaremos $\alpha(t_0, x_0)$ y $\omega(t_0, x_0)$ al
  extremo inferior y superior, respectivamente, de la solución maximal del PVI asociado a la
  condición inicial $x(t_0) = x_0$. Definimos el conjunto
  \[\Omega = \{(t, t_0, x_0) \in \mathbb{R}\times\mathbb{R}\times\mathbb{R}^d : (t_0, x_0) \in D,
    \alpha(t_0, x_0) < t < \omega(t_0, x_0)\}\] y la llamada \emph{solución general}
  $X: \Omega \to \mathbb{R}^d$ a partir del Teorema de Picard-Lindelöf como aquella función
  diferenciable en la primera variable que verifica $X_t(t, t_0, x_0) = f(t, X(t, t_0, x_0))$ y
  $X(t_0, t_0, x_0) = x_0$. Esto es, la función $\varphi(t) = X(t, t_0, x_0)$ es la solución maximal
  del PVI asociado a $(t_0, x_0)$.
\end{definition}

Cuando no haya ambigüedad en la solución maximal que estamos considerando utilizaremos
$]\alpha, \omega[$ para indicar su dominio. Algunos autores escriben $]\omega_-, \omega_+[$ en lugar
de $]\alpha, \omega[$. No obstante, la notación que hemos escogido nos facilita la escritura.

\section{EDOs autónomas escalares}

La solución general de una EDO se simplifica enormemente cuando la función $f$ no depende de la
variable $t$. En tal caso, podemos obtener una gran cantidad de información acerca de las soluciones
de la EDO solamente utilizando la unicidad del Teorema de Picard - Lindelöf.

\begin{definition}
  Una EDO autónoma escalar es una EDO de la forma
  \begin{equation}
    \label{eq:edo:ae}
    x' = f(x), \ x \in I,
    \tag{AE}
  \end{equation}
  donde $I \subset \R$ es un intervalo abierto y $f \colon I \to \R$ es una función localmente
  lipschitziana. Para cada $x_0 \in I$ denotamos por $X(t; x_0)$ a la única solución maximal de
  \eqref{eq:edo:ae} que verifica $x(0) = x_0$.
\end{definition}

Nótese que $X(t; x_0)$ está bien definida gracias al Teorema de Picard-Lindelöf. De aquí en adelante
nos referiremos a \eqref{eq:edo:ae} cuando utilicemos EDOs autónomas escalares en un contexto
general.

\begin{lemma} \label{lem:ae:trans} Sea $\varphi \in \mathcal{C}^1(]\alpha, \omega[)$ solución de
  \eqref{eq:edo:ae} y sea $\tau > 0$. Entonces, la función $\Psi \colon ]\alpha-\tau, \omega-\tau[ \to \R$
  dada por $\Psi(t) = \varphi(t+\tau)$ es solución de \eqref{eq:edo:ae}.
\end{lemma}

\begin{definition}
  Se define la órbita de \eqref{eq:edo:ae} asociada a $x_0$ como el conjunto imagen de $X(t; x_0)$ y
  se denota $\Theta(x_0)$.
\end{definition}

Por el Lema \ref{lem:ae:trans} existen infinitas soluciones con la misma órbita.  Nótese que la
órbita asociada a $x_0$ es un intervalo, que solamente es trivial cuando la solución es constante,
lo que sucede si, y solo si, $f(x_0) = 0$. Además, es claro que
\[I = \bigcup_{x_0 \in I} \Theta(x_0).\]

Como consecuencia del Teorema de Picard-Lindelöf obtenemos fácilmente el siguiente resultado.

\begin{prop}
  Para cualesquiera $x_0, x_0' \in I$ o bien $\Theta(x_0) = \Theta(x_0')$ o bien
  $\Theta(x_0) \cap \Theta(x_0') = \emptyset$. Por tanto, Las órbitas de \eqref{eq:edo:ae}
  constituyen una partición de $I$.
\end{prop}
\begin{proof}
  Sean $x_0, x_0' \in I$ tales que $\Theta(x_0) \cap \Theta(x_0') \ne \emptyset$. Veamos que
  $\Theta(x_0) = \Theta(x_0')$. Denotamos $\varphi_1(t) = X(t; x_0)$ y $\varphi_2(t) = X(t;
  x_0')$. Existen $t_1$ y $t_2$ tales que $\varphi(t_1) = \varphi_2(t_2)$. Definimos
  $\phi(t) = \varphi_2(t + (t_2 - t_1))$ en el intervalo que corresponda. Nótese que $\phi$ es
  solución de \eqref{eq:edo:ae} y verifica $\phi(t_1) = \varphi_2(t_2) = \varphi_1(t_1)$. Por el
  Teorema de Picard-Lindelöf obtenemos que $\phi = \varphi_1$ y, por tanto,
  $\Theta(x_0) = \Theta(x_0')$.
\end{proof}

Como consecuencia, si $p$ es un punto de equilibrio de \eqref{eq:edo:ae}, esto es, $f(p) = 0$, y
$x_0 \ne p$, entonces $X(t, x_0) \ne p$ para todo $t \in (\alpha(x_0), \omega(x_0))$. Por tanto,
$\Theta(x_0) \subset (-\infty, p)$ o $\Theta(x_0) \subset (p, \infty)$.
  
\begin{cor}
  Las soluciones de \eqref{eq:edo:ae} o son constantes o son estrictamente monótonas. Por tanto, las
  órbitas o son conjuntos unitarios o son intervalos abiertos.
\end{cor}

Una representación de algunas órbitas de \eqref{eq:edo:ae} recibe el nombre de \emph{diagrama de
  fases}. Nótese el diagrama de fases viene determinado por el diagrama de signos de $f$. Un signo
positivo indica que la solución asociada a esa órbita es estrictamente creciente mientras que un
signo negativo indica que dicha solución es estrictamente decreciente.

\begin{lemma}[Barbalet, versión débil]
  \label{lem:barbalet}
  Sea $\varphi \in D(]\alpha, +\infty[)$ tal que $\lim_{t \to +\infty} \varphi(t) = L \in
  \R$. Entonces, existe $\{t_n\} \subset ]\alpha, +\infty[$ estrictamente creciente con
  $\{t_n\} \to +\infty$ tal que $\{f'(t_n)\} \to 0$.
\end{lemma}
\begin{proof}
  Es una consecuencia sencilla del teorema del valor medio.
\end{proof}

\begin{prop}
  Sea $\varphi \in \mathcal{C}^1(]\alpha, +\infty[)$ solución de \eqref{eq:edo:ae} tal que
  $\lim_{t \to +\infty} \varphi(t) = L \in \R$. Entonces, $p \in \mathrm{Z}_f$.
\end{prop}
\begin{proof}
  Es una consecuencia sencilla del Lema \ref{lem:barbalet}.
\end{proof}

\begin{cor}
  Sean $p_1$ y $p_2$ dos puntos de inflexión de \eqref{eq:edo:ae} consecutivos con $p_1 < p_2$. Sea
  $\varphi$ solución maximal de \eqref{eq:edo:ae} tal que su imagen es $(p_1, p_2)$. Entonces, el
  dominio de $\varphi$ es $\R$. A la solución $\varphi$ se le llama sigmoide.
\end{cor}

\begin{definition} [DIAGRAMA DE FASES]
\end{definition}

\subsection{Región de atracción}

\begin{definition}
  \label{def:ra}
  Sea $I \subset \R$ intervalo abierto, $f: I \to \R$ localmente lipschitziana y $p \in I$ tal que
  $f(p) = 0$. Se define la región de atracción de p como
  \[ \mathcal{R}(p) = \{x_0\in I: \omega(x_0) = +\infty \text{ y } \lim_{t \to +\infty} X(t;x_0) =
    p\}. \]
\end{definition}

Claramente tenemos que $p \in \mathcal{R}(x_0)$.

\begin{definition}
  \label{def:atractor}
  En el contexto de la Definición \ref{def:ra}:
  \begin{itemize}
  \item Si $p \in \mathrm{int}(\mathcal{R}(p))$, entonces diremos que $p$ es un atractor.
  \item Si el diagrama de fases es de la forma $\rightarrow p \leftarrow$, entonces diremos que $p$
    es un sumidero.
  \item Si $\mathcal{R}(p) = I$, entonces diremos que $p$ es un atractor global.
  \item Si el diagrama de fases es de la forma $\leftarrow p \rightarrow$, entonces diremos que $p$
    es una fuente.
  \item Si $\mathcal{R}(p) = \{p\}$ y $p$ es un cero aislado de $f$, entonces diremos que $p$ es un
    repulsor.  Si $\mathrm{int}(\mathcal{R}(p)) \ne \emptyset$ y $p \in \partial \mathcal{R}(p)$,
    entonces diremos que $p$ es un semiatractor.
  \end{itemize}
\end{definition}

\begin{prop}[Test de la derivada primera]
  Sean $I \subset \R$ intervalo abierto, $f \in \mathcal{C}^1(I)$ y $p\in I$ tal que $f(p) = 0$.
  
  \begin{enumerate}
  \item Si $f'(p) < 0$, entonces $p$ es un atractor de $x' = f(x)$.
  \item Si $f'(p) > 0$, entonces $p$ es un repulsor de $x' = f(x)$.
  \end{enumerate}
\end{prop}
\begin{proof}
  \begin{enumerate}
  \item En tal caso el diagrama de fases en $p$ es de la forma $\rightarrow p \leftarrow$, de donde
    se sigue el resultado.
  \item En tal caso el diagrama de fases en $p$ es de la forma $\leftarrow p \rightarrow$, de donde
    se sigue el resultado. \qedhere
  \end{enumerate}
\end{proof}

\begin{ex}[Ecuación logística]
  Sean $\lambda, L > 0$. Consideramos la EDO autónoma escalar
  \begin{equation}
    \label{eq:log}
    x' = \lambda x(L-x)
  \end{equation}
  La función $f(x) = \lambda x (x -L)$ definida en $\R$ es de clase $1$ y, por tanto, localmente
  lipschitziana. Tenemos que $\mathrm{Z}_f = \{0, L\}$. El diagrama de fases de la ecuación es
  \[-\infty \leftarrow 0 \rightarrow L \leftarrow +\infty.\] Por tanto, si $0 < x_0 < L$, entonces
  $\omega(x_0) = +\infty$ y $\lim_{t \to +\infty} X(t,x_0) = L$.
\end{ex}


\section{Prolongación y acotación de soluciones}

En este apartado estudiamos cómo prolongar soluciones de un PVI arbitrarios. Encontraremos
prolongaciones bajo ciertas hipótesis sobre la solución y / o el problema. En consecuencia, para las
soluciones maximales fallarán alguna de estas hipótesis, obteniendo información sobre estas
soluciones.

\begin{lemma}
  Sea $\varphi : ]\alpha, \omega[ \to \R^d$ una solución de \eqref{eq:pvi} tal que
  $\lim_{t \to \omega} \varphi(t) = \xi \in \R^d$. Si $(\omega, \xi) \in D$, entonces la solución es
  prolongable.
\end{lemma}
\begin{proof}
  Consideramos el PVI
  \[\begin{cases}x' = f(t, x); \\ x(\omega) = \xi. \end{cases}\]
  Existe $a > 0$ tal que $\widetilde{\varphi}: ]\omega-a, \omega+a[ \to \R^d$ es solución del nuevo
  PVI. La función $\Psi: ]\alpha, \omega+a[ \to \R^d$ dada por
  \[\Psi(t) = \begin{cases} \varphi(t) & \text{si } t \le \omega; \widetilde{\varphi}(t) & \text{si
      } t > \omega;\\\end{cases}\] es una solución de \eqref{eq:pvi} y, por tanto, es una
  prolongación de $\varphi$.
\end{proof}

\begin{lemma}\label{lem:prolongacion}
  Sea $\varphi : ]\alpha, \omega[ \to \R^d$ una solución de \eqref{eq:pvi} tal que existe una
  sucesión $\{t_n\} \subset ]\alpha, \omega[$ con $\{t_n\} \to \omega$ tal que
  $\{\varphi(t_n)\} \to \xi \in \R^d$. Si $(\omega, \xi) \in D$, entonces
  $\lim_{t \to \omega} \varphi(t) = \xi$ y, además, la solución es prolongable.
\end{lemma}
\begin{proof}
  Sean $a, b \in \R^+$ tales que $R_{a,b}(\omega, \xi) \subset D$. Existe
  $M = \max \{||f(t,x)||: (t,x) \in R_{a,b}(\omega, \xi)\}$. Dado $0 < \varepsilon < b$, tomamos
  $\delta > 0$ tal que $\delta < \omega - \alpha$, $\delta M < \varepsilon / 2$. Consideramos la
  región $R_{\delta, \varepsilon}(\omega, \xi)$. Existe $N \in \N$ tal que
  $t_N \in ]\omega-\delta, \omega[$ y $\varphi(t_N) \in \overline{\mathrm{B}}(\xi,
  \varepsilon/2)$. Por tanto, $(t_N, \varphi(t_N)) \in R_{\delta, \varepsilon}(\omega, \xi)$. Vamos
  a demostrar que $(t, \varphi(t)) \in R_{\delta, \varepsilon}(\omega, \xi)$ para todo
  $t \in [t_N, \omega[$. En efecto, tenemos que para cada $t \in [t_N, \omega[$ se verifica
  \[\varphi(t) = \varphi(t_N) + \int_{t_N}^{t} \varphi'(s) \diff s = \int_{t_N}^{t}
    f(s,\varphi(s)) \diff s.\] Consecuentemente, obtenemos que
  \[||\varphi(t)-\xi|| \le ||\varphi(t_N) - \xi|| + ||\int_{t_N}^{t} f(s,\varphi(s)) \diff
    s|| \le \varepsilon / 2 + ||\int_{t_N}^{t} f(s,\varphi(s)) \diff s||.\]

  Supongamos que existe $\tau \in ]t_N, \omega[$ tal que $||\varphi(t) - \xi|| < \varepsilon$ para
  todo $t \in ]t_N, \tau[$ pero $||\varphi(\tau) - \xi|| = \varepsilon$. En tal caso, tenemos que
  \[\varepsilon = ||\varphi(\tau) - \xi|| < \varepsilon / 2 + ||\int_{t_N}^{\tau}
    f(s,\varphi(s)) \diff s|| \le \varepsilon / 2 + |\tau - t_N| < \varepsilon/2 + \varepsilon/2 =
    \varepsilon,\]

  contradicción. Por tanto, se tiene $\varphi(t) \in \overline{\mathrm{B}}(\xi, \varepsilon)$ para
  todo $t \in ]t_N, \omega[$ como se quería.
\end{proof}

\begin{theorem}[Comportamiento de la soluciones maximales en el extremo superior]
  \label{thm:prolongacion}
  Sea $\varphi : ]\alpha, \omega[ \to \R^d$ una solución maximal de \eqref{eq:pvi}. Si
  $\omega < +\infty$, entonces se ha de cumplir una de las siguientes alternativas:
  \begin{enumerate}
  \item\label{item:prolongacion:1} $\lim_{t \to \omega} ||\varphi(t)|| = +\infty$;
  \item\label{item:prolongacion:2} Existe $\xi \in \R^d$ con $(\omega, \xi) \in \partial D$ y existe
    $\{t_n\} \subset ]\alpha, \omega[$ estrictamente creciente tal que $\{t_n\} \to \omega$ y
    $\{\varphi(t_n)\} \to \xi$.
  \end{enumerate}
\end{theorem}
\begin{proof}
  Supongamos que no se cumple \ref{item:prolongacion:1}. En tal caso encontramos una sucesión
  $\{t_n\}$ convergente a $\omega$ tal que $\{\varphi(t_n)\}$ está acotada. Por el teorema de
  Bolzano-Weierstrass encontramos una parcial $\{t_{\sigma(n)}\}$ convergente a $\xi \in
  \R^d$. Nótese que $(\omega, \xi) \not \in D$ por el Lema \ref{lem:prolongacion}. Puesto que
  $\{(t_{\sigma(n)}, \varphi(t_{\sigma(n)}))\} \to (\omega, \xi)$ tenemos que
  $(\omega, \xi) \in \partial D$.
\end{proof}

\begin{definition}
  Sea $\varphi: ]\alpha, \omega[ \to \R^d$ continua. El conjunto $\omega$-límite de \eqref{eq:pvi}
  se define como
  \[L_{\omega}(\varphi) = \{q \in \R^d \mid \exists \{t_n\} \subset ]\alpha, \omega[, \{t_n\} \to
    \omega, \{\varphi(t_n)\} \to q\}.\]
\end{definition}

\begin{remark}
  Nótese que en el Teorema \ref{thm:prolongacion} la alternativa \ref{item:prolongacion:2} es
  equivalente a que $\{\omega\} \times L_{\omega}(\varphi) \cap \partial D \ne \emptyset$.
\end{remark}

\begin{ex}
  Consideremos el PVI
  \[
    \begin{cases}
      x' = x^2 + t; \\
      x(0) = 0.
    \end{cases}
  \]
  Sea $\varphi: ]\alpha, \omega[ \to \R$ solución. Si $\omega < +\infty$, entoces verifica que
  $\lim_{t \to \omega} \varphi(t) = + \infty$, luego $L_{\omega}(\varphi) = \emptyset$.
\end{ex}

\begin{corollary}
  Sean $D = \R \times \R^d$ y sea $\varphi: ]\alpha, \omega[ \to \R^d$ solución maximal de
  \eqref{eq:pvi}.  Si $\omega < +\infty$, entoces se verifica que
  $\lim_{t \to \omega} ||\varphi(t)|| = + \infty$.
\end{corollary}

\begin{corollary}
  Si el conjunto D es acotado y $\varphi: ]\alpha, \omega[ \to \R^d$ solución maximal de
  \eqref{eq:pvi}, entonces $\omega < +\infty$ y además existe $\xi \in \R^d$ con
  $(\omega, \xi) \in \partial D$ y $\{t_n\} \subset ]\alpha, \omega[$ estrictamente creciente tal
  que $\{t_n\} \to \omega$ y $\{\varphi(t_n)\} \to \xi$.
\end{corollary}


\begin{corollary} \label{cor:ab:lim} Sean $D = ]a,b[ \times \R^d$ con
  $-\infty \le a < b \le +\infty$ y $\varphi: ]\alpha, \omega[ \to \R^d$ una solución maximal de
  \eqref{eq:pvi}. Si $\omega < b$, entonces $\lim_{t \to \omega} \varphi(t) = +\infty$.
\end{corollary}

\begin{ex}
  Sea $]\alpha, \omega[$ el intervalo de definición de la solución maximal de
  
  \begin{equation}
    \label{eq:ex:pi}
    \begin{cases}
      x' = \frac{2}{x} + \frac{\sin(t)}{x^2}, \quad x > 0; \\
      x(0) = 2.
    \end{cases}
  \end{equation}
  Demuestra que $\omega = +\infty$.

  Denotemos $\varphi(t) = X(t; 0,2)$. Vamos a demostrar que $1 < \varphi(t) \le 2+3t$ para todo
  $t \ge 0$. Razonemos por reducción al absurdo. Si $1 > \varphi(t')$ para cierto $t' > 0$, podemos
  encontrar $\tau = \inf \{t > 0: \varphi(t) < 1\} > 0$. Nótese que como $\varphi(t)$ es contínua
  debe tenerse $\varphi(\tau) = 1$ y $\varphi(t) > 1$ para todo $0 < t < \tau$. Como consecuencia,
  debe tenerse $\varphi'(\tau) \le 0$. Sin embargo, usando que $\varphi$ es solución de
  \eqref{eq:ex:pi} se tiene que $\varphi'(\tau) = 2 + \sin(\tau) > 0$, contradicción. Utilizando
  esta información y \eqref{eq:ex:pi} deducimos que
  \[\varphi'(t) \le \varphi'(t) \varphi(t) = 2 + \frac{\sin(t)}{\varphi(t)} \le 3.\]

  Por tanto, deducimos que
  \[ \varphi(t) = 2 + \int_{0}^{t} \varphi'(s) \diff s \le 2 +3t.\]

  Por último, la cota que hemos encontrado impide que se verifiquen las alternativas del Teorema
  \ref{thm:prolongacion}, esto es, $\omega = +\infty$.
\end{ex}

\begin{lemma}[Lema de Gronwall]
  Sea $y:[t_0, \omega[ \to \R$ continua tal que existen dos constantes $C, R \in \R$ con $R \ge 0$
  tales que para cada $t \in [t_0, \omega[$ se verifica
  \[ y(t) \le C + R \int_{t_0}^{t} y(s) \diff s. \] Entonces $y(t) \le C e^{R(t-t_0)}$ par
  todo $t \in [t_0, \omega[$.
\end{lemma}
\begin{proof}
  De la tesis deducimos en particular que $y'(t) \le R y(t)$ para todo $t \in [t_0, \omega[$. Si
  $y = 0$, entonces la tesis es clara. En caso contrario, para cada $t' \in [t_0, \omega[$ existe un
  intervalo $t' \in I = [t_1, t_2] \subset [t_0, \omega[$ tal que $y(t) \ne 0$ para todo $t \in
  I$. En consecuencia, encontramos que para cada $t \in I$ se verifica
  \[ \log(y(t)) - \log(y(t_1)) = \int_{t_1}^{t} \frac{y'(s)}{y(s)} \diff s \le R(t-t_1).\]

  De la monotonía de la función exponencial se deduce que...

\end{proof}

\begin{definition}
  Sea $D = ]a,b[ \times R^2$ y $f: ]a,b[\times\R^d$. Diremos que $f$ tiene a lo sumo crecimiento
  lineal en la variable $x$ si existen dos funciones continuas $m, n : ]a,b[ \to \R^+_0$ tales que
  para cada $(t,x) \in D$ se tiene
  \[ ||f(t,x)|| \le m(t) ||x|| + n(t).\] En tal caso escribiremos que $f$ es CSL.
\end{definition}

\begin{remark}
  Todas las funciones definidas en $D$ que sean globalmente lipschitzianas respecto de la variable
  $x$ son CSL.
\end{remark}

\begin{remark}
  Sea $f : \R \times \R$ continua tal que existen $\lim_{x \to +\infty} f(x)/x$ y
  $\lim_{x \to -\infty} f(x)/x$. Entonces $f$ tiene crecimiento a lo sumo lineal.
\end{remark}

\begin{theorem}[Teorema de crecimiento a lo sumo lineal]
  Si $f$ tiene crecimiento a lo sumo lineal, entonces todas las soluciones maximales de
  \eqref{eq:pvi} verifican que $]\alpha, \omega[ = ]a,b[$.
\end{theorem}
\begin{proof}
  Sean $m, n : ]a,b[ \to \R_0^+$ continuas de la deficinicón de crecimiento a lo umo lineal. Sea
  $\varphi: ]\alpha, \omega[ \to \R^d$ solución de \eqref{eq:pvi}. Veamos que $\omega =
  b$. Supongamos por reducción al absurdo que $\omega < b$. Tenemos que
  $[t_0, \omega] \subset ]a,b[$. Por el teorema de Weierstrass existe
  $M = \max \{m(t) \mid t \in [t_0,\omega]\}$ y $N = \max \{n(t) \mid t \in [t_0,\omega]\}$ Por
  tanto, se cumple que
  \[ ||\varphi(t)|| \le ||x_0|| + \int_{t_0}^{t} ||f(s, \varphi(s))|| \diff s \le ||x_0|| +
    \int_{t_0}^{t} M||\varphi(s)|| + N \diff s ) \le ||x_0|| + N(\omega - t_0) + M \int_{t_0}^{t}
    ||\varphi(s)|| \diff s. \]

  Podemos aplicar pues el Lema de Gronwall, obteniendo que
  \[||\varphi(t)|| \le (||x_0|| + N (\omega-t_0)) \exp(M(t - t_0)) \le (||x_0|| + N (\omega-t_0))
    \exp(M(\omega - t_0)),\]

  y, por tanto, $\varphi$ está acotada, lo que contradice el Corolario \ref{cor:ab:lim}. Para
  demostrar que $\alpha = a$ basta aplicar el argumento anterior sobre el PVI dual, que también es
  CSL.
\end{proof}

\begin{ex}
  Sea $x_0 \in \R$. Consideremos el PVI
  \[
    \begin{cases}
      x' = \frac{x^3}{x^2+1} & x \in \R; \\
      x(0) = x_0.
    \end{cases}
  \]
  Es autónoma escalar y su diagrama de fases se corresponde con $\leftarrow 0
  \rightarrow$. Consecuentemente, $\alpha(0,x_0) = -\infty$. Por el Teorema previo obtenemos que
  $\omega(0,x_0) = +\infty$.
\end{ex}

\begin{remark}
  Las EDOs lineales son CSL. En efecto, éstas son de la forma
  \[ x' = A(t) x + b(t), \] con $A \in \mathcal{C}(\R, \mathcal{M}_{d}(\R))$ y
  $b \in \mathcal{C}(\R, \R^d)$. En efecto, tenemos que
  \[ ||A(t)x +b(t)|| \le ||A(t)x||+||b(t)|| \le ||A(t)|| \,||x|| + b(t)||. \]
\end{remark}

\begin{remark}
  Veremos en una relación de problemas que cuando $f(x)$ tiene un crecimiento mayor que una función
  potencial $x^p$ con $p > 1$, entonces ciertas soluciones explotan en tiempo finito.
\end{remark}

Las hipótesis del Teorema CSL no son óptimas ya que hay funciones con crecimiento super lineal tales
que las soluciones de las EDOs asociadas no explotan en tiempo finito. Esto queda reflejado en el
siguiente ejemplo.

\begin{ex}
  Consideramos el PVI
  \[
    \begin{cases}
      x' = x \log(x), \quad x > 0; \\
      x(0) = 2.
    \end{cases}
  \]
  Nótese que $\varphi(t) = X(t, 2) = \exp(\log(2) e^t)$ y $\alpha = -\infty$ y $\omega = +\infty$.
\end{ex}

\begin{remark}
  La hipótesis del Teorema de CSL se puede cambiar por las siguientes:
  
\begin{enumerate}
\item Existe $A \subset \R^d$ tal que $\varphi(t) \in A$ para todo $t \in ]\alpha, \omega[$.
\item Existen $m, n: ]a,b[ \to \R_0^+$ continuas tales que $||f(t,x)|| \le m(t) ||x|| + n(t)$ para todo $t \in ]a,b[$ y $x \in A$.
\end{enumerate}
\end{remark}


\subsection{Acotación de soluciones. Funciones guía}

Consideramos el conjunto abierto $D = ]a, +\infty[ \times \R^d$, con $-\infty \le a$. Sea
$f: D \to \R^d$ una función continua y sea $(t_0, x_0) \in D$. Consideramos el PVI \eqref{eq:pvi}.

\begin{definition}
  Sea $\varphi: ]\alpha, \omega[ \to \R^d$ una solución maximal de \eqref{eq:pvi}. Diremos que la
  solución $\varphi$ está acotada en el futuro si $\omega = +\infty$ y
  $\sup \{||\varphi(t)||: t \in [t_0, +\infty[\} < +\infty$. Análogamente, diremos que la solución
  $\varphi$ está acotada en el pasado si $\alpha = -\infty$ y
  $\sup \{||\varphi(t)||: t \in ]-\infty, t_0]\} < +\infty$.
\end{definition}

\begin{proposition}
  Supongamos que $d = 1$. Sea $\varphi: ]\alpha, \omega[ \to \R$ una solución maximal de
  \eqref{eq:pvi}. Si existen $p_1 < p_2$ tales que $x_0 \in ]p_1, p_2[$, $f(t, p_1) > 0$ y
  $f(t, p_2) < 0$, entonces $\varphi(t) \in ]p_1, p_2[$ para todo $t \ge t_0$ y, por tanto, la
  solución es acotada en el futuro.
\end{proposition}
\begin{proof}
  Veamos que $\varphi(t) > p_1$ para todo $t \ge t_0$. En efecto, $\varphi(t_0) = x_0 >
  p_1$. Supongamos que el resultado es falso. Existe $\tau > t_0$ tal que $\varphi(t) > p_1$ para
  todo $t \in [t_0, \tau[$ y $\varphi(\tau) = p_1$. Por tanto, $\varphi'(\tau) \le 0$. Sin embargo,
  $\varphi'(\tau) = f(\tau, p_1) > 0$, contradicción. Análogamente se prueba la segunda afirmación.
\end{proof}

\begin{definition}
  Consideremos una EDO \eqref{eq:edo}.  Sea $V : \R^d \to \R$ tal que $V \in
  \mathcal{C}^1(\R^d)$. Se define $\dot{V}: D \to \R$ como
  $\dot{V}(t,x) = \langle \nabla V(x), f(t,x)\rangle$.  Se dice que $V$ es una función guía de la
  EDO si $\dot{V}$ tiene signo constante (para valores $x$ suficientemente grandes).
\end{definition}

Sea $\varphi: ]\alpha, \omega[ \to \R^d$ una solución maximal de \eqref{eq:pvi}. Nótese que la
función $V(\varphi(t))$ es derivable y su derivada viene dada por
\[ \frac{\partial}{\partial t} V(\varphi(t)) = \langle \nabla V(\varphi(t)), \varphi'(t)\rangle =
  \dot{V}(t, \varphi(t)). \]


\begin{definition}
  Sea $V: \R^d \to \R$ continua y $r \in \R$, se definen el conjunto de nivel $r$
  \[C_r = V^{-1}(r) = \{x \in \R^d: V(x) = r\}\]

  y el conjunto de subnivel $r$
  \[\Omega_r = \bigcup_{s \le r} C_r(s) = V^{-1}(]-\infty,r]).\]

  Diremos que $V$ es una función coerciva si todos sus conjuntos de subnivel son compactos.
\end{definition}


\begin{prop}
  En el contexto actual, se verifican las siguientes afirmaciones:
  \begin{itemize}
  \item Si $d = 1$, $V$ es coerciva si, y solo si, $\lim_{x \to \pm \infty}V(x) = +\infty$.
  \item Si $V(x_1, \ldots, x_d) = V_1(x_1)+ \cdots + V_d(x_d)$, entonces es coerciva si, y solo si,
    todos los sumandos $V_i$ son coercivas en $\R$.
  \item Si $V(x) \ge W(x)$ para todo $x \in \R^d$ y $W$ es coerciva, entonces $V$ es coerciva.
  \item Si $V$ es una función cuadrática (polinómica de grado $2$), entonces es coerciva si, y solo
    si, $\mathrm{Hess} V$ es una matriz definida positiva.
  \item Si $V \in \mathcal{C}^1(\R^d)$ es una función convexa y tiene un único punto crítico,
    entonces $V$ es coerciva.
  \end{itemize}
\end{prop}

\begin{theorem}
  \label{thm:guia}
  Si existe $V\in \mathcal{C}^1(\R^d)$ coerciva tal que $\dot{V}(t,x) \le 0$ para todo
  $(t,x) \in D$, entonces las soluciones maximales de \eqref{eq:pvi} están acotadas en el futuro.
\end{theorem}
\begin{proof}
  Sea $\varphi: ]\alpha, \omega[ \to \R^d$ una solución maximal de \eqref{eq:pvi}. Definimos
  $y = V \circ \varphi$. Tenemos que $y'(t) = \dot{V}(t, \varphi(t)) \le 0$, luego $y$ es
  decreciente. Por tanto, $V(\varphi(t)) \le V(\varphi(t)_0) = r_0$ para todo $t \ge t_0$. Por
  tanto, $\varphi([t_0, \omega[) \subset \Omega_{r_0}$ por ser $V$ coerciva. Como $\Omega_{r_0}$ es
  compacto, la función $\varphi$ está acotada en $[t_0, \omega[$. Consecuentemente, del Teorema de
  prolongación de soluciones se deduce que $\omega = +\infty$.
\end{proof}

\begin{corollary}
  Si existe $V\in \mathcal{C}^1(\R^d)$ coerciva tal que $\dot{V}(t,x) = 0$ para todo $(t,x) \in D$,
  entonces las soluciones maximales de \eqref{eq:pvi} están definidas en $]-\infty, +\infty[$ y
  están acotadas. Además, $\varphi(t) \in C_{V(x_0)}$ para todo $t \in \R$.
\end{corollary}

\begin{remark}
  Las hipótesis del Teorema \ref{thm:guia} se pueden debilitar. Concretamente, la afirmación
  $\dot{V}(t,x) \le 0$ para todo $(t,x) \in D$ puede substituirse por $\dot{V}(t,x) \le 0$ para todo
  $t \in ]t_0, +\infty[$, $x \in \R^d - B(p, r)$ para algunos $p \in \R^d$ y $ r > 0$. Además, en
  lugar de suponer que $V$ es coerciva basta suponer que las componenetes conexax de los conjuntos
  de subnivel son acotadas. El siguiente resultado es una versión más general del Teorema
  \ref{thm:guia}.
\end{remark}

\begin{theorem}
  \label{thm:guia:2}
  Si existe $V\in \mathcal{C}^1(\R^d)$ coerciva y existe $r \in \R$ tal que el conjunto
  \[ \Omega_r = \{x \in \R^d \mid V(x) \le r\} \]

  tiene componentes conexas acotadas y además $\dot{V}(t,x) \le 0$ para todo $t \ge
  t_0$,$x \in \Omega_r$, entonces las soluciones maximales de \eqref{eq:pvi} están acotadas en el
  futuro para $(t_0, x_0) \in ]a, +\infty[ \times \Omega_r$.
\end{theorem}


\begin{ex}[Relación de ejercicios 3, ejercicio 8] 
  Busca funciones guía que nos permitan asegurar que las soluciones de las siguientes ecuaciones
  diferenciales están acotadas en el futuro:
  \begin{enumerate}
  \item $x' = \sin(t) - x^{3}$. Es fácil ver que $V(x) = x^2$ es una función guía.
  \item Estudiamos el sistema
    \[
      \begin{cases}
        x_1' = x_2; \\
        x_2' = -\sin(x_1)-x_1.
      \end{cases}
    \]

    Buscamos una función guía de la forma $V(x_1, x_2) = V_1(x_1) + V_2(x_2)$. En tal caso se
    tendría
    $\dot{V}(x_1, x_2) = \left\langle (V_1'(x_1), V_2'(x_2)), (x_2, -\sin(x_1) - x_1) \right\rangle
    = V_1'(x_1) x_2 - V_2'(x_2)(\sin(x_1) + x_1)$. Bastaría que $V_1'(x_1) = \sin(x_1) + x_1$ y que
    $V_2'(x_2) = x_2$. Esto sucede para $V_1(x_1) = - \cos(x_1) + x_1^2/ 2$ y
    $V_2(x_2) = x_2^2 / 2$. Nótese que estas funciones son coercivas. Por tanto, $V$ es una función
    guía.
  \item Estudiamos el sistema
    \[
      \begin{cases}
        x_1' = -x_1 + x_2; \\
        x_2' = -x_2.
      \end{cases}
    \]
  \end{enumerate}
  Vamos a comprobar que $V(x_1, x_2) = x_1^2 + x_2^2$ es una función guía. Tenemos que
  $\dot{V}(x_1, x_2) = 2x_1(-x_1+x_2) 2x_2(-x_2) = - x_1^2 - x_2^2 - (x_1-x_2)^2 \le 0$, que es una forma
  definida negativa.
\end{ex}


\end{document}


